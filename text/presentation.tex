\documentclass{beamer}
\usepackage[utf8]{inputenc}
\usepackage{tikz}
\usetikzlibrary{intersections,positioning,calc,arrows,shapes,
                decorations.pathreplacing,spy,automata}
\usepackage{pgfplots}
\usepackage[ruled]{algorithm2e}
\usetheme{boxes}
\title[Fast Solver of Closely Related Quadratic Programming Problems]
      {Fast Solver of Closely Related Quadratic Programming Problems}
\author{Andreas Halle}
%\institute{Department of Informatics}
\date{June 10, 2013}


\definecolor{tleblue}{RGB}{178,175,255}
\DeclareMathOperator*{\argmin}{arg\,min}


\begin{document}



\begin{frame}
\titlepage
\end{frame}



\section{Introduksjon}



\begin{frame}{PROMAPS}
\begin{itemize}
\item Utviklet av Goodtech og MathConsult.
\item Kalkurerer leveransepåliteligheten i et nettverk
\item Dette er formulert som en rekke veldig like QP-problemer
      (Quadratic Programming).
\item QP-løseren en flaskehals. Skjermbildet (PROMAPS) oppdaterer seg hvert
      femte minutt.
\item \url{http://www.tu.no/energi/2011/10/07/her-beregnes-risikoen-for-svikt-i-kraftnettet}
\end{itemize}
\end{frame}



%\begin{frame}{QP-problemet}
%\begin{itemize}
%\item Objektfunksjonen representerer kostnader for strømleveranse.
%\item Sidekravene representerer nettverket, hvor summen av flyten i en
%      node er lik $0$.
%\item Mer spesifikt...
%\end{itemize}
%\end{frame}



\begin{frame}{Objektfunksjonen}
\[
f(x) = x^T \Phi D x + (g - c)^T x
\]
\begin{itemize}
\item $f(x)$ representerer leveransekostnader ($E/s$)
\item $x$ representerer strømmen over grenene ($W$)
\item $\Phi$ representerer strømtap ($1/W$)
\item $D$ representerer overføringskostnader ($E/J$)
\item $g$ representerer kostnader for å generere strøm ($E/J$)
\item $c$ representerer leveransepris ($E/J$)
\end{itemize}
\end{frame}



\begin{frame}{Objektfunksjonen}
\[
f(x) = x^T H x + b^T x
\]
\begin{itemize}
\item $H = \Phi D$ representerer kostnader ($E/(W^2 s)$)
\item $b = g - c$ representerer kostnader ($E/J$)
\item $H$ er en matrise på størrelsen $n \times n$
\item $b$ og $x$ er vektorer i $\mathbb{R}^n$
\end{itemize}
\end{frame}



\begin{frame}{Optimeringsproblemet}
Vi definerer et konveks QP-problem
\[
\min_{x \in \mathbb{R}^n} f(x) \quad \textrm{subject to} ~ Ax = 0, ~ l \leq x \leq u
\]
\begin{itemize}
\item A er en $m \times n$ insidensmatrise for et strømnettverk
\item $m$ noder
\item $n$ grener
\item $l$ og $u$ er nedre og øvre grenkapasitet ($W$)
\end{itemize}
\end{frame}



\begin{frame}{Utfall}
Vi ønsker å modellere utfall.
\begin{itemize}
\item Setter $l_i = u_i = 0$
\item Goodtech ønsker å løse QP-problemet for ulike utfall
\item Definerer subinstanser for kombinasjoner av grener som faller ut
\item Instans er et QP-problem uten utfall
\item Subinstans er en instans med utfall
\end{itemize}
\end{frame}



\begin{frame}{Utfall}
\begin{itemize}
\item Vil løse så mange subinstanser som mulig.
\item Usannsynlig at det er mange utfall.
\item Vi prøver å løse alle subinstanser som har mindre eller lik $\beta$ utfall.
\end{itemize}
\[
\sigma (\beta, n) = \sum_{j=0}^{\beta} \binom{n}{j}
\]
\end{frame}



\begin{frame}{Subinstanser}
\begin{itemize}
\item En mengde av variabler som representerer utfall $\mathcal{M}_k$
\item $k = \sum_{j \in \mathcal{M}_k}^{} 2^{j-1}$
\item Eks. $\mathcal{M}_k = \left\{1,3,5 \right\}$. $k = 21$
\item En subinstans $\mathcal{Q}_k$ defineres av $\mathcal{M}_k$
\item Optimal løsning til subinstans $\mathcal{Q}_k$ noterer vi som $x_k^*$
\end{itemize}
\end{frame}



\begin{frame}{Instanser fra Goodtech}
\begin{tikzpicture}
\begin{axis}[
%    xlabel={Column},
%    ylabel={Value},
    scatter/classes={
    a={mark=square*,blue},%
    b={mark=triangle*,red},%
    c={mark=o,draw=black}}]
%    b={mark=o,draw=black},
%    a={mark=o,draw=black}}]
    % \addplot[] is better than \addplot+[] here:
    % it avoids scalings of the cycle list

    \legend{\textit{small},\textit{large},\textit{vlarge}}
    \addplot[scatter,only marks,
             scatter src=explicit symbolic]
             coordinates {
(15, 0.013704) [a]
(16, 0.004299) [a]
(17, 0.014245) [a]
(18, 0.001473) [a]
(19, 0.008387) [a]
(20, 0.004771) [a]
(21, 0.003282) [a]
(22, 0.003282) [a]
(23, 0.007019) [a]
(24, 0.005940) [a]
(25, 0.009227) [a]
(26, 0.004674) [a]
(27, 0.002725) [a]
(28, 0.004941) [a]
(29, 0.005775) [a]
(30, 0.002207) [a]
(31, 0.005586) [a]
(32, 0.004039) [a]
(33, 0.006443) [a]
(34, 0.003821) [a]
(35, 0.000862) [a]
(36, 0.000643) [a]
(37, 0.005352) [a]
(38, 0.001233) [a]
(39, 0.006597) [a]
(40, 0.005749) [a]
(41, 0.001600) [a]
(42, 0.006202) [a]
(43, 0.000629) [a]
(44, 0.013353) [a]
(45, 0.025716) [a]
(46, 0.025716) [a]
(47, 0.027476) [a]
(48, 0.000099) [a]
(49, 0.000099) [a]
(50, 0.011142) [a]
(51, 0.008257) [a]
(89, 0.003500) [a]
(90, 0.059229) [a]
(91, 0.013223) [a]
(92, 0.022727) [a]
(93, 0.019726) [a]
(94, 0.023370) [a]
(95, 0.012250) [a]
(96, 0.004121) [a]
(97, 0.016391) [a]
(98, 0.021051) [a]
(99, 0.010078) [a]
(100, 0.004017) [a]
(101, 0.028020) [a]
(102, 0.006028) [a]
(103, 0.040845) [a]
(104, 0.011336) [a]
(134, 0.013704) [a]
(135, 0.004299) [a]
(136, 0.014245) [a]
(137, 0.001473) [a]
(138, 0.008387) [a]
(139, 0.004771) [a]
(140, 0.003282) [a]
(141, 0.003282) [a]
(142, 0.007019) [a]
(143, 0.005940) [a]
(144, 0.009227) [a]
(145, 0.004674) [a]
(146, 0.002725) [a]
(147, 0.004941) [a]
(148, 0.005775) [a]
(149, 0.002207) [a]
(150, 0.005586) [a]
(151, 0.004039) [a]
(152, 0.006443) [a]
(153, 0.003821) [a]
(154, 0.000862) [a]
(155, 0.000643) [a]
(156, 0.005352) [a]
(157, 0.001233) [a]
(158, 0.006597) [a]
(159, 0.005749) [a]
(160, 0.001600) [a]
(161, 0.006202) [a]
(162, 0.000629) [a]
(163, 0.013353) [a]
(164, 0.025716) [a]
(165, 0.025716) [a]
(166, 0.027476) [a]
(167, 0.000099) [a]
(168, 0.000099) [a]
(169, 0.011142) [a]
(170, 0.008257) [a]
(186, 0.018453) [a]
(189, 0.018453) [a]
(208, 0.003500) [a]
(209, 0.059229) [a]
(210, 0.013223) [a]
(211, 0.022727) [a]
(212, 0.019726) [a]
(213, 0.023370) [a]
(214, 0.012250) [a]
(215, 0.004121) [a]
(216, 0.016391) [a]
(217, 0.021051) [a]
(218, 0.010078) [a]
(219, 0.004017) [a]
(220, 0.028020) [a]
(221, 0.006028) [a]
(222, 0.040845) [a]
(223, 0.011336) [a]
(15, 0.013704) [b]
(16, 0.004299) [b]
(17, 0.014245) [b]
(18, 0.001473) [b]
(19, 0.008387) [b]
(20, 0.004771) [b]
(21, 0.003282) [b]
(22, 0.003282) [b]
(23, 0.007019) [b]
(24, 0.005940) [b]
(25, 0.009227) [b]
(26, 0.004674) [b]
(27, 0.002725) [b]
(28, 0.004941) [b]
(29, 0.005775) [b]
(30, 0.002207) [b]
(31, 0.005586) [b]
(32, 0.004039) [b]
(33, 0.006443) [b]
(34, 0.003821) [b]
(35, 0.000862) [b]
(36, 0.000643) [b]
(37, 0.005352) [b]
(38, 0.001233) [b]
(39, 0.006597) [b]
(40, 0.005749) [b]
(41, 0.001600) [b]
(42, 0.006202) [b]
(43, 0.000629) [b]
(44, 0.013353) [b]
(45, 0.025716) [b]
(46, 0.025716) [b]
(47, 0.027476) [b]
(48, 0.000099) [b]
(49, 0.000099) [b]
(50, 0.011142) [b]
(51, 0.008257) [b]
(89, 0.003500) [b]
(90, 0.059229) [b]
(91, 0.013223) [b]
(92, 0.022727) [b]
(93, 0.019726) [b]
(94, 0.023370) [b]
(95, 0.012250) [b]
(96, 0.004121) [b]
(97, 0.016391) [b]
(98, 0.021051) [b]
(99, 0.010078) [b]
(100, 0.004017) [b]
(101, 0.028020) [b]
(102, 0.006028) [b]
(103, 0.040845) [b]
(104, 0.011336) [b]
(134, 0.013704) [b]
(135, 0.004299) [b]
(136, 0.014245) [b]
(137, 0.001473) [b]
(138, 0.008387) [b]
(139, 0.004771) [b]
(140, 0.003282) [b]
(141, 0.003282) [b]
(142, 0.007019) [b]
(143, 0.005940) [b]
(144, 0.009227) [b]
(145, 0.004674) [b]
(146, 0.002725) [b]
(147, 0.004941) [b]
(148, 0.005775) [b]
(149, 0.002207) [b]
(150, 0.005586) [b]
(151, 0.004039) [b]
(152, 0.006443) [b]
(153, 0.003821) [b]
(154, 0.000862) [b]
(155, 0.000643) [b]
(156, 0.005352) [b]
(157, 0.001233) [b]
(158, 0.006597) [b]
(159, 0.005749) [b]
(160, 0.001600) [b]
(161, 0.006202) [b]
(162, 0.000629) [b]
(163, 0.013353) [b]
(164, 0.025716) [b]
(165, 0.025716) [b]
(166, 0.027476) [b]
(167, 0.000099) [b]
(168, 0.000099) [b]
(169, 0.011142) [b]
(170, 0.008257) [b]
(186, 0.018453) [b]
(189, 0.018453) [b]
(208, 0.003500) [b]
(209, 0.059229) [b]
(210, 0.013223) [b]
(211, 0.022727) [b]
(212, 0.019726) [b]
(213, 0.023370) [b]
(214, 0.012250) [b]
(215, 0.004121) [b]
(216, 0.016391) [b]
(217, 0.021051) [b]
(218, 0.010078) [b]
(219, 0.004017) [b]
(220, 0.028020) [b]
(221, 0.006028) [b]
(222, 0.040845) [b]
(223, 0.011336) [b]
(253, 0.013704) [b]
(254, 0.004299) [b]
(255, 0.014245) [b]
(256, 0.001473) [b]
(257, 0.008387) [b]
(258, 0.004771) [b]
(259, 0.003282) [b]
(260, 0.003282) [b]
(261, 0.007019) [b]
(262, 0.005940) [b]
(263, 0.009227) [b]
(264, 0.004674) [b]
(265, 0.002725) [b]
(266, 0.004941) [b]
(267, 0.005775) [b]
(268, 0.002207) [b]
(269, 0.005586) [b]
(270, 0.004039) [b]
(271, 0.006443) [b]
(272, 0.003821) [b]
(273, 0.000862) [b]
(274, 0.000643) [b]
(275, 0.005352) [b]
(276, 0.001233) [b]
(277, 0.006597) [b]
(278, 0.005749) [b]
(279, 0.001600) [b]
(280, 0.006202) [b]
(281, 0.000629) [b]
(282, 0.013353) [b]
(283, 0.025716) [b]
(284, 0.025716) [b]
(285, 0.027476) [b]
(286, 0.000099) [b]
(287, 0.000099) [b]
(288, 0.011142) [b]
(289, 0.008257) [b]
(327, 0.003500) [b]
(328, 0.059229) [b]
(329, 0.013223) [b]
(330, 0.022727) [b]
(331, 0.019726) [b]
(332, 0.023370) [b]
(333, 0.012250) [b]
(334, 0.004121) [b]
(335, 0.016391) [b]
(336, 0.021051) [b]
(337, 0.010078) [b]
(338, 0.004017) [b]
(339, 0.028020) [b]
(340, 0.006028) [b]
(341, 0.040845) [b]
(342, 0.011336) [b]
(372, 0.013704) [b]
(373, 0.004299) [b]
(374, 0.014245) [b]
(375, 0.001473) [b]
(376, 0.008387) [b]
(377, 0.004771) [b]
(378, 0.003282) [b]
(379, 0.003282) [b]
(380, 0.007019) [b]
(381, 0.005940) [b]
(382, 0.009227) [b]
(383, 0.004674) [b]
(384, 0.002725) [b]
(385, 0.004941) [b]
(386, 0.005775) [b]
(387, 0.002207) [b]
(388, 0.005586) [b]
(389, 0.004039) [b]
(390, 0.006443) [b]
(391, 0.003821) [b]
(392, 0.000862) [b]
(393, 0.000643) [b]
(394, 0.005352) [b]
(395, 0.001233) [b]
(396, 0.006597) [b]
(397, 0.005749) [b]
(398, 0.001600) [b]
(399, 0.006202) [b]
(400, 0.000629) [b]
(401, 0.013353) [b]
(402, 0.025716) [b]
(403, 0.025716) [b]
(404, 0.027476) [b]
(405, 0.000099) [b]
(406, 0.000099) [b]
(407, 0.011142) [b]
(408, 0.008257) [b]
(424, 0.018453) [b]
(427, 0.018453) [b]
(446, 0.003500) [b]
(447, 0.059229) [b]
(448, 0.013223) [b]
(449, 0.022727) [b]
(450, 0.019726) [b]
(451, 0.023370) [b]
(452, 0.012250) [b]
(453, 0.004121) [b]
(454, 0.016391) [b]
(455, 0.021051) [b]
(456, 0.010078) [b]
(457, 0.004017) [b]
(458, 0.028020) [b]
(459, 0.006028) [b]
(460, 0.040845) [b]
(461, 0.011336) [b]
(491, 0.013704) [b]
(492, 0.004299) [b]
(493, 0.014245) [b]
(494, 0.001473) [b]
(495, 0.008387) [b]
(496, 0.004771) [b]
(497, 0.003282) [b]
(498, 0.003282) [b]
(499, 0.007019) [b]
(500, 0.005940) [b]
(501, 0.009227) [b]
(502, 0.004674) [b]
(503, 0.002725) [b]
(504, 0.004941) [b]
(505, 0.005775) [b]
(506, 0.002207) [b]
(507, 0.005586) [b]
(508, 0.004039) [b]
(509, 0.006443) [b]
(510, 0.003821) [b]
(511, 0.000862) [b]
(512, 0.000643) [b]
(513, 0.005352) [b]
(514, 0.001233) [b]
(515, 0.006597) [b]
(516, 0.005749) [b]
(517, 0.001600) [b]
(518, 0.006202) [b]
(519, 0.000629) [b]
(520, 0.013353) [b]
(521, 0.025716) [b]
(522, 0.025716) [b]
(523, 0.027476) [b]
(524, 0.000099) [b]
(525, 0.000099) [b]
(526, 0.011142) [b]
(527, 0.008257) [b]
(565, 0.003500) [b]
(566, 0.059229) [b]
(567, 0.013223) [b]
(568, 0.022727) [b]
(569, 0.019726) [b]
(570, 0.023370) [b]
(571, 0.012250) [b]
(572, 0.004121) [b]
(573, 0.016391) [b]
(574, 0.021051) [b]
(575, 0.010078) [b]
(576, 0.004017) [b]
(577, 0.028020) [b]
(578, 0.006028) [b]
(579, 0.040845) [b]
(580, 0.011336) [b]
(610, 0.013704) [b]
(611, 0.004299) [b]
(612, 0.014245) [b]
(613, 0.001473) [b]
(614, 0.008387) [b]
(615, 0.004771) [b]
(616, 0.003282) [b]
(617, 0.003282) [b]
(618, 0.007019) [b]
(619, 0.005940) [b]
(620, 0.009227) [b]
(621, 0.004674) [b]
(622, 0.002725) [b]
(623, 0.004941) [b]
(624, 0.005775) [b]
(625, 0.002207) [b]
(626, 0.005586) [b]
(627, 0.004039) [b]
(628, 0.006443) [b]
(629, 0.003821) [b]
(630, 0.000862) [b]
(631, 0.000643) [b]
(632, 0.005352) [b]
(633, 0.001233) [b]
(634, 0.006597) [b]
(635, 0.005749) [b]
(636, 0.001600) [b]
(637, 0.006202) [b]
(638, 0.000629) [b]
(639, 0.013353) [b]
(640, 0.025716) [b]
(641, 0.025716) [b]
(642, 0.027476) [b]
(643, 0.000099) [b]
(644, 0.000099) [b]
(645, 0.011142) [b]
(646, 0.008257) [b]
(662, 0.018453) [b]
(665, 0.018453) [b]
(684, 0.003500) [b]
(685, 0.059229) [b]
(686, 0.013223) [b]
(687, 0.022727) [b]
(688, 0.019726) [b]
(689, 0.023370) [b]
(690, 0.012250) [b]
(691, 0.004121) [b]
(692, 0.016391) [b]
(693, 0.021051) [b]
(694, 0.010078) [b]
(695, 0.004017) [b]
(696, 0.028020) [b]
(697, 0.006028) [b]
(698, 0.040845) [b]
(699, 0.011336) [b]
(729, 0.013704) [b]
(730, 0.004299) [b]
(731, 0.014245) [b]
(732, 0.001473) [b]
(733, 0.008387) [b]
(734, 0.004771) [b]
(735, 0.003282) [b]
(736, 0.003282) [b]
(737, 0.007019) [b]
(738, 0.005940) [b]
(739, 0.009227) [b]
(740, 0.004674) [b]
(741, 0.002725) [b]
(742, 0.004941) [b]
(743, 0.005775) [b]
(744, 0.002207) [b]
(745, 0.005586) [b]
(746, 0.004039) [b]
(747, 0.006443) [b]
(748, 0.003821) [b]
(749, 0.000862) [b]
(750, 0.000643) [b]
(751, 0.005352) [b]
(752, 0.001233) [b]
(753, 0.006597) [b]
(754, 0.005749) [b]
(755, 0.001600) [b]
(756, 0.006202) [b]
(757, 0.000629) [b]
(758, 0.013353) [b]
(759, 0.025716) [b]
(760, 0.025716) [b]
(761, 0.027476) [b]
(762, 0.000099) [b]
(763, 0.000099) [b]
(764, 0.011142) [b]
(765, 0.008257) [b]
(803, 0.003500) [b]
(804, 0.059229) [b]
(805, 0.013223) [b]
(806, 0.022727) [b]
(807, 0.019726) [b]
(808, 0.023370) [b]
(809, 0.012250) [b]
(810, 0.004121) [b]
(811, 0.016391) [b]
(812, 0.021051) [b]
(813, 0.010078) [b]
(814, 0.004017) [b]
(815, 0.028020) [b]
(816, 0.006028) [b]
(817, 0.040845) [b]
(818, 0.011336) [b]
(848, 0.013704) [b]
(849, 0.004299) [b]
(850, 0.014245) [b]
(851, 0.001473) [b]
(852, 0.008387) [b]
(853, 0.004771) [b]
(854, 0.003282) [b]
(855, 0.003282) [b]
(856, 0.007019) [b]
(857, 0.005940) [b]
(858, 0.009227) [b]
(859, 0.004674) [b]
(860, 0.002725) [b]
(861, 0.004941) [b]
(862, 0.005775) [b]
(863, 0.002207) [b]
(864, 0.005586) [b]
(865, 0.004039) [b]
(866, 0.006443) [b]
(867, 0.003821) [b]
(868, 0.000862) [b]
(869, 0.000643) [b]
(870, 0.005352) [b]
(871, 0.001233) [b]
(872, 0.006597) [b]
(873, 0.005749) [b]
(874, 0.001600) [b]
(875, 0.006202) [b]
(876, 0.000629) [b]
(877, 0.013353) [b]
(878, 0.025716) [b]
(879, 0.025716) [b]
(880, 0.027476) [b]
(881, 0.000099) [b]
(882, 0.000099) [b]
(883, 0.011142) [b]
(884, 0.008257) [b]
(900, 0.018453) [b]
(903, 0.018453) [b]
(922, 0.003500) [b]
(923, 0.059229) [b]
(924, 0.013223) [b]
(925, 0.022727) [b]
(926, 0.019726) [b]
(927, 0.023370) [b]
(928, 0.012250) [b]
(929, 0.004121) [b]
(930, 0.016391) [b]
(931, 0.021051) [b]
(932, 0.010078) [b]
(933, 0.004017) [b]
(934, 0.028020) [b]
(935, 0.006028) [b]
(936, 0.040845) [b]
(937, 0.011336) [b]
(0, 0.000733) [c]
(1, 0.000562) [c]
(2, 0.000562) [c]
(3, 0.000733) [c]
(4, 0.000733) [c]
(5, 0.000562) [c]
(6, 0.000562) [c]
(7, 0.000733) [c]
(13, 0.001076) [c]
(23, 0.002873) [c]
(25, 0.002873) [c]
(26, 0.001601) [c]
(27, 0.001601) [c]
(28, 0.001456) [c]
(29, 0.009326) [c]
(30, 0.005862) [c]
(31, 0.001456) [c]
(32, 0.003255) [c]
(33, 0.001139) [c]
(35, 0.000351) [c]
(37, 0.014495) [c]
(38, 0.004533) [c]
(39, 0.009322) [c]
(41, 0.005715) [c]
(44, 0.007893) [c]
(45, 0.007893) [c]
(49, 0.007380) [c]
(50, 0.019245) [c]
(51, 0.026239) [c]
(52, 0.029297) [c]
(54, 0.006225) [c]
(55, 0.031698) [c]
(56, 0.011934) [c]
(58, 0.005524) [c]
(59, 0.003050) [c]
(60, 0.004760) [c]
(61, 0.027995) [c]
(62, 0.006278) [c]
(63, 0.003486) [c]
(64, 0.000638) [c]
(65, 0.004984) [c]
(66, 0.005233) [c]
(69, 0.003881) [c]
(70, 0.005040) [c]
(73, 0.008580) [c]
(74, 0.009763) [c]
(76, 0.000515) [c]
(78, 0.000515) [c]
(79, 0.005214) [c]
(80, 0.005693) [c]
(85, 0.004939) [c]
(86, 0.004500) [c]
(87, 0.006466) [c]
(88, 0.040320) [c]
(90, 0.013441) [c]
(92, 0.002056) [c]
(94, 0.002314) [c]
(97, 0.003098) [c]
(101, 0.021872) [c]
(102, 0.003918) [c]
(107, 0.003244) [c]
(109, 0.002625) [c]
(120, 0.002766) [c]
(123, 0.007699) [c]
(124, 0.001022) [c]
(125, 0.014762) [c]
(126, 0.003514) [c]
(127, 0.001596) [c]
(128, 0.007724) [c]
(129, 0.006812) [c]
(132, 0.012468) [c]
(133, 0.016714) [c]
(134, 0.003170) [c]
(136, 0.011017) [c]
(137, 0.015925) [c]
(139, 0.000150) [c]
(143, 0.000360) [c]
(144, 0.001338) [c]
(148, 0.003501) [c]
(149, 0.001516) [c]
(152, 0.018448) [c]
(153, 0.000539) [c]
(157, 0.004236) [c]
(158, 0.001953) [c]
(159, 0.003396) [c]
(160, 0.005719) [c]
(161, 0.001558) [c]
(164, 0.013001) [c]
(165, 0.000799) [c]
(166, 0.049011) [c]
(167, 0.002342) [c]
(168, 0.002085) [c]
(170, 0.000472) [c]
(171, 0.027306) [c]
(172, 0.001630) [c]
(174, 0.008634) [c]
(177, 0.000073) [c]
(179, 0.014448) [c]
(181, 0.001792) [c]
(182, 0.001772) [c]
(183, 0.019460) [c]
(184, 0.000429) [c]
(185, 0.016433) [c]
(189, 0.013538) [c]
(190, 0.001940) [c]
(191, 0.014448) [c]
(193, 0.000305) [c]
(194, 0.001716) [c]
(195, 0.000911) [c]
(196, 0.012132) [c]
(197, 0.019420) [c]
(198, 0.000479) [c]
(200, 0.009250) [c]
(202, 0.004789) [c]
(205, 0.011993) [c]
(206, 0.015143) [c]
(207, 0.000072) [c]
(208, 0.016797) [c]
(209, 0.004192) [c]
(210, 0.000262) [c]
(211, 0.003867) [c]
(212, 0.010100) [c]
(213, 0.009989) [c]
(214, 0.005998) [c]
(215, 0.003265) [c]
(217, 0.004192) [c]
(219, 0.002513) [c]
(220, 0.003867) [c]
(221, 0.005679) [c]
(222, 0.005242) [c]
(223, 0.000371) [c]
(224, 0.002059) [c]
(225, 0.002513) [c]
(226, 0.012971) [c]
(227, 0.004714) [c]
(229, 0.002942) [c]
(231, 0.005054) [c]
(232, 0.015625) [c]
(233, 0.002952) [c]
(234, 0.005867) [c]
(236, 0.000262) [c]
(237, 0.000049) [c]
(238, 0.008168) [c]
(239, 0.019013) [c]
(240, 0.006512) [c]
(241, 0.003254) [c]
(244, 0.008027) [c]
(245, 0.007060) [c]
(248, 0.002059) [c]
(252, 0.002416) [c]
(253, 0.000049) [c]
(255, 0.013679) [c]
(256, 0.000109) [c]
(257, 0.009387) [c]
(258, 0.015200) [c]
(259, 0.021779) [c]
(261, 0.008322) [c]
(262, 0.003437) [c]
(264, 0.000281) [c]
(265, 0.003066) [c]
(267, 0.010391) [c]
(268, 0.001267) [c]
(270, 0.011709) [c]
(271, 0.014134) [c]
(272, 0.001017) [c]
(273, 0.004300) [c]
(274, 0.000951) [c]
(275, 0.008574) [c]
(276, 0.006020) [c]
(277, 0.002431) [c]
(278, 0.005191) [c]
(279, 0.004608) [c]
(280, 0.004608) [c]
(282, 0.000891) [c]
(283, 0.004300) [c]
(284, 0.002431) [c]
(286, 0.004198) [c]
(287, 0.001370) [c]
(288, 0.002141) [c]
(289, 0.000330) [c]
(290, 0.008114) [c]
(291, 0.002010) [c]
(292, 0.004320) [c]
(293, 0.002598) [c]
(294, 0.003093) [c]
(295, 0.003187) [c]
(296, 0.002598) [c]
(297, 0.004703) [c]
(298, 0.000185) [c]
(299, 0.003762) [c]
(301, 0.005599) [c]
(304, 0.006201) [c]
(305, 0.004703) [c]
(306, 0.001623) [c]
(307, 0.002188) [c]
(308, 0.005263) [c]
(309, 0.008606) [c]
(311, 0.002481) [c]
(313, 0.006002) [c]
(314, 0.002330) [c]
(315, 0.015154) [c]
(316, 0.005599) [c]
(318, 0.009978) [c]
(319, 0.004487) [c]
(320, 0.003318) [c]
(321, 0.000545) [c]
(322, 0.001389) [c]
(326, 0.002330) [c]
(327, 0.001693) [c]
(328, 0.003280) [c]
(329, 0.001646) [c]
(330, 0.002542) [c]
(331, 0.002792) [c]
(332, 0.009392) [c]
(333, 0.001727) [c]
(335, 0.001565) [c]
(336, 0.013431) [c]
(337, 0.009392) [c]
(338, 0.007271) [c]
(339, 0.002953) [c]
(340, 0.004088) [c]
(341, 0.004088) [c]
(342, 0.001646) [c]
(343, 0.002569) [c]
(344, 0.001682) [c]
(345, 0.003179) [c]
(346, 0.014158) [c]
(347, 0.009978) [c]
(348, 0.001687) [c]
(349, 0.001553) [c]
(350, 0.011985) [c]
(352, 0.002362) [c]
(353, 0.002953) [c]
(354, 0.002291) [c]
(355, 0.016548) [c]
(356, 0.001801) [c]
(357, 0.001063) [c]
(358, 0.005356) [c]
(359, 0.006004) [c]
(360, 0.001249) [c]
(361, 0.002310) [c]
(362, 0.006686) [c]
(363, 0.001820) [c]
(365, 0.016548) [c]
(366, 0.002575) [c]
(367, 0.009246) [c]
(368, 0.001940) [c]
(369, 0.014158) [c]
(370, 0.011862) [c]
(371, 0.014518) [c]
(372, 0.000022) [c]
(373, 0.009185) [c]
(374, 0.012762) [c]
(375, 0.007216) [c]
(376, 0.006259) [c]
(377, 0.005938) [c]
(378, 0.010086) [c]
(380, 0.014518) [c]
(381, 0.007740) [c]
(383, 0.003596) [c]
(384, 0.018699) [c]
(385, 0.000919) [c]
(386, 0.005730) [c]
(387, 0.003540) [c]
(389, 0.007216) [c]
(390, 0.001687) [c]
(391, 0.002836) [c]
(392, 0.001675) [c]
(393, 0.010684) [c]
(394, 0.012762) [c]
(395, 0.004680) [c]
(396, 0.004680) [c]
(397, 0.013733) [c]
(398, 0.006004) [c]
(399, 0.000618) [c]
(400, 0.001104) [c]
(401, 0.001093) [c]
(403, 0.002735) [c]
(404, 0.002046) [c]
(405, 0.008548) [c]
(406, 0.003384) [c]
(407, 0.000335) [c]
(409, 0.001812) [c]
(410, 0.021588) [c]
(411, 0.020269) [c]
(414, 0.011266) [c]
(415, 0.000086) [c]
(416, 0.002836) [c]
(417, 0.023151) [c]
(418, 0.004301) [c]
(421, 0.001734) [c]
(422, 0.002650) [c]
(423, 0.007538) [c]
(424, 0.000466) [c]
(426, 0.000944) [c]
(427, 0.001824) [c]
(428, 0.010684) [c]
(429, 0.000773) [c]
(430, 0.004123) [c]
(431, 0.003540) [c]
(432, 0.020269) [c]
(433, 0.000374) [c]
(434, 0.000619) [c]
(435, 0.000466) [c]
(436, 0.004576) [c]
(440, 0.004260) [c]
(443, 0.003810) [c]
(444, 0.005953) [c]
(445, 0.000513) [c]
(446, 0.013733) [c]
(447, 0.005159) [c]
(448, 0.021574) [c]
(449, 0.012830) [c]
(450, 0.000019) [c]
(452, 0.001373) [c]
(453, 0.006340) [c]
(454, 0.005403) [c]
(455, 0.007017) [c]
(456, 0.000257) [c]
(458, 0.014213) [c]
(459, 0.016894) [c]
(460, 0.000237) [c]
(462, 0.000840) [c]
(465, 0.003810) [c]
(466, 0.015583) [c]
(467, 0.008779) [c]
(468, 0.001041) [c]
(469, 0.003007) [c]
(472, 0.000458) [c]
(473, 0.003776) [c]
(474, 0.000582) [c]
(475, 0.006233) [c]
(476, 0.000497) [c]
(477, 0.004755) [c]
(478, 0.000510) [c]
(479, 0.007441) [c]
(480, 0.011128) [c]
(481, 0.000268) [c]
(482, 0.000707) [c]
(483, 0.005712) [c]
(484, 0.006574) [c]
(485, 0.014978) [c]
(486, 0.000400) [c]
(488, 0.000375) [c]
(489, 0.000193) [c]
(491, 0.001323) [c]
(492, 0.000371) [c]
(494, 0.000501) [c]
(496, 0.000390) [c]
(497, 0.001040) [c]
(498, 0.001040) [c]
(499, 0.004821) [c]
(500, 0.001379) [c]
(501, 0.000591) [c]
(502, 0.000375) [c]
(503, 0.000561) [c]
(504, 0.021885) [c]
(505, 0.000243) [c]
(506, 0.000813) [c]
(507, 0.000390) [c]
(508, 0.015800) [c]
(509, 0.000927) [c]
(510, 0.000904) [c]
(511, 0.000544) [c]
(513, 0.004345) [c]
(516, 0.000257) [c]
(517, 0.011521) [c]
(518, 0.001221) [c]
(519, 0.000927) [c]
(522, 0.002484) [c]
(523, 0.000696) [c]
(524, 0.000638) [c]
(525, 0.000192) [c]
(526, 0.002825) [c]
(527, 0.004642) [c]
(528, 0.000305) [c]
(529, 0.000128) [c]
(530, 0.000561) [c]
(532, 0.025359) [c]
(540, 0.000343) [c]
(543, 0.000618) [c]
(544, 0.001106) [c]
(547, 0.002778) [c]
(548, 0.002778) [c]
(549, 0.000267) [c]
(551, 0.003046) [c]
(553, 0.003688) [c]
(555, 0.001041) [c]
(556, 0.002930) [c]
(557, 0.001679) [c]
(558, 0.000813) [c]
(560, 0.000128) [c]
(561, 0.000540) [c]
(562, 0.003267) [c]
(563, 0.010412) [c]
(566, 0.004361) [c]
(567, 0.003098) [c]
(568, 0.002499) [c]
(574, 0.006948) [c]
(575, 0.000343) [c]
(576, 0.000638) [c]
(578, 0.002499) [c]
(581, 0.000697) [c]
(585, 0.007533) [c]
(586, 0.000208) [c]
(588, 0.019178) [c]
(589, 0.022771) [c]
(590, 0.011979) [c]
(592, 0.009914) [c]
(594, 0.000022) [c]
(595, 0.009119) [c]
(596, 0.003964) [c]
(597, 0.001041) [c]
(598, 0.018930) [c]
(599, 0.008897) [c]
(600, 0.008073) [c]
(601, 0.014115) [c]
(604, 0.001576) [c]
(605, 0.024515) [c]
(606, 0.005157) [c]
(608, 0.011374) [c]
(609, 0.006632) [c]
(611, 0.000661) [c]
(612, 0.014183) [c]
(615, 0.003631) [c]
(616, 0.009245) [c]
(617, 0.000436) [c]
(618, 0.021462) [c]
(619, 0.001327) [c]
(621, 0.003826) [c]
(622, 0.001762) [c]
(623, 0.008115) [c]
(625, 0.003825) [c]
(626, 0.014316) [c]
(627, 0.018498) [c]
(628, 0.010216) [c]
(629, 0.001341) [c]
(630, 0.006099) [c]
(631, 0.009146) [c]
(633, 0.022892) [c]
(634, 0.009880) [c]
(635, 0.000115) [c]
(636, 0.025469) [c]
(637, 0.005342) [c]
(638, 0.010670) [c]
(639, 0.020960) [c]
(641, 0.011473) [c]
(642, 0.002809) [c]
(643, 0.014316) [c]
(644, 0.020833) [c]
(645, 0.007504) [c]
(646, 0.021462) [c]
(647, 0.008434) [c]
(648, 0.000242) [c]
(649, 0.023968) [c]
(650, 0.003951) [c]
(651, 0.003631) [c]
(652, 0.017116) [c]
(653, 0.000313) [c]
(654, 0.000201) [c]
(655, 0.000305) [c]
(656, 0.000305) [c]
(657, 0.000314) [c]
(658, 0.018285) [c]
(659, 0.014385) [c]
(660, 0.015126) [c]
(661, 0.002846) [c]
(662, 0.000415) [c]
(663, 0.011303) [c]
(664, 0.005304) [c]
(665, 0.000246) [c]
(666, 0.009229) [c]
(667, 0.011584) [c]
(668, 0.000563) [c]
(669, 0.018285) [c]
(671, 0.001470) [c]
(673, 0.000099) [c]
(674, 0.004584) [c]
(676, 0.015846) [c]
(677, 0.003873) [c]
(678, 0.007658) [c]
(679, 0.003363) [c]
(680, 0.008077) [c]
(681, 0.003112) [c]
(682, 0.008849) [c]
(683, 0.005812) [c]
(684, 0.011425) [c]
(685, 0.003447) [c]
(687, 0.001435) [c]
(688, 0.006430) [c]
(689, 0.003363) [c]
(691, 0.004357) [c]
(692, 0.019178) [c]
(693, 0.000096) [c]
(694, 0.019014) [c]
(695, 0.001514) [c]
(696, 0.013279) [c]
(698, 0.016125) [c]
(699, 0.000939) [c]
(700, 0.026796) [c]
(701, 0.016001) [c]
(703, 0.004849) [c]
(704, 0.008611) [c]
(705, 0.007271) [c]
(706, 0.005506) [c]
(709, 0.004088) [c]
(711, 0.017411) [c]
(712, 0.027870) [c]
(714, 0.004026) [c]
(715, 0.017138) [c]
(716, 0.016880) [c]
(718, 0.000377) [c]
(719, 0.002549) [c]
(721, 0.000642) [c]
(722, 0.000270) [c]
(723, 0.006146) [c]
(727, 0.005965) [c]
(728, 0.009697) [c]
(729, 0.003355) [c]
(730, 0.002220) [c]
(731, 0.011143) [c]
(732, 0.017920) [c]
(733, 0.010060) [c]
(736, 0.010275) [c]
(737, 0.001480) [c]
(738, 0.010999) [c]
(739, 0.008849) [c]
(740, 0.002009) [c]
(742, 0.011959) [c]
(744, 0.014328) [c]
(745, 0.004490) [c]
(746, 0.009631) [c]
(747, 0.010999) [c]
(748, 0.002386) [c]
(749, 0.011894) [c]
(750, 0.013240) [c]
(751, 0.000476) [c]
(752, 0.000480) [c]
(755, 0.002063) [c]
(756, 0.000498) [c]
(757, 0.004560) [c]
(758, 0.002063) [c]
(759, 0.002063) [c]
(760, 0.039076) [c]
(761, 0.024591) [c]
(762, 0.002063) [c]
(763, 0.025722) [c]
(764, 0.006001) [c]
(765, 0.007126) [c]
(766, 0.002141) [c]
(767, 0.016243) [c]
(768, 0.000662) [c]
(769, 0.014662) [c]
(770, 0.017300) [c]
(771, 0.009322) [c]
(772, 0.030494) [c]
(773, 0.003842) [c]
(774, 0.000022) [c]
(775, 0.011305) [c]
(776, 0.001189) [c]
(777, 0.022090) [c]
(778, 0.010763) [c]
(779, 0.004848) [c]
(780, 0.003276) [c]
(781, 0.012926) [c]
(782, 0.014212) [c]
(784, 0.000077) [c]
(785, 0.000709) [c]
(786, 0.025704) [c]
(787, 0.000022) [c]
(788, 0.011596) [c]
(789, 0.033423) [c]
(790, 0.010939) [c]
(791, 0.016216) [c]
(792, 0.002255) [c]
(793, 0.020711) [c]
(794, 0.001845) [c]
(795, 0.002059) [c]
(796, 0.015542) [c]
(798, 0.003145) [c]
(799, 0.032213) [c]
(800, 0.002688) [c]
(801, 0.039076) [c]
(802, 0.002017) [c]
(803, 0.019086) [c]
(804, 0.001841) [c]
(805, 0.003229) [c]
(806, 0.040177) [c]
(807, 0.017609) [c]
(808, 0.004201) [c]
(809, 0.003089) [c]
(810, 0.029750) [c]
(812, 0.023679) [c]
(813, 0.025229) [c]
(814, 0.013441) [c]
(816, 0.000050) [c]
(817, 0.012946) [c]
(819, 0.000404) [c]
(820, 0.007205) [c]
(821, 0.029531) [c]
(822, 0.018035) [c]
(825, 0.014857) [c]
(827, 0.003273) [c]
(829, 0.009990) [c]
(830, 0.004585) [c]
(831, 0.007677) [c]
(832, 0.012078) [c]
(833, 0.012098) [c]
(836, 0.007655) [c]
(837, 0.000050) [c]
(838, 0.016873) [c]
(839, 0.019562) [c]
(840, 0.012634) [c]
(841, 0.025229) [c]
(842, 0.009004) [c]
(843, 0.029874) [c]
(844, 0.007705) [c]
(845, 0.000495) [c]
(846, 0.006250) [c]
(847, 0.009342) [c]
(849, 0.007705) [c]
(850, 0.004550) [c]
(851, 0.009004) [c]
(852, 0.004550) [c]
(853, 0.019842) [c]
(855, 0.000591) [c]
(856, 0.000103) [c]
(857, 0.005470) [c]
(858, 0.011772) [c]
(859, 0.015309) [c]
(860, 0.005615) [c]
(861, 0.010956) [c]
(863, 0.001539) [c]
(864, 0.001378) [c]
(865, 0.009342) [c]
(866, 0.024001) [c]
(869, 0.002808) [c]
(870, 0.000683) [c]
(872, 0.005150) [c]
(873, 0.006877) [c]
(874, 0.001420) [c]
(875, 0.001612) [c]
(876, 0.001033) [c]
(878, 0.002452) [c]
(884, 0.003455) [c]
(885, 0.000868) [c]
(886, 0.002451) [c]
(888, 0.001027) [c]
(892, 0.007525) [c]
(894, 0.000981) [c]
(899, 0.004402) [c]
(904, 0.000026) [c]
(905, 0.005756) [c]
(907, 0.001769) [c]
(908, 0.001335) [c]
(909, 0.012037) [c]
(913, 0.002508) [c]
(914, 0.003579) [c]
(916, 0.012132) [c]
(918, 0.000387) [c]
(923, 0.007225) [c]
(929, 0.001762) [c]
(931, 0.002965) [c]
(932, 0.000135) [c]
(933, 0.000821) [c]
(944, 0.004485) [c]
(945, 0.000457) [c]
(949, 0.019093) [c]
(952, 0.000101) [c]
(954, 0.015999) [c]
(960, 0.004485) [c]
(963, 0.000481) [c]
(965, 0.000287) [c]
(967, 0.000991) [c]
(968, 0.000698) [c]
(970, 0.000026) [c]
(977, 0.000153) [c]
(978, 0.000147) [c]
(983, 0.004926) [c]
(984, 0.017295) [c]
(991, 0.000124) [c]
(994, 0.002040) [c]
(995, 0.003961) [c]
(1000, 0.001396) [c]
(1001, 0.003342) [c]
(1002, 0.007964) [c]
(1005, 0.001107) [c]
(1006, 0.000245) [c]
(1007, 0.000350) [c]
(1008, 0.003451) [c]
(1011, 0.000653) [c]
(1013, 0.003753) [c]
(1015, 0.005970) [c]
(1016, 0.004527) [c]
(1017, 0.007311) [c]
(1018, 0.013008) [c]
(1019, 0.004194) [c]
(1020, 0.003741) [c]
(1021, 0.002764) [c]
(1025, 0.004362) [c]
(1026, 0.001016) [c]
(1027, 0.007122) [c]
(1028, 0.004315) [c]
(1029, 0.000737) [c]
(1030, 0.002246) [c]
(1031, 0.006787) [c]
(1032, 0.002149) [c]
(1033, 0.005676) [c]
(1034, 0.003041) [c]
(1035, 0.000063) [c]
(1036, 0.003994) [c]
(1037, 0.001641) [c]
(1038, 0.001172) [c]
(1040, 0.001641) [c]
(1042, 0.001214) [c]
(1043, 0.000166) [c]
(1044, 0.004554) [c]
(1046, 0.004271) [c]
(1047, 0.004439) [c]
(1048, 0.000092) [c]
(1049, 0.002329) [c]
(1050, 0.003015) [c]
(1051, 0.000212) [c]
(1052, 0.001712) [c]
(1053, 0.002516) [c]
(1054, 0.002246) [c]
(1055, 0.002235) [c]
(1056, 0.001104) [c]
(1057, 0.000203) [c]
(1058, 0.004188) [c]
(1059, 0.000377) [c]
(1060, 0.002337) [c]
(1061, 0.002270) [c]
(1062, 0.002267) [c]
(1063, 0.004376) [c]
(1064, 0.003509) [c]
(1065, 0.001098) [c]
(1066, 0.002970) [c]
(1067, 0.005290) [c]
(1068, 0.001363) [c]
(1069, 0.000384) [c]
(1070, 0.005484) [c]
(1071, 0.004386) [c]
(1072, 0.006852) [c]
(1073, 0.002641) [c]
(1074, 0.000026) [c]
(1075, 0.000026) [c]
(1076, 0.004288) [c]
(1077, 0.000377) [c]
(1078, 0.000875) [c]
(1079, 0.016067) [c]
(1080, 0.003372) [c]
(1082, 0.000617) [c]
(1084, 0.000026) [c]
(1085, 0.000026) [c]
(1086, 0.000285) [c]
(1087, 0.001767) [c]
(1088, 0.002020) [c]
(1089, 0.002793) [c]
(1090, 0.009200) [c]
(1091, 0.000800) [c]
(1092, 0.014330) [c]
(1093, 0.001730) [c]
(1094, 0.002676) [c]
(1095, 0.001730) [c]
(1096, 0.003221) [c]
(1097, 0.002168) [c]
(1098, 0.004398) [c]
(1099, 0.003786) [c]
(1100, 0.001504) [c]
(1101, 0.003101) [c]
(1102, 0.008191) [c]
(1103, 0.006863) [c]
(1104, 0.000431) [c]
(1105, 0.000321) [c]
(1107, 0.001910) [c]
(1108, 0.003867) [c]
(1109, 0.001465) [c]
(1110, 0.003320) [c]
(1111, 0.002078) [c]
(1112, 0.000075) [c]
(1113, 0.000094) [c]
(1115, 0.001633) [c]
(1116, 0.011836) [c]
(1117, 0.000441) [c]
(1119, 0.001534) [c]
(1120, 0.000014) [c]
(1121, 0.003298) [c]
(1123, 0.012705) [c]
(1124, 0.000143) [c]
(1125, 0.010712) [c]
(1126, 0.000932) [c]
(1127, 0.005152) [c]
(1128, 0.006086) [c]
(1129, 0.003795) [c]
(1130, 0.000022) [c]
(1131, 0.005672) [c]
(1132, 0.000932) [c]
(1133, 0.003127) [c]
(1134, 0.002181) [c]
(1135, 0.002052) [c]
(1136, 0.001500) [c]
(1137, 0.000766) [c]
(1138, 0.000028) [c]
(1139, 0.000282) [c]
(1140, 0.000766) [c]
(1141, 0.000834) [c]
(1142, 0.001539) [c]
(1143, 0.001223) [c]
(1144, 0.006338) [c]
(1145, 0.000591) [c]
(1146, 0.004734) [c]
(1147, 0.000462) [c]
(1148, 0.000027) [c]
(1149, 0.000819) [c]
(1150, 0.000312) [c]
(1151, 0.002550) [c]
(1152, 0.000333) [c]
(1153, 0.009163) [c]
(1154, 0.000395) [c]
(1155, 0.002457) [c]
(1156, 0.000163) [c]
(1157, 0.001376) [c]
(1158, 0.005478) [c]
(1159, 0.000212) [c]
(1160, 0.001942) [c]
(1161, 0.002752) [c]
(1162, 0.003028) [c]
(1163, 0.003681) [c]
(1164, 0.003786) [c]
(1165, 0.002682) [c]
(1166, 0.002705) [c]
(1168, 0.003951) [c]
(1169, 0.004189) [c]
(1170, 0.000112) [c]
(1171, 0.004654) [c]
(1173, 0.000014) [c]
(1174, 0.002238) [c]
(1176, 0.000112) [c]
(1177, 0.009243) [c]
(1178, 0.012024) [c]
(1179, 0.004472) [c]
(1180, 0.001086) [c]
(1181, 0.000893) [c]
(1182, 0.001407) [c]
(1183, 0.003307) [c]
(1184, 0.001385) [c]
(1185, 0.000186) [c]
(1186, 0.000336) [c]
(1187, 0.002116) [c]
(1188, 0.003887) [c]
(1189, 0.000937) [c]
(1190, 0.001366) [c]
(1191, 0.004956) [c]
(1192, 0.004056) [c]
(1193, 0.001005) [c]
(1194, 0.003708) [c]
(1195, 0.000249) [c]
(1196, 0.001811) [c]
(1197, 0.001157) [c]
(1198, 0.000046) [c]
(1200, 0.001796) [c]
(1201, 0.001731) [c]
(1202, 0.000576) [c]
(1203, 0.002872) [c]
(1204, 0.001776) [c]
(1205, 0.000309) [c]
(1206, 0.000884) [c]
(1207, 0.004478) [c]
(1208, 0.000346) [c]
(1209, 0.001218) [c]
(1210, 0.001544) [c]
(1211, 0.000728) [c]
(1212, 0.003636) [c]
(1213, 0.001129) [c]
(1214, 0.000192) [c]
(1215, 0.000011) [c]
(1216, 0.000122) [c]
(1218, 0.001798) [c]
(1219, 0.002491) [c]
(1220, 0.000122) [c]
(1221, 0.003760) [c]
(1222, 0.000122) [c]
(1225, 0.002876) [c]
(1226, 0.002522) [c]
(1227, 0.002679) [c]
(1228, 0.003357) [c]
(1229, 0.000124) [c]
(1230, 0.000199) [c]
(1232, 0.000122) [c]
(1233, 0.004560) [c]
(1234, 0.003556) [c]
(1235, 0.001338) [c]
(1237, 0.003692) [c]
};
\end{axis}
\end{tikzpicture}

\end{frame}



\begin{frame}{Instanser fra Goodtech}
\begin{table}[ht!]
    \centering
    \caption{Størrelse for hver instans}
    \begin{tabular}{lrrr}
    Problemstørrelse & \textit{small} & \textit{large} & \textit{vlarge} \\\hline
    Rader        & 82             & 328            & 1127 \\
    Kolonne      & 238            & 952            & 3437 \\
    Ikke-nuller A & 348            & 1392           & 4840 \\
    Ikke-nuller H & 108            & 432            & 894 \\
    \end{tabular}
    \label{table:sizes}
\end{table}
\begin{table}[ht!]
    \centering
    \caption{Verdier i objektfunksjonen}

    \begin{tabular}{lrr}
      & \textit{small} og \textit{large}         & \textit{vlarge} \\\hline
    $\max(h_{ii})$      & $2.9614 \times 10^{-2}$ & $4.9011 \times 10^{-2}$ \\
    $\min(h_{ii})$      & $4.9290 \times 10^{-5}$ & $1.1026 \times 10^{-5}$ \\
$\textrm{mean}(h_{ii})$ & $5.2864 \times 10^{-3}$ & $5.8984 \times 10^{-3}$ \\
    $\max(b_{i})$       & 20                      & 20 \\
    $\min(b_{i})$       & -70                     & -50 \\
    \end{tabular}
    \label{table:maxmin}
\end{table}
\end{frame}



\begin{frame}{Instanser fra Goodtech}
\begin{itemize}
\item Vi ser at det lineære leddet har mye høyere verdier enn det kvadratiske
\item Prøver å lineærisere objektfunksjonen
\item Lineær Taylor-utvikling av objektfunksjonen i punkt $a$: \\
      $T_a(x) = -a^T H a + 2a^T H x + b^T x$ \\
      $T_0(x) = b^T x$
\end{itemize}
Definerer et LP $\mathcal{L}$ for hvert QP $\mathcal{Q}$
\[
\min_{x \in \mathbb{R}^n} g(x) \quad \textrm{subject to} ~ Ax = 0, ~ l \leq x \leq u
\]
\begin{itemize}
\item $g(x) = T_0(x) = b^T x$
\end{itemize}
\end{frame}



\begin{frame}{Hvor like er $\mathcal{L}$ og $\mathcal{Q}$?}
\begin{itemize}
\item Optimal løsning til $\mathcal{Q}$ noteres $x^*$
\item Optimal løsning til $\mathcal{L}$ noteres $\hat{x}$
\item Vi noterer avvik mellom $f(\hat{x})$ og $f(x^*)$ for
      $\Delta = \left| \frac{f(\hat{x}) - f(x^*)}{f(x^*)} \right|$
\item Vi genererer tilfeldige instanser og sjekker $\Delta$
\end{itemize}
\end{frame}



\begin{frame}{$100\Delta$}
\begin{figure}[ht!]
\begin{tikzpicture}
    \begin{axis}[
        width=\textwidth*0.80,
%        grid=major,
        view={245}{30},
%        xmax=99,
%        ymax=99,
%        zmax=1,
%        restrict z to domain=0:1,
        xlabel=x,
        ylabel=y,
        zlabel=$100\Delta$]

        \addplot3[surf,mesh/rows=20,mesh/cols=20] table[x=b,y=h,z=dev]
                                                   {include/data/delta3d.dat};


%        \addplot3[
%            surf,
%            mesh/rows=101,
%            mesh/cols=100,
%        ] 
%        table[x=b,y=h,z=dev] {include/data/delta3d.dat};
\end{axis}
\end{tikzpicture}

\caption{Avvik som en funksjon av tettheten i objektfunksjonen. $x$ er prosent
         nuller på diagonalen til $H$. $y$ er prosent nuller i $b$}
\end{figure}
\end{frame}



\begin{frame}{Hvor like er $\mathcal{L}$ og $\mathcal{Q}$?}
Oppnår $95\%$ optimal verdi etter løst $\mathcal{L}$.
\begin{itemize}
\item Metode basert på successive linear programming (SLP)
\item Lar vi startverdien $x_0 = 0$, når vi rundt $95\%$ av optimal målfunksjonsverdi etter
      første iterasjon
\item Taylor-utvikling i $x_k$ noterer vi som $T_k$
\item Definerer $\mathcal{L}_k$ som LP-problemet å minimere $T_k$ underlagt
      sidekravene til $\mathcal{Q}$.
\end{itemize}
\end{frame}



\begin{frame}{Slp}
\begin{algorithm}[ht!]
\caption{\texttt{slp($x_0$, $\epsilon \ge 0$)}}
\label{alg:iter}
\SetAlgoVlined
\DontPrintSemicolon
%Given $x_0$ and some $\epsilon \ge 0$\;
Set $k \leftarrow 0$\;

\Repeat{$\displaystyle \frac{f(x_{k-1}) - f(x_k)}{|f(x_{k-1})|} \leq \epsilon$} {
    $T_k \leftarrow - x_k^THx_k + 2x_k^THx + b^Tx$ \\
    $\hat{x}_k \leftarrow \textrm{optimal solution of } \mathcal{L}_k$ (Solve) \\
    $\alpha_k \in \argmin_{\alpha \leq 1} f((1-\alpha) x_k + \alpha \hat{x}_k)$ \\
%   $\alpha_k \in \argmin_{\alpha \leq 1} f(\alpha x_k + (1-\alpha)\hat{x}_k)$\;
    $x_{k+1} \leftarrow (1-\alpha_k) x_k + \alpha_k\hat{x}_k$ \\
    $k \leftarrow k +1$
}
\end{algorithm}

\end{frame}



\begin{frame}{Et eksempel}
\[
\begin{array}{lcrcrcl}
\textrm{minimize}           & &   (x-1)^2 &+&   (y-1)^2 & - &  2 \\
\textrm{subject to}         & &         x &+&         y &\leq& 3 \\
                            & &         x &-&         y &\leq& 1 \\
                            & &         x &+&       3 y &\leq& 4 \\
                 \multicolumn{5}{r}{x,y}                &\geq& 0
\end{array}
\]
\begin{itemize}
\item $T_0 = -2 x - 2 y$ blir lineær objektfunksjon
\end{itemize}
\end{frame}



\begin{frame}{$\mathcal{L}_0$}
\begin{center}
\begin{tikzpicture}
    % grid and axes
    \draw[scale=1.5,->,name path=xaxis] (-0.2,0) -- (2.2,0) node[right] {$x$};
    \draw[scale=1.5,->,name path=yaxis] (0,-0.2) -- (0,2.2) node[above] {$y$};

    % draw lines
    \draw[scale=1.5,name path=line1, domain=0.6:2.5] plot(\x,{-\x + 3})
                                                    node[above right=0.2cm] {};
    \draw[scale=1.5,name path=line2, domain=-0.2:1.6] plot(\x,{0.33*\x + 1.33})
                                                    node[above right=0.2cm] {};
    \draw[scale=1.5,name path=line3, domain=0.7:2.3] plot(\x,{\x - 1})
                                                    node[above right=0.2cm] {};

    % calculate intersection points
    \node[name intersections={of=line1 and line2}] (a) at (intersection-1) {};
    \node[name intersections={of=line1 and line3}] (b) at (intersection-1) {};
    \node[name intersections={of=xaxis and line3}] (c) at (intersection-1) {};
    \node[name intersections={of=xaxis and yaxis}] (d) at (intersection-1) {};
    \node[name intersections={of=line2 and yaxis}] (e) at (intersection-1) {};

    % write the coordinates of the corners.
    \path let \p0 = (a) in node [above right=0.0cm of a] {($1.25$, $1.75$)};
    \path let \p0 = (b) in node [right=0.1cm of b] {($2$, $1$)};
    \path let \p0 = (c) in node [below right=-0.35cm of c] {($1$, $0$)};
    %\path let \p0 = (d) in node [left=0.0cm of d] {($0$, $0$)};
    \path let \p0 = (e) in node [left=0.0cm of e] {($0$, $1.\overline{3}$)};

    \node at (1.5, 1.5) (qopt) {};
    \node at (3, 1.5) (lopt) {};
    \node at (-0.75, 1.125) (qoptdesc) {$x^*$};
    \node at (-1.5, 0.375) (x0desc) {$x_0$};
    \node[text=red] at (3.75, 2.625) (loptdesc) {$\hat{x}_0$};
    \draw[scale=1.5,->] (qoptdesc) .. controls ([xshift=1cm] qoptdesc) and ([xshift=-1cm]
                                                               qopt) .. (qopt);
    \draw[scale=1.5,->] (loptdesc) .. controls ([xshift=-1cm] loptdesc) and ([yshift=1cm]
                                                               lopt) .. (lopt);
    \draw[scale=1.5,->] (x0desc) .. controls ([xshift=1cm] x0desc) and ([xshift=-1cm] d)
                                                                     .. (d);

    \draw[scale=1.5,fill] (qopt) circle [radius=0.02];

    % draw the big polygin
    \fill[very thick,fill=blue,fill opacity=0.1] (a.center) -- (b.center) --
                                                 (c.center) -- (d.center) --
                                                 (e.center) -- cycle;

    \draw[scale=1.5,thick, red, name path=lobj, domain=-0.4:0.4] plot(\x, {-\x}) {};

    % draw the quadratic objective function
    \draw[scale=1.5,thin, dashed] plot[id=qobj1, raw gnuplot] function {
        f(x,y) = x**2 + y**2 - 2*x - 2*y + 1.75;
        set xrange[-1:2];
        set yrange[-2:2];
        set view 0,0;
        set isosamples 1000,1000;
        set size square;
        set cont base;
        set cntrparam levels incre 0,0.1,0;
        unset surface;
        splot f(x,y);
    };
    \draw[scale=1.5,thin, dashed] plot[id=qobj2, raw gnuplot] function {
        f(x,y) = x**2 + y**2 - 2*x - 2*y + 1.9;
        set xrange[-1:2];
        set yrange[-2:2];
        set view 0,0;
        set isosamples 1000,1000;
        set size square;
        set cont base;
        set cntrparam levels incre 0,0.1,0;
        unset surface;
        splot f(x,y);
    };
    \draw[scale=1.5,thin, dashed] plot[id=qobj3, raw gnuplot] function {
        f(x,y) = x**2 + y**2 - 2*x - 2*y + 1.98;
        set xrange[-1:2];
        set yrange[-2:2];
        set view 0,0;
        set isosamples 1000,1000;
        set size square;
        set cont base;
        set cntrparam levels incre 0,0.1,0;
        unset surface;
        splot f(x,y);
    };
\end{tikzpicture}

\end{center}
\end{frame}



\begin{frame}{Linjesøk}
\begin{center}
\begin{tikzpicture}
    % grid and axes
    \draw[scale=1.5,->,name path=xaxis] (-0.2,0) -- (2.2,0) node[right] {$x$};
    \draw[scale=1.5,->,name path=yaxis] (0,-0.2) -- (0,2.2) node[above] {$y$};

    % draw lines
    \draw[scale=1.5,name path=line1, domain=0.6:2.5] plot(\x,{-\x + 3});
    \draw[scale=1.5,name path=line2, domain=-0.2:1.6] plot(\x,{0.33*\x + 1.33});
    \draw[scale=1.5,name path=line3, domain=0.7:2.3] plot(\x,{\x - 1});

    % calculate intersection points
    \node[name intersections={of=line1 and line2}] (a) at (intersection-1) {};
    \node[name intersections={of=line1 and line3}] (b) at (intersection-1) {};
    \node[name intersections={of=xaxis and line3}] (c) at (intersection-1) {};
    \node[name intersections={of=xaxis and yaxis}] (d) at (intersection-1) {};
    \node[name intersections={of=line2 and yaxis}] (e) at (intersection-1) {};

    % draw the big polygon
    \fill[very thick,fill=tleblue] (a.center) -- (b.center)
                                              -- (c.center) -- (d.center)
                                              -- (e.center) -- cycle;

    \draw[scale=1.5,name path=linesearch, domain=-0.1:2.1] plot(\x,{0.5*\x});

    \node at (1.2*1.5,0.6*1.5) (x1) {};
    \node at (1.95*1.5, 0.35*1.5) (x1desc) {$x_1$};
    \draw[scale=1.5,->] (x1desc) .. controls ([yshift=0.5cm] x1desc) and ([xshift=0.5cm]
                                                                   x1) .. (x1);


    \node[red] at (0.7*1.5, 0.1*1.5) (alphadesc) {$\alpha$};
    \node at (1.2*1.5, 0.6*1.5) (alpha) {};
    \node at (1*1.5, 1*1.5) (qopt) {};
    \node at (2*1.5, 1*1.5) (lopt) {};
    \node at (-0.5*1.5, 0.75*1.5) (qoptdesc) {$x^*$};
    \node at (2.5*1.5, 1.75*1.5) (loptdesc) {$\hat{x}_0$};
    \node at (-1*1.5, 0.25*1.5) (x0desc) {$x_0$};
    \draw[scale=1.5,->] (qoptdesc) .. controls ([xshift=1cm] qoptdesc) and
                                                ([xshift=-1cm] qopt) .. (qopt);
    \draw[scale=1.5,->] (loptdesc) .. controls ([xshift=-1cm] loptdesc) and
                                                 ([yshift=1cm] lopt) .. (lopt);
    \draw[scale=1.5,->] (x0desc) .. controls ([xshift=1cm] x0desc) and
                                                      ([xshift=-1cm] d) .. (d);

    \draw[scale=1.5,fill] (qopt) circle [radius=0.02];
    \draw[scale=1.5,fill] (alpha) circle [radius=0.02];

    \draw [red, decorate,decoration={brace,mirror}] (d) -- (alpha);

    % draw the quadratic objective function
    \draw[scale=1.5,thin, dashed] plot[id=qobjalpha, raw gnuplot] function {
        f(x,y) = x**2 + y**2 - 2*x - 2*y + 1.8;
        set xrange[-1:2];
        set yrange[-2:2];
        set view 0,0;
        set isosamples 1000,1000;
        set size square;
        set cont base;
        set cntrparam levels incre 0,0.1,0;
        unset surface;
        splot f(x,y);
    };
%    \draw[thin, dashed] plot[id=qobj2, raw gnuplot] function {
%        f(x,y) = x**2 + y**2 - 2*x - 2*y + 1.9;
%        set xrange[-1:2];
%        set yrange[-2:2];
%        set view 0,0;
%        set isosamples 1000,1000;
%        set size square;
%        set cont base;
%        set cntrparam levels incre 0,0.1,0;
%        unset surface;
%        splot f(x,y);
%    };
%    \draw[thin, dashed] plot[id=qobj3, raw gnuplot] function {
%        f(x,y) = x**2 + y**2 - 2*x - 2*y + 1.98;
%        set xrange[-1:2];
%        set yrange[-2:2];
%        set view 0,0;
%        set isosamples 1000,1000;
%        set size square;
%        set cont base;
%        set cntrparam levels incre 0,0.1,0;
%        unset surface;
%        splot f(x,y);
%    };
\end{tikzpicture}

\end{center}
\begin{itemize}
\item $\alpha = 0.6$
\item $x_1 = 0.4x_0 + 0.6 \hat{x}_0 = (1.2, 0.6)$
\item $T_1 = 0.4x - 0.8y - 1.8$
\end{itemize}
\end{frame}




\begin{frame}{$\mathcal{L}_1$}
\begin{center}
\begin{figure}[htbp]
\scalebox{1}{
\begin{minipage}{0.4\linewidth}
\centering
\begin{tikzpicture}
    % grid and axes
    \draw[->,name path=xaxis] (-0.2,0) -- (2.2,0) node[right] {$x$};
    \draw[->,name path=yaxis] (0,-0.2) -- (0,2.2) node[above] {$y$};

    % draw lines
    \draw[name path=line1, domain=0.6:2.5] plot(\x,{-\x + 3})
                                                    node[above right=0.2cm] {};
    \draw[name path=line2, domain=-0.2:1.6] plot(\x,{0.33*\x + 1.33})
                                                    node[above right=0.2cm] {};
    \draw[name path=line3, domain=0.7:2.3] plot(\x,{\x - 1})
                                                    node[above right=0.2cm] {};

    % calculate intersection points
    \node[name intersections={of=line1 and line2}] (a) at (intersection-1) {};
    \node[name intersections={of=line1 and line3}] (b) at (intersection-1) {};
    \node[name intersections={of=xaxis and line3}] (c) at (intersection-1) {};
    \node[name intersections={of=xaxis and yaxis}] (d) at (intersection-1) {};
    \node[name intersections={of=line2 and yaxis}] (e) at (intersection-1) {};

    % write the coordinates of the corners.
    \path let \p0 = (a) in node [above right=0.0cm of a] {($1.25$, $1.75$)};
    \path let \p0 = (b) in node [right=0.1cm of b] {\printpoint{\x0}{\y0}};
    \path let \p0 = (c) in node [below right=-0.35cm of c]
                                                       {\printpoint{\x0}{\y0}};
    \path let \p0 = (d) in node [left=0.0cm of d] {\printpoint{\x0}{\y0}};
    \path let \p0 = (e) in node [left=0.0cm of e] {($0$, $1.\overline{3}$)};
    %\path let \p0 = (e) in node [left=0.0cm of e] {\printpoint{\x0}{\y0}};

    \node at (1, 1) (qopt) {};
    \node at (0, 1.33) (lopt) {};
    \node at (1.2,0.6) (x1) {};
    \node[red] at (-0.5, 1.95) (loptdesc) {$\hat{x}_1$};
    \node at (-0.5, 0.75) (qoptdesc) {$x^*$};
    \node at (1.95, 0.35) (x1desc) {$x_1$};
    \draw[->] (qoptdesc) .. controls ([xshift=1cm] qoptdesc)
                                        and ([xshift=-1cm] qopt) .. (qopt);
    \draw[->] (loptdesc) .. controls ([xshift=0.5cm] loptdesc)
                                        and ([xshift=-0.5cm] lopt) .. (lopt);
    \draw[->] (x1desc) .. controls ([yshift=0.5cm] x1desc)
                                        and ([xshift=0.5cm] x1) .. (x1);

    \draw[fill] (qopt) circle [radius=0.02];

    \draw[fill] (x1) circle [radius=0.02];

    % draw the big polygin
    \fill[very thick,fill=blue,fill opacity=0.3] (a.center) -- (b.center)
                                              -- (c.center) -- (d.center)
                                              -- (e.center) -- cycle;

    \draw[thick, red, name path=lobj, domain=0.8:1.6] plot(\x, {0.5*\x}) {};

    % draw the quadratic objective function
    \draw[thin, dashed] plot[id=qobj1, raw gnuplot] function {
        f(x,y) = x**2 + y**2 - 2*x - 2*y + 1.75;
        set xrange[-1:2];
        set yrange[-2:2];
        set view 0,0;
        set isosamples 1000,1000;
        set size square;
        set cont base;
        set cntrparam levels incre 0,0.1,0;
        unset surface;
        splot f(x,y);
    };
    \draw[thin, dashed] plot[id=qobj2, raw gnuplot] function {
        f(x,y) = x**2 + y**2 - 2*x - 2*y + 1.9;
        set xrange[-1:2];
        set yrange[-2:2];
        set view 0,0;
        set isosamples 1000,1000;
        set size square;
        set cont base;
        set cntrparam levels incre 0,0.1,0;
        unset surface;
        splot f(x,y);
    };
    \draw[thin, dashed] plot[id=qobj3, raw gnuplot] function {
        f(x,y) = x**2 + y**2 - 2*x - 2*y + 1.98;
        set xrange[-1:2];
        set yrange[-2:2];
        set view 0,0;
        set isosamples 1000,1000;
        set size square;
        set cont base;
        set cntrparam levels incre 0,0.1,0;
        unset surface;
        splot f(x,y);
    };
\end{tikzpicture}
\end{minipage}
}
\scalebox{0.9}{
\begin{minipage}{0.6\linewidth}
\centering
\[
\begin{array}{lcrcrl}
    \textrm{Maximize}   &-& 0.4 x &+& 0.8 y \\
    \textrm{subject to} & &     x &+&     y & \leq 3 \\
    \textrm{and}        & &     x &-&     y & \leq 1 \\
    \textrm{and}        &-&     x &+&   3 y & \leq 4 \\
    \textrm{and}        & &     x &,&     y & \geq 0
\end{array}
\]
\end{minipage}
}
\caption{$\mathcal{L}_1$}
\label{fig:lp2}
\end{figure}

\end{center}
\begin{itemize}
\item $\alpha = 0.27$
\item $x_2 = (0.88, 0.8)$
\item $T_2 = -0.25x - 0.4y - 0.4$
\end{itemize}
\end{frame}



\begin{frame}{$\mathcal{L}_2$}
\begin{center}
\begin{tikzpicture}
    % grid and axes
    \draw[->,name path=xaxis] (-0.2,0) -- (2.2,0) node[right] {$x$};
    \draw[->,name path=yaxis] (0,-0.2) -- (0,2.2) node[above] {$y$};

    % draw lines
    \draw[name path=line1, domain=0.6:2.5] plot(\x,{-\x + 3})
                                                    node[above right=0.2cm] {};
    \draw[name path=line2, domain=-0.2:1.6] plot(\x,{0.33*\x + 1.33})
                                                    node[above right=0.2cm] {};
    \draw[name path=line3, domain=0.7:2.3] plot(\x,{\x - 1})
                                                    node[above right=0.2cm] {};

    % calculate intersection points
    \node[name intersections={of=line1 and line2}] (a) at (intersection-1) {};
    \node[name intersections={of=line1 and line3}] (b) at (intersection-1) {};
    \node[name intersections={of=xaxis and line3}] (c) at (intersection-1) {};
    \node[name intersections={of=xaxis and yaxis}] (d) at (intersection-1) {};
    \node[name intersections={of=line2 and yaxis}] (e) at (intersection-1) {};

    % write the coordinates of the corners.
    \path let \p0 = (a) in node [above right=0.0cm of a] {($1.25$, $1.75$)};
    \path let \p0 = (b) in node [right=0.1cm of b] {\printpoint{\x0}{\y0}};
    \path let \p0 = (c) in node [below right=-0.35cm of c]
                                                       {\printpoint{\x0}{\y0}};
    \path let \p0 = (d) in node [left=0.0cm of d] {\printpoint{\x0}{\y0}};
    \path let \p0 = (e) in node [left=0.0cm of e] {($0$, $1.\overline{3}$)};
    %\path let \p0 = (e) in node [left=0.0cm of e] {\printpoint{\x0}{\y0}};

    \node at (1, 1) (qopt) {};
    \node at (1.25, 1.75) (lopt) {};
    \node at (0.88,0.8) (x1) {};
    \node[red] at (0.4, 1.95) (loptdesc) {$\hat{x}_2$};
    \node at (-0.5, 0.75) (qoptdesc) {$x^*$};
    \node at (1.95, 0.35) (x1desc) {$x_2$};
    \draw[->] (qoptdesc) .. controls ([xshift=1cm] qoptdesc)
                                        and ([xshift=-1cm] qopt) .. (qopt);
    \draw[->] (loptdesc) .. controls ([xshift=0.5cm] loptdesc)
                                        and ([xshift=-0.5cm] lopt) .. (lopt);
    \draw[->] (x1desc) .. controls ([yshift=0.5cm] x1desc)
                                        and ([xshift=0.5cm] x1) .. (x1);
    
    \draw[fill] (qopt) circle [radius=0.02];

    \draw[fill] (x1) circle [radius=0.02];

    % draw the big polygin
    \fill[very thick,fill=blue,fill opacity=0.3] (a.center) -- (b.center)
                                              -- (c.center) -- (d.center)
                                              -- (e.center) -- cycle;

    \draw[thick, red, name path=lobj, domain=0.48:1.28] plot(\x, {-0.625*\x + 1.35})
                                                                            {};

    % draw the quadratic objective function
    \draw[thin, dashed] plot[id=qobj1, raw gnuplot] function {
        f(x,y) = x**2 + y**2 - 2*x - 2*y + 1.75;
        set xrange[-1:2];
        set yrange[-2:2];
        set view 0,0;
        set isosamples 1000,1000;
        set size square;
        set cont base;
        set cntrparam levels incre 0,0.1,0;
        unset surface;
        splot f(x,y);
    };
    \draw[thin, dashed] plot[id=qobj2, raw gnuplot] function {
        f(x,y) = x**2 + y**2 - 2*x - 2*y + 1.9;
        set xrange[-1:2];
        set yrange[-2:2];
        set view 0,0;
        set isosamples 1000,1000;
        set size square;
        set cont base;
        set cntrparam levels incre 0,0.1,0;
        unset surface;
        splot f(x,y);
    };
    \draw[thin, dashed] plot[id=qobj3, raw gnuplot] function {
        f(x,y) = x**2 + y**2 - 2*x - 2*y + 1.98;
        set xrange[-1:2];
        set yrange[-2:2];
        set view 0,0;
        set isosamples 1000,1000;
        set size square;
        set cont base;
        set cntrparam levels incre 0,0.1,0;
        unset surface;
        splot f(x,y);
    };
\end{tikzpicture}

\end{center}
\begin{itemize}
\item $\alpha = 0.23$
\item $x_3 = (0.96, 1.02)$
\end{itemize}
\end{frame}



\begin{frame}{Sti}
\begin{center}
\begin{tikzpicture}
    % grid and axes
    \draw[scale=1.5,->,name path=xaxis] (-0.2,0) -- (2.2,0) node[right] {$x$};
    \draw[scale=1.5,->,name path=yaxis] (0,-0.2) -- (0,2.2) node[above] {$y$};

    % draw lines
    \draw[scale=1.5,name path=line1, domain=0.6:2.5] plot(\x,{-\x + 3})
                                                    node[above right=0.2cm] {};
    \draw[scale=1.5,name path=line2, domain=-0.2:1.6] plot(\x,{0.33*\x + 1.33})
                                                    node[above right=0.2cm] {};
    \draw[scale=1.5,name path=line3, domain=0.7:2.3] plot(\x,{\x - 1})
                                                    node[above right=0.2cm] {};

    % calculate intersection points
    \node[name intersections={of=line1 and line2}] (a) at (intersection-1) {};
    \node[name intersections={of=line1 and line3}] (b) at (intersection-1) {};
    \node[name intersections={of=xaxis and line3}] (c) at (intersection-1) {};
    \node[name intersections={of=xaxis and yaxis}] (d) at (intersection-1) {};
    \node[name intersections={of=line2 and yaxis}] (e) at (intersection-1) {};

    % draw the big polygon
    \fill[very thick,fill=tleblue] (a.center) -- (b.center) --
                                                 (c.center) -- (d.center) --
                                                 (e.center) -- cycle;

    \node at (1*1.5, 1*1.5) (qopt) {};
    \node at (0, 0) (x0) {};
    \node at (1.2*1.5,0.6*1.5) (x1) {};
    \node at (0.88*1.5,0.8*1.5) (x2) {};
    \node at (0.96*1.5,1.02*1.5) (x3) {};

    \draw[scale=1.5,fill] (x1) circle [radius=0.02];
    \draw[scale=1.5,fill] (x2) circle [radius=0.02];
    \draw[scale=1.5,fill] (x3) circle [radius=0.02];
    \draw[scale=1.5,fill] (qopt) circle [radius=0.02];

    \draw[scale=1.5,-,thin] (x0.center) -- (x1.center) -- (x2.center) --
                            (x3.center) -- (qopt.center);

    % draw the quadratic objective function
    \draw[scale=1.5,thin, dashed] plot[id=qobjpath1, raw gnuplot] function {
        f(x,y) = x**2 + y**2 - 2*x - 2*y + 1.8;
        set xrange[-1:2];
        set yrange[-2:2];
        set view 0,0;
        set isosamples 1000,1000;
        set size square;
        set cont base;
        set cntrparam levels incre 0,0.1,0;
        unset surface;
        splot f(x,y);
    };
    \draw[scale=1.5,thin, dashed] plot[id=qobjpath2, raw gnuplot] function {
        f(x,y) = x**2 + y**2 - 2*x - 2*y + 1.9456;
        set xrange[-1:2];
        set yrange[-2:2];
        set view 0,0;
        set isosamples 1000,1000;
        set size square;
        set cont base;
        set cntrparam levels incre 0,0.1,0;
        unset surface;
        splot f(x,y);
    };
    \draw[scale=1.5,thin, dashed] plot[id=qobjpath3, raw gnuplot] function {
        f(x,y) = x**2 + y**2 - 2*x - 2*y + 1.998;
        set xrange[-1:2];
        set yrange[-2:2];
        set view 0,0;
        set isosamples 1000,1000;
        set size square;
        set cont base;
        set cntrparam levels incre 0,0.1,0;
        unset surface;
        splot f(x,y);
    };
\end{tikzpicture}

\end{center}
\end{frame}



\begin{frame}{Endrer et sidekrav}
\begin{center}
\begin{tikzpicture}
    % grid and axes
    \draw[->,name path=xaxis] (-0.2,0) -- (2.2,0) node[right] {$x$};
    \draw[->,name path=yaxis] (0,-0.2) -- (0,2.2) node[above] {$y$};

    % draw lines
    \draw[name path=line2, domain=-0.1:1.7] plot(\x,{0.33*\x})
                                                    node[above right=0.2cm] {};
    \draw[name path=line3, domain=0.9:1.7] plot(\x,{\x - 1})
                                                    node[above right=0.2cm] {};

    % calculate intersection points
    \node[name intersections={of=xaxis and yaxis}] (a) at (intersection-1) {};
    \node[name intersections={of=line2 and line3}] (b) at (intersection-1) {};
    \node[name intersections={of=xaxis and line3}] (c) at (intersection-1) {};

    % write the coordinates of the corners.
    \path let \p0 = (a) in node [left=-0.0cm of a] {\printpoint{\x0}{\y0}};
    \path let \p0 = (b) in node [right=0.15cm of b] {(1.5,0.5)};
    \path let \p0 = (c) in node [below=-0.2cm of c] {\printpoint{\x0}{\y0}};

    \node at (1, 1) (xu) {};
    \node at (0, 0) (xhat) {};
    \node at (1.2, 0.4) (xstar) {};
    \node at (1.5,0.5) (x0) {};
    \node at (1.95, 1.3) (xudesc) {$x^u$};
    \node at (0.4, 0.4) (xstardesc) {$x^*$};
    \node[red] at (-0.4, -1.0) (xhatdesc) {$\hat{x}_0$};
    \node at (1.8, -0.3) (x0desc) {$x_0$};
    \draw[->] (xudesc) .. controls ([xshift=-1cm] xudesc)
                                        and ([xshift=1cm] xu) .. (xu);
    \draw[->] (x0desc) .. controls ([yshift=0.1cm] x0desc)
                                        and ([xshift=0.5cm] x0) .. (x0);
%    \draw[->] (loptdesc) .. controls ([xshift=0.5cm] loptdesc)
%                                        and ([xshift=-0.5cm] lopt) .. (lopt);
    \draw[->] (xhatdesc) .. controls ([yshift=1cm] xhatdesc)
                                        and ([yshift=-1cm] xhat) .. (xhat);
    \draw[->] (xstardesc) .. controls ([xshift=1cm] xstardesc)
                                        and ([xshift=-1cm] xstar) .. (xstar);
    
    \draw[fill] (xu) circle [radius=0.02];

    \draw[fill] (xstar) circle [radius=0.02];

    % draw the big polygin
%    \fill[very thick,fill=blue,fill opacity=0.3] (a.center) -- (b.center)
%                                              -- (c.center) -- (d.center)
%                                              -- (e.center) -- cycle;

    \fill[very thick,fill=blue,fill opacity=0.3] (a.center) -- (b.center)
                                              -- (c.center) -- cycle;

%    \draw[thick, red, name path=lobj, domain=0.48:1.28] plot(\x, {-0.625*\x + 1.35})
%                                                                            {};

    % draw the quadratic objective function
    \draw[thin, dashed] plot[id=qobj1, raw gnuplot] function {
        f(x,y) = x**2 + y**2 - 2*x - 2*y + 1.75;
        set xrange[-1:2];
        set yrange[-2:2];
        set view 0,0;
        set isosamples 1000,1000;
        set size square;
        set cont base;
        set cntrparam levels incre 0,0.1,0;
        unset surface;
        splot f(x,y);
    };
    \draw[thin, dashed] plot[id=qobj2, raw gnuplot] function {
        f(x,y) = x**2 + y**2 - 2*x - 2*y + 1.9;
        set xrange[-1:2];
        set yrange[-2:2];
        set view 0,0;
        set isosamples 1000,1000;
        set size square;
        set cont base;
        set cntrparam levels incre 0,0.1,0;
        unset surface;
        splot f(x,y);
    };
    \draw[thin, dashed] plot[id=qobj3, raw gnuplot] function {
        f(x,y) = x**2 + y**2 - 2*x - 2*y + 1.98;
        set xrange[-1:2];
        set yrange[-2:2];
        set view 0,0;
        set isosamples 1000,1000;
        set size square;
        set cont base;
        set cntrparam levels incre 0,0.1,0;
        unset surface;
        splot f(x,y);
    };
\end{tikzpicture}

\end{center}
\begin{itemize}
\item $-x + 3y \leq 4$
\item $-x + 3y \leq 0$
\item $\alpha = 0.2$
\item $x_1 = 0.8x_0 + 0.2 \hat{x}_0 = (1.2,0.4) = x^*$
\end{itemize}
\end{frame}



\begin{frame}{Like optimale løsninger}
\begin{itemize}
\item En mengde $\mathcal{M}_k$ med grener som faller ut.
\item En subinstans $\mathcal{Q}_k$ definert av $\mathcal{Q}$ og $\mathcal{M}_k$
\item Optimal løsning til $\mathcal{Q}_k$ noteres som $x_k^*$
\item En mengde med variabler som er $0$ i $x_k^*$ noteres som $\mathcal{Z}_k$.
\item $2^n - 1$ subinstanser. $\left| \mathcal{Z}_0 \right| = 1749$ i \textit{vlarge}.
\item $2^{1749} \approx 3 \times 10^{526}$ subinstanser har løsning $x_0^*$
\end{itemize}
\end{frame}



\begin{frame}{Søker etter $\mathcal{M}_{14} = \left\{  2,3,4 \right\}$}
\begin{center}
\begin{tikzpicture}[level/.style={sibling distance=10mm/#1, level distance=10mm}]
\tikzset{every node/.style={shape=circle,fill=black!25,minimum size=7mm}}
%\tikzset{every node/.style={shape=circle,
%                            font=\bfseries \Large,
%                            minimum size=3cm,
%                            scale=0.4
%                           }}
\node[fill=red!40] (root) {$0$}
    child {
        node {$2$}
        child {
            node {$3$}
            child {
                node {$7$}
            }
        };
    \path edge from parent[draw,line width=3pt,-,red!20];
    }
    child {
        node {$8$}
        child {
            node {$12$}
            child {
                node {$28$}
                child {node {$29$}
                }
            }
        };
    \path edge from parent[draw,line width=3pt,-,red!20];
    }
    child {node {$16$}}
    child {
        node {$10$}
        child {
            node {$11$}
        };
    \path edge from parent[draw,line width=3pt,-,red!20];
    }
    ;
\end{tikzpicture}

\[
\small
    \begin{array}{ll}
        \mathcal{M}_0    = \left\{{}\right\}           & \mathcal{Z}_0    = \left\{{1,3}\right\} \\
        \mathcal{M}_2    = \left\{{2}\right\}          & \mathcal{Z}_2    = \left\{{2,3,5}\right\} \\
        \mathcal{M}_3    = \left\{{1,2}\right\}        & \mathcal{Z}_3    = \left\{{1,2,4,5}\right\} \\
        \mathcal{M}_7    = \left\{{1,2,3}\right\}      & \mathcal{Z}_7    = \left\{{1,2,3,5}\right\} \\
        \mathcal{M}_8    = \left\{{4}\right\}          & \mathcal{Z}_8    = \left\{{1,4,5}\right\} \\
        \mathcal{M}_{10} = \left\{{2,4}\right\}        & \mathcal{Z}_{10} = \left\{{2,3,4,5}\right\}
    \end{array}
~
    \begin{array}{ll}
        \mathcal{M}_{12} = \left\{{3,4}\right\}        & \mathcal{Z}_{12} = \left\{{4,3,1}\right\} \\
        \mathcal{M}_{15} = \left\{{1,2,3,4}\right\}    & \mathcal{Z}_{15} = \left\{{1,2,3,4}\right\} \\
        \mathcal{M}_{16} = \left\{{5}\right\}          & \mathcal{Z}_{16} = \left\{{1,3,5}\right\} \\
        \mathcal{M}_{28} = \left\{{3,4,5}\right\}      & \mathcal{Z}_{28} = \left\{{2,3,4,5}\right\} \\
        \mathcal{M}_{29} = \left\{{1,3,4,5}\right\}    & \mathcal{Z}_{29} = \left\{{1,2,3,4,5}\right\}
    \end{array}
\]
\end{center}
\end{frame}



\begin{frame}{Søker etter $\mathcal{M}_{14} = \left\{  2,3,4 \right\}$}
\begin{center}
\begin{tikzpicture}
[level 1/.style={sibling distance=14mm, level distance=10mm},
level 2/.style={sibling distance=9mm, level distance=10mm}
]
\tikzset{every node/.style={shape=circle,fill=black!25,minimum size=7mm}}
%\tikzset{every node/.style={shape=circle,
%                            font=\bfseries \Large,
%                            minimum size=3cm,
%                            scale=0.4
%                           }}
\node[fill=red!40] (root) {$0$}
    child {
        node[fill=red!40] {$2$}
        child {
            node[fill=red!40] {$3$}
            child {
                node {$7$}
            };
            \path edge from parent[draw,line width=3pt,-,red!20];
        };
    \path edge from parent[draw,line width=3pt,-,red!20];
    }
    child {
        node {$8$}
        child {
            node {$10$}
        }
        child {
            node {$12$}
            child {
                node {$15$}
            }
        };
    \path edge from parent[draw,line width=3pt,-,red!20];
    }
    child {
        node {$16$}
        child {
            node {$28$}
            child {
                node {$29$}
            }
        }
    }
    ;
\end{tikzpicture}

\[
\small
    \begin{array}{ll}
        \mathcal{M}_0    = \left\{{}\right\}           & \mathcal{Z}_0    = \left\{{1,3}\right\} \\
        \mathcal{M}_2    = \left\{{2}\right\}          & \mathcal{Z}_2    = \left\{{2,3,5}\right\} \\
        \mathcal{M}_3    = \left\{{1,2}\right\}        & \mathcal{Z}_3    = \left\{{1,2,4,5}\right\} \\
        \mathcal{M}_7    = \left\{{1,2,3}\right\}      & \mathcal{Z}_7    = \left\{{1,2,3,5}\right\} \\
        \mathcal{M}_8    = \left\{{4}\right\}          & \mathcal{Z}_8    = \left\{{1,4,5}\right\} \\
        \mathcal{M}_{10} = \left\{{2,4}\right\}        & \mathcal{Z}_{10} = \left\{{2,3,4,5}\right\}
    \end{array}
~
    \begin{array}{ll}
        \mathcal{M}_{12} = \left\{{3,4}\right\}        & \mathcal{Z}_{12} = \left\{{4,3,1}\right\} \\
        \mathcal{M}_{15} = \left\{{1,2,3,4}\right\}    & \mathcal{Z}_{15} = \left\{{1,2,3,4}\right\} \\
        \mathcal{M}_{16} = \left\{{5}\right\}          & \mathcal{Z}_{16} = \left\{{1,3,5}\right\} \\
        \mathcal{M}_{28} = \left\{{3,4,5}\right\}      & \mathcal{Z}_{28} = \left\{{2,3,4,5}\right\} \\
        \mathcal{M}_{29} = \left\{{1,3,4,5}\right\}    & \mathcal{Z}_{29} = \left\{{1,2,3,4,5}\right\}
    \end{array}
\]
\end{center}
\end{frame}



\begin{frame}{Søker etter $\mathcal{M}_{14} = \left\{  2,3,4 \right\}$}
\begin{center}
\begin{tikzpicture}[level/.style={sibling distance=10mm/#1, level distance=10mm}]
\tikzset{every node/.style={shape=circle,fill=black!25,minimum size=7mm}}
%\tikzset{every node/.style={shape=circle,
%                            font=\bfseries \Large,
%                            minimum size=3cm,
%                            scale=0.4
%                           }}
\node[fill=red!40] (root) {$0$}
    child {
        node {$1$}
        child {
            node {$5$}
            child {
                node {$8$
                }
            }
        }
    }
    child[fill=red!40] {
        node[fill=red!40] {$2$}
        child {
            node[fill=red!40] {$6$}
            child {
                node {$9$}
                child {node {$10$}
                };
                \path edge from parent[draw,line width=3pt,-,red!30];
            };
            \path edge from parent[draw,line width=3pt,-,red!30];
        };
    \path edge from parent[draw,line width=3pt,-,red!30];
    }
    child {node {$3$}}
    child {
        node {$4$}
        child {
            node {$7$}
        };
    \path edge from parent[draw,line width=3pt,-,red!30];
    }
    ;
\end{tikzpicture}

\[
\small
    \begin{array}{ll}
        \mathcal{M}_0    = \left\{{}\right\}           & \mathcal{Z}_0    = \left\{{1,3}\right\} \\
        \mathcal{M}_2    = \left\{{2}\right\}          & \mathcal{Z}_2    = \left\{{2,3,5}\right\} \\
        \mathcal{M}_3    = \left\{{1,2}\right\}        & \mathcal{Z}_3    = \left\{{1,2,4,5}\right\} \\
        \mathcal{M}_7    = \left\{{1,2,3}\right\}      & \mathcal{Z}_7    = \left\{{1,2,3,5}\right\} \\
        \mathcal{M}_8    = \left\{{4}\right\}          & \mathcal{Z}_8    = \left\{{1,4,5}\right\} \\
        \mathcal{M}_{10} = \left\{{2,4}\right\}        & \mathcal{Z}_{10} = \left\{{2,3,4,5}\right\}
    \end{array}
~
    \begin{array}{ll}
        \mathcal{M}_{12} = \left\{{3,4}\right\}        & \mathcal{Z}_{12} = \left\{{4,3,1}\right\} \\
        \mathcal{M}_{15} = \left\{{1,2,3,4}\right\}    & \mathcal{Z}_{15} = \left\{{1,2,3,4}\right\} \\
        \mathcal{M}_{16} = \left\{{5}\right\}          & \mathcal{Z}_{16} = \left\{{1,3,5}\right\} \\
        \mathcal{M}_{28} = \left\{{3,4,5}\right\}      & \mathcal{Z}_{28} = \left\{{2,3,4,5}\right\} \\
        \mathcal{M}_{29} = \left\{{1,3,4,5}\right\}    & \mathcal{Z}_{29} = \left\{{1,2,3,4,5}\right\}
    \end{array}
\]
\end{center}
\end{frame}



\begin{frame}{Søker etter $\mathcal{M}_{14} = \left\{  2,3,4 \right\}$}
\begin{center}
\begin{tikzpicture}[level/.style={sibling distance=10mm/#1, level distance=10mm}]
\tikzset{every node/.style={shape=circle,fill=black!25,minimum size=7mm}}
%\tikzset{every node/.style={shape=circle,
%                            font=\bfseries \Large,
%                            minimum size=3cm,
%                            scale=0.4
%                           }}
\node[fill=red!40] (root) {$0$}
    child {
        node {$2$}
        child {
            node {$8$}
            child {
                node {$7$
                }
            }
        }
    }
    child {
        node {$8$}
        child {
            node {$12$}
            child {
                node {$28$}
                child {node {$29$}
                }
            }
        }
    }
    child {node {$16$}}
    child {
        node[fill=red!40] {$10$}
        child {
            node[fill=red!40] {$11$
            };
            \path edge from parent[draw,line width=3pt,-,red!30];
        };
        \path edge from parent[draw,line width=3pt,-,red!30];
    }
    ;
\end{tikzpicture}

\[
\small
    \begin{array}{ll}
        \mathcal{M}_0    = \left\{{}\right\}           & \mathcal{Z}_0    = \left\{{1,3}\right\} \\
        \mathcal{M}_2    = \left\{{2}\right\}          & \mathcal{Z}_2    = \left\{{2,3,5}\right\} \\
        \mathcal{M}_3    = \left\{{1,2}\right\}        & \mathcal{Z}_3    = \left\{{1,2,4,5}\right\} \\
        \mathcal{M}_7    = \left\{{1,2,3}\right\}      & \mathcal{Z}_7    = \left\{{1,2,3,5}\right\} \\
        \mathcal{M}_8    = \left\{{4}\right\}          & \mathcal{Z}_8    = \left\{{1,4,5}\right\} \\
        \mathcal{M}_{10} = \left\{{2,4}\right\}        & \mathcal{Z}_{10} = \left\{{2,3,4,5}\right\}
    \end{array}
~
    \begin{array}{ll}
        \mathcal{M}_{12} = \left\{{3,4}\right\}        & \mathcal{Z}_{12} = \left\{{4,3,1}\right\} \\
        \mathcal{M}_{15} = \left\{{1,2,3,4}\right\}    & \mathcal{Z}_{15} = \left\{{1,2,3,4}\right\} \\
        \mathcal{M}_{16} = \left\{{5}\right\}          & \mathcal{Z}_{16} = \left\{{1,3,5}\right\} \\
        \mathcal{M}_{28} = \left\{{3,4,5}\right\}      & \mathcal{Z}_{28} = \left\{{2,3,4,5}\right\} \\
        \mathcal{M}_{29} = \left\{{1,3,4,5}\right\}    & \mathcal{Z}_{29} = \left\{{1,2,3,4,5}\right\}
    \end{array}
\]
\end{center}
\end{frame}



\begin{frame}{Algoritme: \texttt{find}}
\begin{algorithm}[H]
\caption{\texttt{find($\mathcal{M}_l$, $v_k$)}}
\label{alg:find}
\begin{algorithm}[H]
\caption{\texttt{find($\mathcal{M}_l$, $v_k$)}}
\label{alg:find}
%\KwIn{($\mathcal{M}_l$, $\mathcal{Q}_k$)}
\SetAlgoVlined
\ForEach{child vertex $v_i$ of $v_k$}{
  \If{$\mathcal{M}_i \subseteq \mathcal{M}_l$}{
    \eIf{$\mathcal{M}_l \subseteq \mathcal{Z}_i$}{
      \KwRet{$i$}
    }{
      \KwRet{\texttt{find($\mathcal{M}_l$, $v_i$)}}
    }
  }
}
\KwRet{$-1$}
\end{algorithm}

\end{algorithm}
\end{frame}


\begin{frame}{Eksperiment 1}
%\begin{itemize}
%\item Vi løser \textit{small}, med $\sigma(2, 238) = 28 442$ subinstanser
%\end{itemize}
Vi løser \textit{small}, med $\sigma(2, 238) = 28 442$ subinstanser
\begin{table}[ht!]
\centering
\caption{Resultater av de forskjellige implementasjonen med endrende toleranse.}
\begin{tabular}{rrrrr}
$\epsilon$ & \texttt{cClp} & \texttt{cSlp} & \texttt{nClp} & \texttt{nSlp} \\ \hline
$10^{-1}$ & 45.51 & 55.61 & 72.32 & 85.51 \\
$10^{-2}$ & 46.34 & 55.89 & 73.11 & 85.51 \\
$10^{-3}$ & 51.12 & 59.04 & 75.60 & 85.28 \\
$10^{-4}$ & 52.46 & 73.79 & 77.83 & 107.39 \\
$10^{-5}$ & 54.48 & 232.53 & 81.16 & 355.47 \\
$10^{-6}$ & 65.42 & 1363.46 & 93.29 & 2022.25 \\
$10^{-7}$ & 70.78 & 6522.91 & 100.85 & 9395.92
\end{tabular}
\label{table:expone}
\end{table}
\end{frame}



\begin{frame}{Eksperiment 1}
\begin{figure}[ht!]
    \centering
    \begin{tikzpicture}
    \begin{loglogaxis}[
        legend pos=north east]

        \addplot[black] plot coordinates {
(0.1,       45.51)
(0.01,      46.34)
(0.001,     51.12)
(0.0001,    52.46)
(0.00001,   54.48)
(0.000001,  65.42)
(0.0000001, 70.78)
        };

        \addplot[red] plot coordinates {
(0.1,       55.61)
(0.01,      55.89)
(0.001,     59.04)
(0.0001,    73.79)
(0.00001,   232.53)
(0.000001,  1363.46)
(0.0000001, 6522.91)
        };

        \addplot[green] plot coordinates {
(0.1,       72.32)
(0.01,      73.11)
(0.001,     75.60)
(0.0001,    77.83)
(0.00001,   81.16)
(0.000001,  93.29)
(0.0000001, 100.85)
        };

        \addplot[blue] plot coordinates {
(0.1,       85.51)
(0.01,      85.51)
(0.001,     85.28)
(0.0001,    107.39)
(0.00001,   355.47)
(0.000001,  2022.25)
(0.0000001, 9395.92)
        };

        \legend{\texttt{construct\_clp}, \texttt{construct\_slp}, \texttt{naive\_clp}, \texttt{naive\_slp}}
    \end{loglogaxis}
\end{tikzpicture}

    \caption{Kjøretid i CPU-sekunder for å løse \textit{small} og dens subinstanser.}
    \label{fig:smalltolerance}
\end{figure}
\end{frame}



\begin{frame}{Eksperiment 2}
\begin{table}
\centering
\caption{Kjøretid i CPU-sekunder for å løse de tre instansene.}
\label{table:eps4instances}
\begin{tabular}{crrr}
\textrm{Implementasjon} & \textit{small} & \textit{large} & \textit{vlarge} \\ \hline
\texttt{cClp}           & 0.52           & 9.55           & 76.18 \\
\texttt{cSlp}           & 0.71           & 32.88          & 585.60 \\
\texttt{nClp}           & 0.65           & 11.68          & 157.53 \\
\texttt{nSlp}           & 0.89           & 39.87          & 1173.74
\end{tabular}
\end{table}
\end{frame}



\begin{frame}{Tilfeldige instanser}
\begin{itemize}
\item $m = \lfloor \frac{7}{20}n \rfloor$
\item $b_i$ har 50\% sannsynlighet for å være null. Ellers $10 \leq | b_i | \leq 70$.
\item $h_{ii}$ har 50\% sannsynlighet for å være null. Ellers $10^{-5} \leq h_{ii} \leq 10^{-1}$.
\end{itemize}
\end{frame}



\begin{frame}{Eksperiment 3}
\begin{table}[ht!]
    \centering
    \caption{Kjøretid for å løse tilfeldige instanser med økende $n$. $\beta = 1$.}
    \label{table:expfour}
\begin{tabular}{rrrc}
    $n$ & \texttt{cClp}  & \texttt{nClp}  & Relativ Speedup \\ \hline
    500 & 4.9   & 5.9   & 16.9\% \\
   1000 & 42.1  & 53.0  & 20.6\% \\
   1500 & 181.5 & 234.5 & 22.6\% \\
   2000 & 547.1 & 710.2 & 23.0\%
\end{tabular}
\end{table}
\end{frame}



\begin{frame}{Eksperiment 4}
\begin{table}[ht!]
\centering
\caption{Kjøretid i CPU-sekunder for $n = 50$ og økende $\beta$.}
\begin{tabular}{rrrcr}
      $\beta$ & \texttt{cClp} & \texttt{nClp} & Relativ Speedup & Distinkte løsninger\\ \hline
       1  & 0.03 & 0.04 & 25.0\% & 37.4 ($74.3$\%) \\
       2  & 0.64 & 0.94 & 31.9\% & 744.3 ($58.3$\%) \\
       3  & 7.06 & 15.90 & 55.6\% & 9484.7 ($45.4$\%) \\
       4  & 77.59 & 188.83 & 58.9\% & 82262.5 ($32.8$\%) \\
       5  & 586.54 & 1758.23 & 66.6\% & 574685.0 ($24.2$\%) \\
\end{tabular}
\label{table:exptwo}
\end{table}
\end{frame}



\begin{frame}{Eksperiment 4}
\begin{tikzpicture}
    \begin{axis}[
        xlabel=$\beta$,
        ylabel=CPU-seconds,
        legend pos=south west]

        \addplot[black] plot coordinates {
(1, 0.0006)
(2, 0.000501961)
(3, 0.000338204)
(4, 0.000308908)
(5, 0.000247492)
        };

        \addplot[red] plot coordinates {
(1, 0.0008)
(2, 0.000737255)
(3, 0.000761677)
(4, 0.000751787)
(5, 0.00074189)
        };

        \legend{\texttt{cClp}, \texttt{nClp}}
    \end{axis}
\end{tikzpicture}

\end{frame}



\end{document}
