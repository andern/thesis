\documentclass{beamer}
\usepackage[utf8]{inputenc}
\usepackage{tikz}
\usetikzlibrary{intersections,positioning,calc,arrows,shapes,
                decorations.pathreplacing,spy,automata}
\usepackage{pgfplots}
\usepackage[ruled]{algorithm2e}
\usetheme{boxes}
\title[Fast Solver of Closely Related Quadratic Programming Problems]
      {Fast Solver of Closely Related Quadratic Programming Problems}
\author{Andreas Halle}
%\institute{Department of Informatics}
\date{June 10, 2013}


\definecolor{tleblue}{RGB}{178,175,255}
\DeclareMathOperator*{\argmin}{arg\,min}


\begin{document}



\begin{frame}
\titlepage
\end{frame}



\section{Introduksjon}



\begin{frame}{PROMAPS}
\begin{itemize}
\item Utviklet av Goodtech og MathConsult.
\item Kalkurerer leveransepåliteligheten i et nettverk
\item Dette er formulert som en rekke veldig like QP-problemer
      (Quadratic Programming).
\item QP-løseren en flaskehals. Skjermbildet (PROMAPS) oppdaterer seg hvert
      femte minutt.
\item \url{http://www.tu.no/energi/2011/10/07/her-beregnes-risikoen-for-svikt-i-kraftnettet}
\end{itemize}
\end{frame}



%\begin{frame}{QP-problemet}
%\begin{itemize}
%\item Objektfunksjonen representerer kostnader for strømleveranse.
%\item Sidekravene representerer nettverket, hvor summen av flyten i en
%      node er lik $0$.
%\item Mer spesifikt...
%\end{itemize}
%\end{frame}



\begin{frame}{Objektfunksjonen}
\[
f(x) = x^T \Phi D x + (g - c)^T x
\]
\begin{itemize}
\item $f(x)$ representerer leveransekostnader ($E/s$)
\item $x$ representerer strømmen over grenene ($W$)
\item $\Phi$ representerer strømtap ($1/W$)
\item $D$ representerer overføringskostnader ($E/J$)
\item $g$ representerer kostnader for å generere strøm ($E/J$)
\item $c$ representerer leveransepris ($E/J$)
\end{itemize}
\end{frame}



\begin{frame}{Objektfunksjonen}
\[
f(x) = x^T H x + b^T x
\]
\begin{itemize}
\item $H = \Phi D$ representerer kostnader ($E/(W^2 s)$)
\item $b = g - c$ representerer kostnader ($E/J$)
\item $H$ er en matrise på størrelsen $n \times n$
\item $b$ og $x$ er vektorer i $\mathbb{R}^n$
\end{itemize}
\end{frame}



\begin{frame}{Optimeringsproblemet}
Vi definerer et konveks QP-problem
\[
\min_{x \in \mathbb{R}^n} f(x) \quad \textrm{subject to} ~ Ax = 0, ~ l \leq x \leq u
\]
\begin{itemize}
\item A er en $m \times n$ insidensmatrise for et strømnettverk
\item $m$ noder
\item $n$ grener
\item $l$ og $u$ er nedre og øvre grenkapasitet ($W$)
\end{itemize}
\end{frame}



\begin{frame}{Utfall}
Vi ønsker å modellere utfall.
\begin{itemize}
\item Setter $l_i = u_i = 0$
\item Goodtech ønsker å løse QP-problemet for ulike utfall
\item Definerer subinstanser for kombinasjoner av grener som faller ut
\item Instans er et QP-problem uten utfall
\item Subinstans er en instans med utfall
\end{itemize}
\end{frame}



\begin{frame}{Utfall}
\begin{itemize}
\item Vil løse så mange subinstanser som mulig.
\item Usannsynlig at det er mange utfall.
\item Vi prøver å løse alle subinstanser som har mindre eller lik $\beta$ utfall.
\end{itemize}
\[
\sigma (\beta, n) = \sum_{j=0}^{\beta} \binom{n}{j}
\]
\end{frame}



\begin{frame}{Subinstanser}
\begin{itemize}
\item En mengde av variabler som representerer utfall $\mathcal{M}_k$
\item $k = \sum_{j \in \mathcal{M}_k}^{} 2^{j-1}$
\item Eks. $\mathcal{M}_k = \left\{1,3,5 \right\}$. $k = 21$
\item En subinstans $\mathcal{Q}_k$ defineres av $\mathcal{M}_k$
\item Optimal løsning til subinstans $\mathcal{Q}_k$ noterer vi som $x_k^*$
\end{itemize}
\end{frame}



\begin{frame}{Instanser fra Goodtech}
\input{include/newhistH}
\end{frame}



\begin{frame}{Instanser fra Goodtech}
\begin{table}[ht!]
    \centering
    \caption{Størrelse for hver instans}
    \begin{tabular}{lrrr}
    Problemstørrelse & \textit{small} & \textit{large} & \textit{vlarge} \\\hline
    Rader        & 82             & 328            & 1127 \\
    Kolonne      & 238            & 952            & 3437 \\
    Ikke-nuller A & 348            & 1392           & 4840 \\
    Ikke-nuller H & 108            & 432            & 894 \\
    \end{tabular}
    \label{table:sizes}
\end{table}
\begin{table}[ht!]
    \centering
    \caption{Verdier i objektfunksjonen}

    \begin{tabular}{lrr}
      & \textit{small} og \textit{large}         & \textit{vlarge} \\\hline
    $\max(h_{ii})$      & $2.9614 \times 10^{-2}$ & $4.9011 \times 10^{-2}$ \\
    $\min(h_{ii})$      & $4.9290 \times 10^{-5}$ & $1.1026 \times 10^{-5}$ \\
$\textrm{mean}(h_{ii})$ & $5.2864 \times 10^{-3}$ & $5.8984 \times 10^{-3}$ \\
    $\max(b_{i})$       & 20                      & 20 \\
    $\min(b_{i})$       & -70                     & -50 \\
    \end{tabular}
    \label{table:maxmin}
\end{table}
\end{frame}



\begin{frame}{Instanser fra Goodtech}
\begin{itemize}
\item Vi ser at det lineære leddet har mye høyere verdier enn det kvadratiske
\item Prøver å lineærisere objektfunksjonen
\item Lineær Taylor-utvikling av objektfunksjonen i punkt $a$: \\
      $T_a(x) = -a^T H a + 2a^T H x + b^T x$ \\
      $T_0(x) = b^T x$
\end{itemize}
Definerer et LP $\mathcal{L}$ for hvert QP $\mathcal{Q}$
\[
\min_{x \in \mathbb{R}^n} g(x) \quad \textrm{subject to} ~ Ax = 0, ~ l \leq x \leq u
\]
\begin{itemize}
\item $g(x) = T_0(x) = b^T x$
\end{itemize}
\end{frame}



\begin{frame}{Hvor like er $\mathcal{L}$ og $\mathcal{Q}$?}
\begin{itemize}
\item Optimal løsning til $\mathcal{Q}$ noteres $x^*$
\item Optimal løsning til $\mathcal{L}$ noteres $\hat{x}$
\item Vi noterer avvik mellom $f(\hat{x})$ og $f(x^*)$ for
      $\Delta = \left| \frac{f(\hat{x}) - f(x^*)}{f(x^*)} \right|$
\item Vi genererer tilfeldige instanser og sjekker $\Delta$
\end{itemize}
\end{frame}



\begin{frame}{$100\Delta$}
\begin{figure}[ht!]
\begin{tikzpicture}
    \begin{axis}[
        width=\textwidth*0.80,
%        grid=major,
        view={245}{30},
%        xmax=99,
%        ymax=99,
%        zmax=1,
%        restrict z to domain=0:1,
        xlabel=x,
        ylabel=y,
        zlabel=$100\Delta$]

        \addplot3[surf,mesh/rows=20,mesh/cols=20] table[x=b,y=h,z=dev]
                                                   {include/data/delta3d.dat};


%        \addplot3[
%            surf,
%            mesh/rows=101,
%            mesh/cols=100,
%        ] 
%        table[x=b,y=h,z=dev] {include/data/delta3d.dat};
\end{axis}
\end{tikzpicture}

\caption{Avvik som en funksjon av tettheten i objektfunksjonen. $x$ er prosent
         nuller på diagonalen til $H$. $y$ er prosent nuller i $b$}
\end{figure}
\end{frame}



\begin{frame}{Hvor like er $\mathcal{L}$ og $\mathcal{Q}$?}
Oppnår $95\%$ optimal verdi etter løst $\mathcal{L}$.
\begin{itemize}
\item Metode basert på successive linear programming (SLP)
\item Lar vi startverdien $x_0 = 0$, når vi rundt $95\%$ av optimal målfunksjonsverdi etter
      første iterasjon
\item Taylor-utvikling i $x_k$ noterer vi som $T_k$
\item Definerer $\mathcal{L}_k$ som LP-problemet å minimere $T_k$ underlagt
      sidekravene til $\mathcal{Q}$.
\end{itemize}
\end{frame}



\begin{frame}{Slp}
\begin{algorithm}[ht!]
\caption{\texttt{slp($x_0$, $\epsilon \ge 0$)}}
\label{alg:iter}
\SetAlgoVlined
\DontPrintSemicolon
%Given $x_0$ and some $\epsilon \ge 0$\;
Set $k \leftarrow 0$\;

\Repeat{$\displaystyle \frac{f(x_{k-1}) - f(x_k)}{|f(x_{k-1})|} \leq \epsilon$} {
    $T_k \leftarrow - x_k^THx_k + 2x_k^THx + b^Tx$ \\
    $\hat{x}_k \leftarrow \textrm{optimal solution of } \mathcal{L}_k$ (Solve) \\
    $\alpha_k \in \argmin_{\alpha \leq 1} f((1-\alpha) x_k + \alpha \hat{x}_k)$ \\
%   $\alpha_k \in \argmin_{\alpha \leq 1} f(\alpha x_k + (1-\alpha)\hat{x}_k)$\;
    $x_{k+1} \leftarrow (1-\alpha_k) x_k + \alpha_k\hat{x}_k$ \\
    $k \leftarrow k +1$
}
\end{algorithm}

\end{frame}



\begin{frame}{Et eksempel}
\[
\begin{array}{lcrcrcl}
\textrm{minimize}           & &   (x-1)^2 &+&   (y-1)^2 & - &  2 \\
\textrm{subject to}         & &         x &+&         y &\leq& 3 \\
                            & &         x &-&         y &\leq& 1 \\
                            & &         x &+&       3 y &\leq& 4 \\
                 \multicolumn{5}{r}{x,y}                &\geq& 0
\end{array}
\]
\begin{itemize}
\item $T_0 = -2 x - 2 y$ blir lineær objektfunksjon
\end{itemize}
\end{frame}



\begin{frame}{$\mathcal{L}_0$}
\begin{center}
\begin{tikzpicture}
    % grid and axes
    \draw[scale=1.5,->,name path=xaxis] (-0.2,0) -- (2.2,0) node[right] {$x$};
    \draw[scale=1.5,->,name path=yaxis] (0,-0.2) -- (0,2.2) node[above] {$y$};

    % draw lines
    \draw[scale=1.5,name path=line1, domain=0.6:2.5] plot(\x,{-\x + 3})
                                                    node[above right=0.2cm] {};
    \draw[scale=1.5,name path=line2, domain=-0.2:1.6] plot(\x,{0.33*\x + 1.33})
                                                    node[above right=0.2cm] {};
    \draw[scale=1.5,name path=line3, domain=0.7:2.3] plot(\x,{\x - 1})
                                                    node[above right=0.2cm] {};
    % calculate intersection points
    \node[name intersections={of=line1 and line2}] (a) at (intersection-1) {};
    \node[name intersections={of=line1 and line3}] (b) at (intersection-1) {};
    \node[name intersections={of=xaxis and line3}] (c) at (intersection-1) {};
    \node[name intersections={of=xaxis and yaxis}] (d) at (intersection-1) {};
    \node[name intersections={of=line2 and yaxis}] (e) at (intersection-1) {};

    % draw the big polygin
    \fill[very thick,fill=tleblue] (a.center) -- (b.center) --
                                                 (c.center) -- (d.center) --
                                                 (e.center) -- cycle;

    % write the coordinates of the corners.
    \path let \p0 = (a) in node [above right=0.0cm of a] {($1.25$, $1.75$)};
    \path let \p0 = (b) in node [right=0.1cm of b] {($2$, $1$)};
    \path let \p0 = (c) in node [below right=-0.35cm of c] {($1$, $0$)};
    %\path let \p0 = (d) in node [left=0.0cm of d] {($0$, $0$)};
    \path let \p0 = (e) in node [left=0.0cm of e] {($0$, $1.\overline{3}$)};

    \node at (1.5, 1.5) (qopt) {};
    \node at (3, 1.5) (lopt) {};
    \node at (-0.75, 1.125) (qoptdesc) {$x^*$};
    \node at (-1.5, 0.375) (x0desc) {$x_0$};
    \node[text=red] at (3.75, 2.625) (loptdesc) {$\hat{x}_0$};
    \draw[scale=1.5,->] (qoptdesc) .. controls ([xshift=1cm] qoptdesc) and ([xshift=-1cm]
                                                               qopt) .. (qopt);
    \draw[scale=1.5,->] (loptdesc) .. controls ([xshift=-1cm] loptdesc) and ([yshift=1cm]
                                                               lopt) .. (lopt);
    \draw[scale=1.5,->] (x0desc) .. controls ([xshift=1cm] x0desc) and ([xshift=-1cm] d)
                                                                     .. (d);

    \draw[scale=1.5,fill] (qopt) circle [radius=0.02];

    \draw[scale=1.5,thick, red, name path=lobj, domain=-0.4:0.4] plot(\x, {-\x}) {};

    % draw the quadratic objective function
    \draw[scale=1.5,thin, dashed] plot[id=qobj1, raw gnuplot] function {
        f(x,y) = x**2 + y**2 - 2*x - 2*y + 1.75;
        set xrange[-1:2];
        set yrange[-2:2];
        set view 0,0;
        set isosamples 1000,1000;
        set size square;
        set cont base;
        set cntrparam levels incre 0,0.1,0;
        unset surface;
        splot f(x,y);
    };
    \draw[scale=1.5,thin, dashed] plot[id=qobj2, raw gnuplot] function {
        f(x,y) = x**2 + y**2 - 2*x - 2*y + 1.9;
        set xrange[-1:2];
        set yrange[-2:2];
        set view 0,0;
        set isosamples 1000,1000;
        set size square;
        set cont base;
        set cntrparam levels incre 0,0.1,0;
        unset surface;
        splot f(x,y);
    };
    \draw[scale=1.5,thin, dashed] plot[id=qobj3, raw gnuplot] function {
        f(x,y) = x**2 + y**2 - 2*x - 2*y + 1.98;
        set xrange[-1:2];
        set yrange[-2:2];
        set view 0,0;
        set isosamples 1000,1000;
        set size square;
        set cont base;
        set cntrparam levels incre 0,0.1,0;
        unset surface;
        splot f(x,y);
    };
\end{tikzpicture}

\end{center}
\end{frame}



\begin{frame}{Linjesøk}
\begin{center}
\begin{figure}[ht!]
\centering
\begin{tikzpicture}
    \tikzset{
        position label/.style={
            below = 3pt,
            text height = 1.5ex,
            text depth = 1ex
        },
        brace/.style={
            decoration={brace, mirror},
            decorate
        }
    }

    % grid and axes
    \draw[->,name path=xaxis] (-0.2,0) -- (2.2,0) node[right] {$x$};
    \draw[->,name path=yaxis] (0,-0.2) -- (0,2.2) node[above] {$y$};

    % draw lines
    \draw[name path=line1, domain=0.6:2.5] plot(\x,{-\x + 3});
    \draw[name path=line2, domain=-0.2:1.6] plot(\x,{0.33*\x + 1.33});
    \draw[name path=line3, domain=0.7:2.3] plot(\x,{\x - 1});
    \draw[name path=linesearch, domain=-0.1:2.1] plot(\x,{0.5*\x});

    % calculate intersection points
    \node[name intersections={of=line1 and line2}] (a) at (intersection-1) {};
    \node[name intersections={of=line1 and line3}] (b) at (intersection-1) {};
    \node[name intersections={of=xaxis and line3}] (c) at (intersection-1) {};
    \node[name intersections={of=xaxis and yaxis}] (d) at (intersection-1) {};
    \node[name intersections={of=line2 and yaxis}] (e) at (intersection-1) {};

    \node at (1.2,0.6) (x1) {};
    \node at (1.95, 0.35) (x1desc) {$x_1$};
    \draw[->] (x1desc) .. controls ([yshift=0.5cm] x1desc) and ([xshift=0.5cm]
                                                                   x1) .. (x1);


    \node[red] at (0.7, 0.1) (alphadesc) {$\alpha$};
    \node at (1.2, 0.6) (alpha) {};
    \node at (1, 1) (qopt) {};
    \node at (2, 1) (lopt) {};
    \node at (-0.5, 0.75) (qoptdesc) {$x^*$};
    \node at (2.5, 1.75) (loptdesc) {$\hat{x}_0$};
    \node at (-1, 0.25) (x0desc) {$x_0$};
    \draw[->] (qoptdesc) .. controls ([xshift=1cm] qoptdesc) and
                                                ([xshift=-1cm] qopt) .. (qopt);
    \draw[->] (loptdesc) .. controls ([xshift=-1cm] loptdesc) and
                                                 ([yshift=1cm] lopt) .. (lopt);
    \draw[->] (x0desc) .. controls ([xshift=1cm] x0desc) and
                                                      ([xshift=-1cm] d) .. (d);

    \draw[fill] (qopt) circle [radius=0.02];
    \draw[fill] (alpha) circle [radius=0.03];

    \draw [red,brace] (d) -- (alpha);

    % draw the big polygon
    \fill[very thick,fill=blue,fill opacity=0.3] (a.center) -- (b.center)
                                              -- (c.center) -- (d.center)
                                              -- (e.center) -- cycle;

    % draw the quadratic objective function
    \draw[thin, dashed] plot[id=qobjalpha, raw gnuplot] function {
        f(x,y) = x**2 + y**2 - 2*x - 2*y + 1.8;
        set xrange[-1:2];
        set yrange[-2:2];
        set view 0,0;
        set isosamples 1000,1000;
        set size square;
        set cont base;
        set cntrparam levels incre 0,0.1,0;
        unset surface;
        splot f(x,y);
    };
%    \draw[thin, dashed] plot[id=qobj2, raw gnuplot] function {
%        f(x,y) = x**2 + y**2 - 2*x - 2*y + 1.9;
%        set xrange[-1:2];
%        set yrange[-2:2];
%        set view 0,0;
%        set isosamples 1000,1000;
%        set size square;
%        set cont base;
%        set cntrparam levels incre 0,0.1,0;
%        unset surface;
%        splot f(x,y);
%    };
%    \draw[thin, dashed] plot[id=qobj3, raw gnuplot] function {
%        f(x,y) = x**2 + y**2 - 2*x - 2*y + 1.98;
%        set xrange[-1:2];
%        set yrange[-2:2];
%        set view 0,0;
%        set isosamples 1000,1000;
%        set size square;
%        set cont base;
%        set cntrparam levels incre 0,0.1,0;
%        unset surface;
%        splot f(x,y);
%    };
\end{tikzpicture}
\caption{Line search}
\label{fig:steplength}
\end{figure}

\end{center}
\begin{itemize}
\item $\alpha = 0.6$
\item $x_1 = 0.4x_0 + 0.6 \hat{x}_0 = (1.2, 0.6)$
\item $T_1 = 0.4x - 0.8y - 1.8$
\end{itemize}
\end{frame}




\begin{frame}{$\mathcal{L}_1$}
\begin{center}
\input{include/slpthird}
\end{center}
\begin{itemize}
\item $\alpha = 0.27$
\item $x_2 = (0.88, 0.8)$
\item $T_2 = -0.25x - 0.4y - 0.4$
\end{itemize}
\end{frame}



\begin{frame}{$\mathcal{L}_2$}
\begin{center}
\input{include/slpfourth}
\end{center}
\begin{itemize}
\item $\alpha = 0.23$
\item $x_3 = (0.96, 1.02)$
\end{itemize}
\end{frame}



\begin{frame}{Sti}
\begin{center}
\begin{tikzpicture}
    % grid and axes
    \draw[scale=1.5,->,name path=xaxis] (-0.2,0) -- (2.2,0) node[right] {$x$};
    \draw[scale=1.5,->,name path=yaxis] (0,-0.2) -- (0,2.2) node[above] {$y$};

    % draw lines
    \draw[scale=1.5,name path=line1, domain=0.6:2.5] plot(\x,{-\x + 3})
                                                    node[above right=0.2cm] {};
    \draw[scale=1.5,name path=line2, domain=-0.2:1.6] plot(\x,{0.33*\x + 1.33})
                                                    node[above right=0.2cm] {};
    \draw[scale=1.5,name path=line3, domain=0.7:2.3] plot(\x,{\x - 1})
                                                    node[above right=0.2cm] {};

    % calculate intersection points
    \node[name intersections={of=line1 and line2}] (a) at (intersection-1) {};
    \node[name intersections={of=line1 and line3}] (b) at (intersection-1) {};
    \node[name intersections={of=xaxis and line3}] (c) at (intersection-1) {};
    \node[name intersections={of=xaxis and yaxis}] (d) at (intersection-1) {};
    \node[name intersections={of=line2 and yaxis}] (e) at (intersection-1) {};

    % draw the big polygon
    \fill[very thick,fill=tleblue] (a.center) -- (b.center) --
                                                 (c.center) -- (d.center) --
                                                 (e.center) -- cycle;

    \node at (1*1.5, 1*1.5) (qopt) {};
    \node at (0, 0) (x0) {};
    \node at (1.2*1.5,0.6*1.5) (x1) {};
    \node at (0.88*1.5,0.8*1.5) (x2) {};
    \node at (0.96*1.5,1.02*1.5) (x3) {};

    \draw[scale=1.5,fill] (x1) circle [radius=0.02];
    \draw[scale=1.5,fill] (x2) circle [radius=0.02];
    \draw[scale=1.5,fill] (x3) circle [radius=0.02];
    \draw[scale=1.5,fill] (qopt) circle [radius=0.02];

    \draw[scale=1.5,-,thin] (x0.center) -- (x1.center) -- (x2.center) --
                            (x3.center) -- (qopt.center);

    % draw the quadratic objective function
    \draw[scale=1.5,thin, dashed] plot[id=qobjpath1, raw gnuplot] function {
        f(x,y) = x**2 + y**2 - 2*x - 2*y + 1.8;
        set xrange[-1:2];
        set yrange[-2:2];
        set view 0,0;
        set isosamples 1000,1000;
        set size square;
        set cont base;
        set cntrparam levels incre 0,0.1,0;
        unset surface;
        splot f(x,y);
    };
    \draw[scale=1.5,thin, dashed] plot[id=qobjpath2, raw gnuplot] function {
        f(x,y) = x**2 + y**2 - 2*x - 2*y + 1.9456;
        set xrange[-1:2];
        set yrange[-2:2];
        set view 0,0;
        set isosamples 1000,1000;
        set size square;
        set cont base;
        set cntrparam levels incre 0,0.1,0;
        unset surface;
        splot f(x,y);
    };
    \draw[scale=1.5,thin, dashed] plot[id=qobjpath3, raw gnuplot] function {
        f(x,y) = x**2 + y**2 - 2*x - 2*y + 1.998;
        set xrange[-1:2];
        set yrange[-2:2];
        set view 0,0;
        set isosamples 1000,1000;
        set size square;
        set cont base;
        set cntrparam levels incre 0,0.1,0;
        unset surface;
        splot f(x,y);
    };
\end{tikzpicture}

\end{center}
\end{frame}



\begin{frame}{Endrer et sidekrav}
\begin{center}
\begin{figure}[htbp]
\scalebox{1}{
\begin{minipage}{0.5\linewidth}
\centering
\begin{tikzpicture}
    % grid and axes
    \draw[->,name path=xaxis] (-0.2,0) -- (2.2,0) node[right] {$x$};
    \draw[->,name path=yaxis] (0,-0.2) -- (0,2.2) node[above] {$y$};

    % draw lines
    \draw[name path=line2, domain=-0.1:1.7] plot(\x,{0.33*\x})
                                                    node[above right=0.2cm] {};
    \draw[name path=line3, domain=0.9:1.7] plot(\x,{\x - 1})
                                                    node[above right=0.2cm] {};

    % calculate intersection points
    \node[name intersections={of=xaxis and yaxis}] (a) at (intersection-1) {};
    \node[name intersections={of=line2 and line3}] (b) at (intersection-1) {};
    \node[name intersections={of=xaxis and line3}] (c) at (intersection-1) {};

    % write the coordinates of the corners.
    \path let \p0 = (a) in node [left=-0.0cm of a] {\printpoint{\x0}{\y0}};
    \path let \p0 = (b) in node [right=0.15cm of b] {(1.5,0.5)};
    \path let \p0 = (c) in node [below=-0.2cm of c] {\printpoint{\x0}{\y0}};

    \node at (1, 1) (xu) {};
    \node at (0, 0) (xhat) {};
    \node at (1.2, 0.4) (xstar) {};
    \node at (1.5,0.5) (x0) {};
    \node at (1.95, 1.3) (xudesc) {$x^u$};
    \node at (0.4, 0.4) (xstardesc) {$x^*$};
    \node[red] at (-0.4, -1.0) (xhatdesc) {$\hat{x}_0$};
    \node at (1.8, -0.3) (x0desc) {$x_0$};
    \draw[->] (xudesc) .. controls ([xshift=-1cm] xudesc)
                                        and ([xshift=1cm] xu) .. (xu);
    \draw[->] (x0desc) .. controls ([yshift=0.1cm] x0desc)
                                        and ([xshift=0.5cm] x0) .. (x0);
%    \draw[->] (loptdesc) .. controls ([xshift=0.5cm] loptdesc)
%                                        and ([xshift=-0.5cm] lopt) .. (lopt);
    \draw[->] (xhatdesc) .. controls ([yshift=1cm] xhatdesc)
                                        and ([yshift=-1cm] xhat) .. (xhat);
    \draw[->] (xstardesc) .. controls ([xshift=1cm] xstardesc)
                                        and ([xshift=-1cm] xstar) .. (xstar);
    
    \draw[fill] (xu) circle [radius=0.02];

    \draw[fill] (xstar) circle [radius=0.02];

    % draw the big polygin
%    \fill[very thick,fill=blue,fill opacity=0.3] (a.center) -- (b.center)
%                                              -- (c.center) -- (d.center)
%                                              -- (e.center) -- cycle;

    \fill[very thick,fill=blue,fill opacity=0.3] (a.center) -- (b.center)
                                              -- (c.center) -- cycle;

%    \draw[thick, red, name path=lobj, domain=0.48:1.28] plot(\x, {-0.625*\x + 1.35})
%                                                                            {};

    % draw the quadratic objective function
    \draw[thin, dashed] plot[id=qobj1, raw gnuplot] function {
        f(x,y) = x**2 + y**2 - 2*x - 2*y + 1.75;
        set xrange[-1:2];
        set yrange[-2:2];
        set view 0,0;
        set isosamples 1000,1000;
        set size square;
        set cont base;
        set cntrparam levels incre 0,0.1,0;
        unset surface;
        splot f(x,y);
    };
    \draw[thin, dashed] plot[id=qobj2, raw gnuplot] function {
        f(x,y) = x**2 + y**2 - 2*x - 2*y + 1.9;
        set xrange[-1:2];
        set yrange[-2:2];
        set view 0,0;
        set isosamples 1000,1000;
        set size square;
        set cont base;
        set cntrparam levels incre 0,0.1,0;
        unset surface;
        splot f(x,y);
    };
    \draw[thin, dashed] plot[id=qobj3, raw gnuplot] function {
        f(x,y) = x**2 + y**2 - 2*x - 2*y + 1.98;
        set xrange[-1:2];
        set yrange[-2:2];
        set view 0,0;
        set isosamples 1000,1000;
        set size square;
        set cont base;
        set cntrparam levels incre 0,0.1,0;
        unset surface;
        splot f(x,y);
    };
\end{tikzpicture}
\end{minipage}
}
\scalebox{0.9}{
\begin{minipage}{0.5\linewidth}
\centering
\[
\begin{array}{lcrcrl}
    \textrm{Maximize}   &-&   x &+&   y \\
    \textrm{subject to} & &   x &+&   y & \leq 3 \\
    \textrm{and}        & &   x &-&   y & \leq 1 \\
    \textrm{and}        &-&   x &+& 3 y & \leq 0 \\
    \textrm{and}        & &   x &,&   y & \geq 0
\end{array}
%    \begin{array}{rcrcrcr}
 %       \zeta &=& 1.01 &-& 0.10 w_3 &-& 0.10 w_1 \\ \hline
 %           y &=& 0.75 &-& 0.25 w_3 &-& 0.25 w_1 \\
 %           x &=& 2.25 &+& 0.25 w_3 &-& 0.75 w_1 \\
 %         w_2 &=&-0.50 &-& 0.50 w_3 &+& 0.50 w_1
 %   \end{array}
\]
\end{minipage}
}
\caption{A linearization of the new QP}
\label{fig:prac1}
\end{figure}

\end{center}
\begin{itemize}
\item $-x + 3y \leq 4$
\item $-x + 3y \leq 0$
\item $\alpha = 0.2$
\item $x_1 = 0.8x_0 + 0.2 \hat{x}_0 = (1.2,0.4) = x^*$
\end{itemize}
\end{frame}



\begin{frame}{Like optimale løsninger}
\begin{itemize}
\item En mengde $\mathcal{M}_k$ med grener som faller ut.
\item En subinstans $\mathcal{Q}_k$ definert av $\mathcal{Q}$ og $\mathcal{M}_k$
\item Optimal løsning til $\mathcal{Q}_k$ noteres som $x_k^*$
\item En mengde med variabler som er $0$ i $x_k^*$ noteres som $\mathcal{Z}_k$.
\item $2^n - 1$ subinstanser. $\left| \mathcal{Z}_0 \right| = 1749$ i \textit{vlarge}.
\item $2^{1749} \approx 3 \times 10^{526}$ subinstanser har løsning $x_0^*$
\end{itemize}
\end{frame}



\begin{frame}{Søker etter $\mathcal{M}_{14} = \left\{  2,3,4 \right\}$}
\begin{center}
\input{include/maptreesets2}
\[
\small
    \begin{array}{ll}
        \mathcal{M}_0    = \left\{{}\right\}           & \mathcal{Z}_0    = \left\{{1,3}\right\} \\
        \mathcal{M}_2    = \left\{{2}\right\}          & \mathcal{Z}_2    = \left\{{2,3,5}\right\} \\
        \mathcal{M}_3    = \left\{{1,2}\right\}        & \mathcal{Z}_3    = \left\{{1,2,4,5}\right\} \\
        \mathcal{M}_7    = \left\{{1,2,3}\right\}      & \mathcal{Z}_7    = \left\{{1,2,3,5}\right\} \\
        \mathcal{M}_8    = \left\{{4}\right\}          & \mathcal{Z}_8    = \left\{{1,4,5}\right\} \\
        \mathcal{M}_{10} = \left\{{2,4}\right\}        & \mathcal{Z}_{10} = \left\{{2,3,4,5}\right\}
    \end{array}
~
    \begin{array}{ll}
        \mathcal{M}_{12} = \left\{{3,4}\right\}        & \mathcal{Z}_{12} = \left\{{4,3,1}\right\} \\
        \mathcal{M}_{15} = \left\{{1,2,3,4}\right\}    & \mathcal{Z}_{15} = \left\{{1,2,3,4}\right\} \\
        \mathcal{M}_{16} = \left\{{5}\right\}          & \mathcal{Z}_{16} = \left\{{1,3,5}\right\} \\
        \mathcal{M}_{28} = \left\{{3,4,5}\right\}      & \mathcal{Z}_{28} = \left\{{2,3,4,5}\right\} \\
        \mathcal{M}_{29} = \left\{{1,3,4,5}\right\}    & \mathcal{Z}_{29} = \left\{{1,2,3,4,5}\right\}
    \end{array}
\]
\end{center}
\end{frame}



\begin{frame}{Søker etter $\mathcal{M}_{14} = \left\{  2,3,4 \right\}$}
\begin{center}
\begin{tikzpicture}[level/.style={sibling distance=10mm/#1, level distance=10mm}]
\tikzset{every node/.style={shape=circle,fill=black!25,minimum size=7mm}}
%\tikzset{every node/.style={shape=circle,
%                            font=\bfseries \Large,
%                            minimum size=3cm,
%                            scale=0.4
%                           }}
\node[fill=red!40] (root) {$0$}
    child {
        node[fill=red!40] {$1$}
        child {
            node[fill=red!40] {$5$}
            child {
                node[fill=red!40] {$8$
                };
                \path edge from parent[draw,line width=3pt,-,red!30];
            };
        \path edge from parent[draw,line width=3pt,-,red!30];
        };
    \path edge from parent[draw,line width=3pt,-,red!30];
    }
    child {
        node {$2$}
        child {
            node {$6$}
            child {
                node {$9$}
                child {node {$10$}
                }
            }
        };
    \path edge from parent[draw,line width=3pt,-,red!30];
    }
    child {node {$3$}}
    child {
        node {$4$}
        child {
            node {$7$}
        };
    \path edge from parent[draw,line width=3pt,-,red!30];
    }
    ;
\end{tikzpicture}

\[
\small
    \begin{array}{ll}
        \mathcal{M}_0    = \left\{{}\right\}           & \mathcal{Z}_0    = \left\{{1,3}\right\} \\
        \mathcal{M}_2    = \left\{{2}\right\}          & \mathcal{Z}_2    = \left\{{2,3,5}\right\} \\
        \mathcal{M}_3    = \left\{{1,2}\right\}        & \mathcal{Z}_3    = \left\{{1,2,4,5}\right\} \\
        \mathcal{M}_7    = \left\{{1,2,3}\right\}      & \mathcal{Z}_7    = \left\{{1,2,3,5}\right\} \\
        \mathcal{M}_8    = \left\{{4}\right\}          & \mathcal{Z}_8    = \left\{{1,4,5}\right\} \\
        \mathcal{M}_{10} = \left\{{2,4}\right\}        & \mathcal{Z}_{10} = \left\{{2,3,4,5}\right\}
    \end{array}
~
    \begin{array}{ll}
        \mathcal{M}_{12} = \left\{{3,4}\right\}        & \mathcal{Z}_{12} = \left\{{4,3,1}\right\} \\
        \mathcal{M}_{15} = \left\{{1,2,3,4}\right\}    & \mathcal{Z}_{15} = \left\{{1,2,3,4}\right\} \\
        \mathcal{M}_{16} = \left\{{5}\right\}          & \mathcal{Z}_{16} = \left\{{1,3,5}\right\} \\
        \mathcal{M}_{28} = \left\{{3,4,5}\right\}      & \mathcal{Z}_{28} = \left\{{2,3,4,5}\right\} \\
        \mathcal{M}_{29} = \left\{{1,3,4,5}\right\}    & \mathcal{Z}_{29} = \left\{{1,2,3,4,5}\right\}
    \end{array}
\]
\end{center}
\end{frame}



\begin{frame}{Søker etter $\mathcal{M}_{14} = \left\{  2,3,4 \right\}$}
\begin{center}
\begin{tikzpicture}[level/.style={sibling distance=10mm/#1, level distance=10mm}]
\tikzset{every node/.style={shape=circle,fill=black!25,minimum size=7mm}}
%\tikzset{every node/.style={shape=circle,
%                            font=\bfseries \Large,
%                            minimum size=3cm,
%                            scale=0.4
%                           }}
\node[fill=red!40] (root) {$0$}
    child {
        node {$2$}
        child {
            node {$3$}
            child {
                node {$7$
                }
            }
        }
    }
    child[fill=red!40] {
        node[fill=red!40] {$8$}
        child {
            node[fill=red!40] {$12$}
            child {
                node {$28$}
                child {node {$29$}
                };
                \path edge from parent[draw,line width=3pt,-,red!30];
            };
            \path edge from parent[draw,line width=3pt,-,red!30];
        };
    \path edge from parent[draw,line width=3pt,-,red!30];
    }
    child {node {$16$}}
    child {
        node {$10$}
        child {
            node {$11$}
        };
    \path edge from parent[draw,line width=3pt,-,red!30];
    }
    ;
\end{tikzpicture}

\[
\small
    \begin{array}{ll}
        \mathcal{M}_0    = \left\{{}\right\}           & \mathcal{Z}_0    = \left\{{1,3}\right\} \\
        \mathcal{M}_2    = \left\{{2}\right\}          & \mathcal{Z}_2    = \left\{{2,3,5}\right\} \\
        \mathcal{M}_3    = \left\{{1,2}\right\}        & \mathcal{Z}_3    = \left\{{1,2,4,5}\right\} \\
        \mathcal{M}_7    = \left\{{1,2,3}\right\}      & \mathcal{Z}_7    = \left\{{1,2,3,5}\right\} \\
        \mathcal{M}_8    = \left\{{4}\right\}          & \mathcal{Z}_8    = \left\{{1,4,5}\right\} \\
        \mathcal{M}_{10} = \left\{{2,4}\right\}        & \mathcal{Z}_{10} = \left\{{2,3,4,5}\right\}
    \end{array}
~
    \begin{array}{ll}
        \mathcal{M}_{12} = \left\{{3,4}\right\}        & \mathcal{Z}_{12} = \left\{{4,3,1}\right\} \\
        \mathcal{M}_{15} = \left\{{1,2,3,4}\right\}    & \mathcal{Z}_{15} = \left\{{1,2,3,4}\right\} \\
        \mathcal{M}_{16} = \left\{{5}\right\}          & \mathcal{Z}_{16} = \left\{{1,3,5}\right\} \\
        \mathcal{M}_{28} = \left\{{3,4,5}\right\}      & \mathcal{Z}_{28} = \left\{{2,3,4,5}\right\} \\
        \mathcal{M}_{29} = \left\{{1,3,4,5}\right\}    & \mathcal{Z}_{29} = \left\{{1,2,3,4,5}\right\}
    \end{array}
\]
\end{center}
\end{frame}



\begin{frame}{Søker etter $\mathcal{M}_{14} = \left\{  2,3,4 \right\}$}
\begin{center}
\begin{tikzpicture}
[level 1/.style={sibling distance=16mm, level distance=12mm},
level 2/.style={sibling distance=10mm, level distance=12mm}
]
\tikzset{every node/.style={shape=circle,draw,fill=black!5,minimum size=9mm}}
%\tikzset{every node/.style={shape=circle,
%                            font=\bfseries \Large,
%                            minimum size=3cm,
%                            scale=0.4
%                           }}
\node[fill=red!40] (root) {$0$}
    child {
        node {$2$}
        child {
            node {$3$}
            child {
                node {$7$}
            }
        }
    }
    child {
        node[fill=red!40] {$8$}
        child {
            node[fill=red!40] {$10$};
            \path edge from parent[draw,line width=3pt,-,red!40];
        }
        child {
            node {$12$}
            child {
                node {$15$}
            };
            \path edge from parent[draw,line width=3pt,-,red!40];
        };
    \path edge from parent[draw,line width=3pt,-,red!40];
    }
    child {
        node {$16$}
        child {
            node {$28$}
            child {
                node {$29$}
            }
        }
    }
    ;
\end{tikzpicture}

\[
\small
    \begin{array}{ll}
        \mathcal{M}_0    = \left\{{}\right\}           & \mathcal{Z}_0    = \left\{{1,3}\right\} \\
        \mathcal{M}_2    = \left\{{2}\right\}          & \mathcal{Z}_2    = \left\{{2,3,5}\right\} \\
        \mathcal{M}_3    = \left\{{1,2}\right\}        & \mathcal{Z}_3    = \left\{{1,2,4,5}\right\} \\
        \mathcal{M}_7    = \left\{{1,2,3}\right\}      & \mathcal{Z}_7    = \left\{{1,2,3,5}\right\} \\
        \mathcal{M}_8    = \left\{{4}\right\}          & \mathcal{Z}_8    = \left\{{1,4,5}\right\} \\
        \mathcal{M}_{10} = \left\{{2,4}\right\}        & \mathcal{Z}_{10} = \left\{{2,3,4,5}\right\}
    \end{array}
~
    \begin{array}{ll}
        \mathcal{M}_{12} = \left\{{3,4}\right\}        & \mathcal{Z}_{12} = \left\{{4,3,1}\right\} \\
        \mathcal{M}_{15} = \left\{{1,2,3,4}\right\}    & \mathcal{Z}_{15} = \left\{{1,2,3,4}\right\} \\
        \mathcal{M}_{16} = \left\{{5}\right\}          & \mathcal{Z}_{16} = \left\{{1,3,5}\right\} \\
        \mathcal{M}_{28} = \left\{{3,4,5}\right\}      & \mathcal{Z}_{28} = \left\{{2,3,4,5}\right\} \\
        \mathcal{M}_{29} = \left\{{1,3,4,5}\right\}    & \mathcal{Z}_{29} = \left\{{1,2,3,4,5}\right\}
    \end{array}
\]
\end{center}
\end{frame}



\begin{frame}{Algoritme: \texttt{find}}
\begin{algorithm}[H]
\caption{\texttt{find($\mathcal{M}_l$, $v_k$)}}
\label{alg:find}
\begin{algorithm}[h!]
\caption{\texttt{find($\mathcal{M}_l$, $v_k$)}}
\label{alg:find}
%\KwIn{($\mathcal{M}_l$, $\mathcal{Q}_k$)}
\SetAlgoVlined
\ForEach{child vertex $v_i$ of $v_k$}{
  \If{$\mathcal{M}_i \subseteq \mathcal{M}_l$}{
    \eIf{$\mathcal{M}_l \subseteq \mathcal{Z}_i$}{
      \textbf{return} $i$
%      \KwRet{$i$}
    }{
      \texttt{find($\mathcal{M}_l$, $v_i$)}
%      \KwRet{find($\mathcal{M}_l$, $v_i$)}
    }
  }
}
\KwRet{$-1$}
\end{algorithm}

\end{algorithm}
\end{frame}


\begin{frame}{Eksperiment 1}
%\begin{itemize}
%\item Vi løser \textit{small}, med $\sigma(2, 238) = 28 442$ subinstanser
%\end{itemize}
Vi løser \textit{small}, med $\sigma(2, 238) = 28 442$ subinstanser
\begin{table}[ht!]
\centering
\caption{Resultater av de forskjellige implementasjonen med endrende toleranse.}
\begin{tabular}{rrrrr}
$\epsilon$ & \texttt{cClp} & \texttt{cSlp} & \texttt{nClp} & \texttt{nSlp} \\ \hline
$10^{-1}$ & 45.51 & 55.61 & 72.32 & 85.51 \\
$10^{-2}$ & 46.34 & 55.89 & 73.11 & 85.51 \\
$10^{-3}$ & 51.12 & 59.04 & 75.60 & 85.28 \\
$10^{-4}$ & 52.46 & 73.79 & 77.83 & 107.39 \\
$10^{-5}$ & 54.48 & 232.53 & 81.16 & 355.47 \\
$10^{-6}$ & 65.42 & 1363.46 & 93.29 & 2022.25 \\
$10^{-7}$ & 70.78 & 6522.91 & 100.85 & 9395.92
\end{tabular}
\label{table:expone}
\end{table}
\end{frame}



\begin{frame}{Eksperiment 1}
\begin{figure}[ht!]
    \centering
    \begin{tikzpicture}
    \begin{loglogaxis}[
        legend pos=north east]

        \addplot[black] plot coordinates {
(0.1,       45.51)
(0.01,      46.34)
(0.001,     51.12)
(0.0001,    52.46)
(0.00001,   54.48)
(0.000001,  65.42)
(0.0000001, 70.78)
        };

        \addplot[red] plot coordinates {
(0.1,       55.61)
(0.01,      55.89)
(0.001,     59.04)
(0.0001,    73.79)
(0.00001,   232.53)
(0.000001,  1363.46)
(0.0000001, 6522.91)
        };

        \addplot[green] plot coordinates {
(0.1,       72.32)
(0.01,      73.11)
(0.001,     75.60)
(0.0001,    77.83)
(0.00001,   81.16)
(0.000001,  93.29)
(0.0000001, 100.85)
        };

        \addplot[blue] plot coordinates {
(0.1,       85.51)
(0.01,      85.51)
(0.001,     85.28)
(0.0001,    107.39)
(0.00001,   355.47)
(0.000001,  2022.25)
(0.0000001, 9395.92)
        };

        \legend{\texttt{construct\_clp}, \texttt{construct\_slp}, \texttt{naive\_clp}, \texttt{naive\_slp}}
    \end{loglogaxis}
\end{tikzpicture}

    \caption{Kjøretid i CPU-sekunder for å løse \textit{small} og dens subinstanser.}
    \label{fig:smalltolerance}
\end{figure}
\end{frame}



\begin{frame}{Eksperiment 2}
\begin{table}
\centering
\caption{Kjøretid i CPU-sekunder for å løse de tre instansene.}
\label{table:eps4instances}
\begin{tabular}{crrr}
\textrm{Implementasjon} & \textit{small} & \textit{large} & \textit{vlarge} \\ \hline
\texttt{cClp}           & 0.52           & 9.55           & 76.18 \\
\texttt{cSlp}           & 0.71           & 32.88          & 585.60 \\
\texttt{nClp}           & 0.65           & 11.68          & 157.53 \\
\texttt{nSlp}           & 0.89           & 39.87          & 1173.74
\end{tabular}
\end{table}
\end{frame}



\begin{frame}{Tilfeldige instanser}
\begin{itemize}
\item $m = \lfloor \frac{7}{20}n \rfloor$
\item $b_i$ har 50\% sannsynlighet for å være null. Ellers $10 \leq | b_i | \leq 70$.
\item $h_{ii}$ har 50\% sannsynlighet for å være null. Ellers $10^{-5} \leq h_{ii} \leq 10^{-1}$.
\end{itemize}
\end{frame}



\begin{frame}{Eksperiment 3}
\begin{table}[ht!]
    \centering
    \caption{Kjøretid for å løse tilfeldige instanser med økende $n$. $\beta = 1$.}
    \label{table:expfour}
\begin{tabular}{rrrc}
    $n$ & \texttt{cClp}  & \texttt{nClp}  & Relativ Speedup \\ \hline
    500 & 4.9   & 5.9   & 16.9\% \\
   1000 & 42.1  & 53.0  & 20.6\% \\
   1500 & 181.5 & 234.5 & 22.6\% \\
   2000 & 547.1 & 710.2 & 23.0\%
\end{tabular}
\end{table}
\end{frame}



\begin{frame}{Eksperiment 4}
\begin{table}[ht!]
\centering
\caption{Kjøretid i CPU-sekunder for $n = 50$ og økende $\beta$.}
\begin{tabular}{rrrcr}
      $\beta$ & \texttt{cClp} & \texttt{nClp} & Relativ Speedup & Distinkte løsninger\\ \hline
       1  & 0.03 & 0.04 & 25.0\% & 37.4 ($74.3$\%) \\
       2  & 0.64 & 0.94 & 31.9\% & 744.3 ($58.3$\%) \\
       3  & 7.06 & 15.90 & 55.6\% & 9484.7 ($45.4$\%) \\
       4  & 77.59 & 188.83 & 58.9\% & 82262.5 ($32.8$\%) \\
       5  & 586.54 & 1758.23 & 66.6\% & 574685.0 ($24.2$\%) \\
\end{tabular}
\label{table:exptwo}
\end{table}
\end{frame}



\begin{frame}{Eksperiment 4}
\begin{tikzpicture}
    \begin{axis}[
        xlabel=$\beta$,
        ylabel=CPU-seconds,
        legend pos=south west]

        \addplot[black] plot coordinates {
(1, 0.0006)
(2, 0.000501961)
(3, 0.000338204)
(4, 0.000308908)
(5, 0.000247492)
        };

        \addplot[red] plot coordinates {
(1, 0.0008)
(2, 0.000737255)
(3, 0.000761677)
(4, 0.000751787)
(5, 0.00074189)
        };

        \legend{\texttt{cClp}, \texttt{nClp}}
    \end{axis}
\end{tikzpicture}

\end{frame}



\end{document}
