\documentclass{beamer}
\usepackage[utf8]{inputenc}
\usepackage{tikz}
\usetikzlibrary{intersections,positioning,calc,arrows,shapes,
                decorations.pathreplacing,spy,automata}
\usepackage{pgfplots}
\usepackage[ruled]{algorithm2e}
\usetheme{boxes}
\title[Fast Solver of Closely Related Quadratic Programming Problems]
      {Fast Solver of Closely Related Quadratic Programming Problems}
\author{Andreas Halle}
\institute{Department of Informatics}
\date{June 10, 2013}


\DeclareMathOperator*{\argmin}{arg\,min}


\begin{document}



\begin{frame}
\titlepage
\end{frame}



\section{Introduksjon}



\begin{frame}{PROMAPS}
\begin{itemize}
\item Utviklet av Goodtech og MathConsult.
\item Kalkurerer leveransepåliteligheten i et nettverk ved å kalkulere risiko
      ved utfall av grener i nettverket.
\item Dette er formulert som en rekke veldig like QP-problemer
      (Quadratic Programming).
\item QP-løseren en flaskehals. Skjermbildet (PROMAPS) oppdaterer seg hvert
      femte minutt.
\item \url{http://www.tu.no/energi/2011/10/07/her-beregnes-risikoen-for-svikt-i-kraftnettet}
\end{itemize}
\end{frame}



%\begin{frame}{QP-problemet}
%\begin{itemize}
%\item Objektfunksjonen representerer kostnader for strømleveranse.
%\item Sidekravene representerer nettverket, hvor summen av flyten i en
%      node er lik $0$.
%\item Mer spesifikt...
%\end{itemize}
%\end{frame}



\begin{frame}{Objektfunksjonen}
\[
f(x) = x^T \Phi D x + (g - c)^T x
\]
\begin{itemize}
\item $f(x)$ representerer leveransekostnader ($E/s$)
\item $x$ representerer strømmen over grenene ($W$)
\item $\Phi$ representerer strømtap ($1/W$)
\item $D$ representerer overføringskostnader ($E/J$)
\item $g$ representerer kostnader for å generere strøm ($E/J$)
\item $c$ representerer leveransepris ($E/J$)
\end{itemize}
\end{frame}



\begin{frame}{Objektfunksjonen}
\[
f(x) = x^T H x + b^T x
\]
\begin{itemize}
\item $H = \Phi D$ representerer kostnader ($E/(W^2 s)$)
\item $b = g - c$ representerer kostnader ($E/J$)
\item $H$ er en matrise på størrelsen $n \times n$
\item $b$ og $x$ er vektorer i $\mathbb{R}^n$
\end{itemize}
\end{frame}



\begin{frame}{Optimeringsproblemet}
Vi definerer et konveks QP-problem
\[
\min_{x \in \mathbb{R}^n} f(x) \quad \textrm{subject to} ~ Ax = 0, ~ l \leq x \leq u
\]
\begin{itemize}
\item A er en $m \times n$ insidensmatrise for et strømnettverk
\item $m$ noder
\item $n$ grener
\item $l$ og $u$ er nedre og øvre grenkapasitet ($W$)
\end{itemize}
\end{frame}



\begin{frame}{Utfall}
Vi ønsker å modellere utfall.
\begin{itemize}
\item Setter $l_i = u_i = 0$
\item Goodtech ønsker å løse QP-problemet for ulike utfall
\item Definerer subinstanser for kombinasjoner av grener som faller ut
\item Instanse er et QP-problem uten utfall
\item Subinstanse er en instanse med utfall
\end{itemize}
\end{frame}



\begin{frame}{Utfall}
\begin{itemize}
\item Vil løse så mange subinstanser som mulig.
\item Usannsynlig at det er mange utfall.
\item Vi prøver å løse alle subinstanser som har mindre eller lik $\beta$ utfall.
\end{itemize}
\[
\sigma (\beta, n) = \sum_{j=0}^{\beta} \binom{n}{j}
\]
\end{frame}



\begin{frame}{Subinstanser}
\begin{itemize}
\item En mengde av variabler som representerer utfall $\mathcal{M}_k$
\item $k = \sum_{j \in \mathcal{M}_k}^{} 2^{j-1}$
\item Eks. $\mathcal{M}_k = \left\{1,3,5 \right\}$. $k = 21$
\item En subinstanse $\mathcal{Q}_k$ defineres av $\mathcal{M}_k$
\item Optimal løsning til subinstanse $\mathcal{Q}_k$ noterer vi som $x_k^*$
\end{itemize}
\end{frame}



\begin{frame}{Instanser fra Goodtech}
\input{include/newhistH}
\end{frame}



\begin{frame}{Instanser fra Goodtech}
\begin{table}[ht!]
    \centering
    \caption{Størrelse for hver instanse}
    \begin{tabular}{lrrr}
    Problemstørrelse & \textit{small} & \textit{large} & \textit{vlarge} \\\hline
    Rader        & 82             & 328            & 1127 \\
    Kolonne      & 238            & 952            & 3437 \\
    Ikke-nuller A & 348            & 1392           & 4840 \\
    Ikke-nuller H & 108            & 432            & 894 \\
    \end{tabular}
    \label{table:sizes}
\end{table}
\begin{table}[ht!]
    \centering
    \caption{Verdier i objektfunksjonen}

    \begin{tabular}{lrr}
      & \textit{small} og \textit{large}         & \textit{vlarge} \\\hline
    $\max(h_{ii})$      & $2.9614 \times 10^{-2}$ & $4.9011 \times 10^{-2}$ \\
    $\min(h_{ii})$      & $4.9290 \times 10^{-5}$ & $1.1026 \times 10^{-5}$ \\
$\textrm{mean}(h_{ii})$ & $5.2864 \times 10^{-3}$ & $5.8984 \times 10^{-3}$ \\
    $\max(b_{i})$       & 20                      & 20 \\
    $\min(b_{i})$       & -70                     & -50 \\
    \end{tabular}
    \label{table:maxmin}
\end{table}
\end{frame}



\begin{frame}{Instanser fra Goodtech}
\begin{itemize}
\item Vi ser at det lineære leddet har mye høyere verdier enn det kvadratiske
\item Prøver å lineærisere objektfunksjonen
\item Lineær taylor-utvikling av objektfunksjonen i punkt $a$: \\
      $T_a(x) = -a^T H a + 2a^T H x + b^T x$ \\
      $T_0(x) = b^T x$
\end{itemize}
Definerer et LP $\mathcal{L}$ for hvert QP $\mathcal{Q}$
\[
\min_{x \in \mathbb{R}^n} g(x) \quad \textrm{subject to} ~ Ax = 0, ~ l \leq x \leq u
\]
\begin{itemize}
\item $g(x) = T_0(x) = b^T x$
\end{itemize}
\end{frame}



\begin{frame}{Hvor like er $\mathcal{L}$ og $\mathcal{Q}$?}
\begin{itemize}
\item Optimal løsning til $\mathcal{Q}$ noteres $x^*$
\item Optimal løsning til $\mathcal{L}$ noteres $\hat{x}$
\item Vi noterer avvik mellom $f(\hat{x})$ og $f(x^*)$ for
      $\Delta = \left| \frac{f(\hat{x}) - f(x^*)}{f(x^*)} \right|$
\item Vi genererer tilfeldige instanser og sjekker $\Delta$
\end{itemize}
\end{frame}



\begin{frame}{$100\Delta$}
\begin{figure}[ht!]
\begin{tikzpicture}
    \begin{axis}[
        width=\textwidth*0.80,
%        grid=major,
        view={245}{30},
%        xmax=99,
%        ymax=99,
%        zmax=1,
%        restrict z to domain=0:1,
        xlabel=x,
        ylabel=y,
        zlabel=$100\Delta$]

        \addplot3[surf,mesh/rows=20,mesh/cols=20] table[x=b,y=h,z=dev]
                                                   {include/data/delta3d.dat};


%        \addplot3[
%            surf,
%            mesh/rows=101,
%            mesh/cols=100,
%        ] 
%        table[x=b,y=h,z=dev] {include/data/delta3d.dat};
\end{axis}
\end{tikzpicture}

\caption{Avvik som en funksjon av tettheten i objektfunksjonen. $x$ er prosent
         nuller på diagonalen til $H$. $y$ er prosent nuller i $b$}
\end{figure}
\end{frame}



\begin{frame}{Hvor like er $\mathcal{L}$ og $\mathcal{Q}$?}
Når vi løser $\mathcal{L}$ så har vi allerede oppnådd minst $95\%$ av optimal
målfunksjonsverdi for $\mathcal{Q}$!
\begin{itemize}
\item Metode basert på successive linear programming (SLP)
\item Lar vi startverdien $x_0 = 0$, når vi rundt $95\%$ av optimal målfunksjonsverdi etter
      første iterasjon
\item Taylor-utvikling i $x_k$ noterer vi som $T_k$
\item Definerer $\mathcal{L}_k$ som LP-problemet å minimere $T_k$ underlagt
      sidekravene til $\mathcal{Q}$.
\end{itemize}
\end{frame}



\begin{frame}{Slp}
\begin{algorithm}[ht!]
\caption{\texttt{slp($x_0$, $\epsilon \ge 0$)}}
\label{alg:iter}
\SetAlgoVlined
\DontPrintSemicolon
%Given $x_0$ and some $\epsilon \ge 0$\;
Set $k \leftarrow 0$\;

\Repeat{$\displaystyle \frac{f(x_{k-1}) - f(x_k)}{|f(x_{k-1})|} \leq \epsilon$} {
    $T_k \leftarrow - x_k^THx_k + 2x_k^THx + b^Tx$ \\
    $\hat{x}_k \leftarrow \textrm{optimal solution of } \mathcal{L}_k$ (Solve) \\
    $\alpha_k \in \argmin_{\alpha \leq 1} f((1-\alpha) x_k + \alpha \hat{x}_k)$ \\
%   $\alpha_k \in \argmin_{\alpha \leq 1} f(\alpha x_k + (1-\alpha)\hat{x}_k)$\;
    $x_{k+1} \leftarrow (1-\alpha_k) x_k + \alpha_k\hat{x}_k$ \\
    $k \leftarrow k +1$
}
\end{algorithm}

\end{frame}



\begin{frame}{Et eksempel}
\[
\begin{array}{lcrcrcl}
\textrm{minimize}           & &   (x-1)^2 &+&   (y-1)^2 & - &  2 \\
\textrm{subject to}         & &         x &+&         y &\leq& 3 \\
                            & &         x &-&         y &\leq& 1 \\
                            & &         x &+&       3 y &\leq& 4 \\
                 \multicolumn{5}{r}{x,y}                &\geq& 0,
\end{array}
\]
\end{frame}



\end{document}
