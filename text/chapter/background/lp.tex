%This section acts as an introduction to linear programming problems and
%quadratic programming problems.
%The terminology and notation we use is influenced by\cite{vanderbei}.

In linear programming, we aim to maximize or minimize a linear function of
some variables $x_j$ for $j=1,2,\ldots,n$. We refer to these variables as
\textit{decision variables}. The linear function to be optimized is defined as
a linear combination of the decision variables:
\[
\zeta = c_1 x_1 + c_2 x_2 + \cdots + c_n x_n,
\]
and is called the \textit{objective function}, where $c_j$
are known coefficients.
This function is maximized
or minimized subject to linear \textit{constraints}. These constraints are
either equalities or inequalities in a linear combination
of the decision variables:
\[
a_1x_1 + a_2 x_2 + \cdots + a_n x_n \left\{\begin{array}{c} \leq \\ = \\ \geq \end{array}\right\} b.
\]
where $a_j$ are known coefficients and $b$ is some given value.
The problem of maximizing or minimizing $\zeta$ subject to one or more
linear constraints is called a linear programming problem.
We can formulate such problems as:
\[
\begin{array}{lrcrcccrcr}
\textrm{minimize}   & c_1 x_1    &+& c_2 x_2    &+& \cdots &+& c_n x_n               \\
\textrm{subject to} & a_{11} x_1 &+& a_{12} x_2 &+& \cdots &+& a_{1n} x_n &\leq& b_1 \\
                    & a_{21} x_1 &+& a_{22} x_2 &+& \cdots &+& a_{2n} x_n &\leq& b_2 \\
                    &            & &            & & \vdots & &            &    &     \\
                    & a_{m1} x_1 &+& a_{m2} x_2 &+& \cdots &+& a_{mn} x_n &\leq& b_m \\
                    \multicolumn{8}{r}{x_1,x_2,\ldots,x_n} &\geq& 0,
\end{array}
\]
where $m$ is the number of constraints and $n$ is the number of decision
variables.
This formulation is referred to as \textit{standard form}\cite{vanderbei}.
We can also formulate LP problems in a more compact form:
\begin{equation}
\min{c^T x},\quad \textrm{subject to}~Ax \leq b, ~ x \geq 0, \label{eq:primal}
\end{equation}
where $c$ and $x$ are vectors in $\mathbb{R}^n$, $b$ is a vector in
$\mathbb{R}^m$, and A is an $m \times n$ matrix.

Constraints are not necessarily written as
\textit{less-than} inequalities, so we have to convert them to get the problem
in standard form. An inequality
\[
a_1 x_1 + a_2 x_2 + \cdots + a_n x_n \geq b
\]
might be converted to a \textit{less-than} inequality by multiplying both
sides by $-1$:
\[
- a_1 x_1 - a_2 x_2 - \cdots - a_n x_n \leq -b
\]
An equality constraint can be converted into two inequality constraints.
Given the equality constraint
\[
a_1 x_1 + a_2 x_2 + \cdots + a_n x_n = b,
\]
we can convert it into the two inequalities
\[
\begin{array}{c}
a_1 x_1 + a_2 x_2 + \cdots + a_n x_n \leq b \\
a_1 x_1 + a_2 x_2 + \cdots + a_n x_n \geq b,
\end{array}
\]
or equivalently
\[
\begin{array}{c}
a_1 x_1 + a_2 x_2 + \cdots + a_n x_n \leq b \\
- a_1 x_1 - a_2 x_2 - \cdots - a_n x_n \leq -b.
\end{array}
\]

Any combination of values assigned to the decision variables is called a
\textit{solution}.
If the values satisfy the constraints of the problem, the solution is
said to be \textit{feasible}. In contrast, if the solution
does not satisfy all the constraints of the problem, the solution is said to
be \textit{infeasible}. The set of all feasible solutions is called the
\textit{feasible region}\footnote{Also referred to as feasible set,
search space or solution space.}\cite{nocedal}. A feasible solution that
attains the minimum (or maximum if we are maximizing) objective value among all
the feasible points is called an \textit{optimal solution}.

A convex polytope may be defined as an intersection of half-spaces, and a
system of linear inequalities may be regarded as an intersection of
half-spaces\cite{branko}.
Hence, the feasible region may be regarded as a convex polytope.
Note that the feasible region might be empty, i.e. there are no feasible
points.
In that case, the linear programming problem itself is considered infeasible.
There is also the concept of \textit{unbounded} problems, where we can attain
an arbitrarily large objective value while still remaining feasible.

Consider the following problem:
\[
\begin{array}{lcrcrl}
    \textrm{minimize}   &-& 2 x_1 &-&   x_2 \\
    \textrm{subject to} & &   x_1 &+&   x_2 & \leq 3 \\
                        & &   x_1 &-&   x_2 & \leq 1 \\
                        &-&   x_1 &+& 3 x_2 & \leq 4 \\
     \multicolumn{5}{r}{x_1,x_2}            & \geq 0.
\end{array}
\]
Since the problem is in only two variables, we can illustrate the feasible
region in two dimensions.
Figure \ref{fig:lpback} shows an illustration of the feasible
region of the LP problem above. It also shows the level sets of the objective
function from left to right when the function takes the values $0$, $-2.5$ and
$-5$, respectively.
The rightmost gray line represents the objective function when it takes
its minimum value. Our optimal solution in this example is $(2,1)$, and the
objective value is $-5$.

\begin{figure}[ht!]
\centering
\begin{tikzpicture}
    % grid and axes
    \draw[scale=1.5,->,name path=xaxis] (-0.2,0) -- (2.2,0) node[right] {$x_1$};
    \draw[scale=1.5,->,name path=yaxis] (0,-0.2) -- (0,2.2) node[above] {$x_2$};

    % draw lines
    \draw[scale=1.5,name path=line1, domain=0.6:2.5] plot(\x,{-\x + 3})
                                                    node[above right=0.2cm] {};
    \draw[scale=1.5,name path=line2, domain=-0.2:1.6] plot(\x,{0.33*\x + 1.33})
                                                    node[above right=0.2cm] {};
    \draw[scale=1.5,name path=line3, domain=0.7:2.3] plot(\x,{\x - 1})
                                                    node[above right=0.2cm] {};

    % calculate intersection points
    \node[name intersections={of=line1 and line2}] (a) at (intersection-1) {};
    \node[name intersections={of=line1 and line3}] (b) at (intersection-1) {};
    \node[name intersections={of=xaxis and line3}] (c) at (intersection-1) {};
    \node[name intersections={of=xaxis and yaxis}] (d) at (intersection-1) {};
    \node[name intersections={of=line2 and yaxis}] (e) at (intersection-1) {};

    % write the coordinates of the corners.
    \path let \p0 = (a) in node [above right=0.0cm of a] {($1.25$, $1.75$)};
    \path let \p0 = (b) in node [right=0.1cm of b] {($2$, $1$)};
    \path let \p0 = (c) in node [below right=-0.35cm of c] {($1$, $0$)};
    \path let \p0 = (d) in node [left=0.0cm of d] {($0$, $0$)};
    \path let \p0 = (e) in node [left=0.0cm of e] {($0$, $1.\overline{3}$)};

%    \draw[fill] (qopt) circle [radius=0.02];

    % draw the big polygin
    \fill[very thick,fill=blue,fill opacity=0.1] (a.center) -- (b.center) --
                                                 (c.center) -- (d.center) --
                                                 (e.center) -- cycle;

    \draw[scale=1.5,gray, name path=lobj, domain=1.6:2.4] plot(\x, {5 - 2*\x}) {};
%    \draw[gray, name path=lobj, domain=1.1:1.9] plot(\x, {3.75 - 2*\x}) {};
    \draw[scale=1.5,gray, name path=lobj, domain=0.6:1.4] plot(\x, {2.5 - 2*\x}) {};
%    \draw[gray, name path=lobj, domain=0.1:0.9] plot(\x, {1.25 - 2*\x}) {};
    \draw[scale=1.5,gray, name path=lobj, domain=-0.4:0.4] plot(\x, {-2*\x}) {};

\end{tikzpicture}

\caption{Illustration of the feasible region and level sets of the objective
         function.}
\label{fig:lpback}
\end{figure}

For every linear program there exists an associated LP problem called
the \textit{dual} problem.
The dual of the dual is equal to the original problem, which we refer to as
the \textit{primal} problem.
Considering the primal problem as denoted in (\ref{eq:primal}), its dual is
\[
\max{b^T y},\quad \textrm{subject to}~A^T y \geq c, ~ y \geq 0.
\]
These two problems are related on a number of levels.
Note the symmetry between the primal problem and its dual.
Also, every feasible solution for a primal problem gives a bound on the optimal
objective function value for its dual problem, and vice
versa\cite{vanderbei,nocedal,boyd}.

An optimization problem with a quadratic objective function and linear
constraints is called a quadratic program. Since quadratic programs also have
linear constraints, everything we have discussed regarding the feasible region
of linear programs still holds for quadratic programs. We formulate a quadratic
program as:
\[
\min{\frac{1}{2}x^T G x + c^T x},\quad \textrm{subject to}~Ax \leq b, ~ x \geq 0.
\]
where G is a symmetrix $n \times n$ matrix, A is a $m \times n$ matrix
and $c$, $x$ and $b$ are vectors in $\mathbb{R}^n$. If the matrix $G$ is
positive semidefinite, then the objective function is convex, and the
QP problem is considered convex.
Convex QP problems are in general easier to solve than
nonconvex QP problems because nonconvex QP problems can have suboptimal
local minima, whereas convex QP problems does not.
In a convex QP problem, all possible local minima are global
minima. In this thesis, we only consider convex QP problems.

We consider the linear program from before, in which we replace the linear
objective function with a quadratic:
\[
(x_1 - 1)^2 + (x_2 - 1)^2.
\]
Figure \ref{fig:qpback} depicts the feasible region along with our new
quadratic function.
The function's minimum value is obviously found at $(1,1)$, which is
depicted with a small dot in the figure, along with the contours of the
function.
\begin{figure}[ht!]
\centering
\begin{tikzpicture}
    % grid and axes
    \draw[->,name path=xaxis] (-0.2,0) -- (2.2,0) node[right] {$x$};
    \draw[->,name path=yaxis] (0,-0.2) -- (0,2.2) node[above] {$y$};

    % draw lines
    \draw[name path=line1, domain=0.6:2.5] plot(\x,{-\x + 3})
                                                    node[above right=0.2cm] {};
    \draw[name path=line2, domain=-0.2:1.6] plot(\x,{0.33*\x + 1.33})
                                                    node[above right=0.2cm] {};
    \draw[name path=line3, domain=0.7:2.3] plot(\x,{\x - 1})
                                                    node[above right=0.2cm] {};

    % calculate intersection points
    \node[name intersections={of=line1 and line2}] (a) at (intersection-1) {};
    \node[name intersections={of=line1 and line3}] (b) at (intersection-1) {};
    \node[name intersections={of=xaxis and line3}] (c) at (intersection-1) {};
    \node[name intersections={of=xaxis and yaxis}] (d) at (intersection-1) {};
    \node[name intersections={of=line2 and yaxis}] (e) at (intersection-1) {};

    % write the coordinates of the corners.
    \path let \p0 = (a) in node [above right=0.0cm of a] {($1.25$, $1.75$)};
    \path let \p0 = (b) in node [right=0.1cm of b] {\printpoint{\x0}{\y0}};
    \path let \p0 = (c) in node [below right=-0.35cm of c]
                                                       {\printpoint{\x0}{\y0}};
    
    \path let \p0 = (e) in node [left=0.0cm of e] {($0$, $1.\overline{3}$)};

    \node at (1, 1) (qopt) {};
    \draw[fill] (qopt) circle [radius=0.02];

    % draw the big polygin
    \fill[very thick,fill=blue,fill opacity=0.3] (a.center) -- (b.center) --
                                                 (c.center) -- (d.center) --
                                                 (e.center) -- cycle;

    % draw the quadratic objective function
    \draw[thin, dashed] plot[id=qobj1, raw gnuplot] function {
        f(x,y) = x**2 + y**2 - 2*x - 2*y + 1.75;
        set xrange[-1:2];
        set yrange[-2:2];
        set view 0,0;
        set isosamples 1000,1000;
        set size square;
        set cont base;
        set cntrparam levels incre 0,0.1,0;
        unset surface;
        splot f(x,y);
    };
    \draw[thin, dashed] plot[id=qobj2, raw gnuplot] function {
        f(x,y) = x**2 + y**2 - 2*x - 2*y + 1.9;
        set xrange[-1:2];
        set yrange[-2:2];
        set view 0,0;
        set isosamples 1000,1000;
        set size square;
        set cont base;
        set cntrparam levels incre 0,0.1,0;
        unset surface;
        splot f(x,y);
    };
    \draw[thin, dashed] plot[id=qobj3, raw gnuplot] function {
        f(x,y) = x**2 + y**2 - 2*x - 2*y + 1.98;
        set xrange[-1:2];
        set yrange[-2:2];
        set view 0,0;
        set isosamples 1000,1000;
        set size square;
        set cont base;
        set cntrparam levels incre 0,0.1,0;
        unset surface;
        splot f(x,y);
    };
\end{tikzpicture}

\caption{Illustration of the feasible region and the contours of the quadratic
         objective function.}
\label{fig:qpback}
\end{figure}

\subsection{Solution Methods}
Note that in the LP problem illustrated in Figure \ref{fig:lpback},
the optimal solution is found at an extreme point
of the feasible region.
It turns out that in any linear program, there exists an extreme point of the
feasible region which attains the optimal objective value.
It follows from the maximum principle that the maximum of a convex function on
a compact convex set is attained on the boundary\cite{rockafellar}.
Due to this fact, there are a number of algorithms for solving linear
programs that only considers the extreme points of the feasible region.
They normally move along the boundary of the feasible region from one extreme
point to another.
Such algorithms are sometimes referred to as \textit{basis-exchange}
algorithms, among which the well-known \textit{simplex method} categorize as.
In contrast to basis-exchange algorithms, there are \textit{interior point}
methods that move in the interior of the feasible region.

In the simplex method, most steps decrease (or increase if we are maximizing)
the objective value.
%By moving from one extreme point to another, the simplex algorithm 
%greedily chooses the adjacent extreme point which will give the most increase
%(or decrease) in the objective value. 
While in practice, the simplex algorithm is quite efficient, and is in fact
the most widely used of all optimization tools today\cite{nocedal}, there are
linear programs such as the Klee-Minty cube\cite{klee} where the simplex method
requires $2^n - 1$ iterations to solve (here $n = m$).

As the simplex method's worst case complexity is proven to be exponential
in the size of the problem, the search for algorithms with better theoretical
properties arose.
Interior point methods have a polynomial worst case time
complexity in the size of the problem. This is much more efficient in
theory, but in practice, the simplex method tends to be faster than interior
point methods on most real-life instances of linear programming.
While interior point methods tend
to do few, but expensive iterations, the simplex method tends to do
many inexpensive iterations.

We mentioned earlier that some QPs are more difficult to solve than others.
For convex QPs, some methods can solve the problem in polynomial time, such
as the \textit{ellipsoid method}\cite{kozlov}. However, solving nonconvex
QPs is shown to be NP-hard\cite{sahni}.
