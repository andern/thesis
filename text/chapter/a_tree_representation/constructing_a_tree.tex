Before we can construct a tree similar to the one in Figure \ref{fig:find},
we need a slighty modified \texttt{find} that we call \texttt{mfind}.
\texttt{mfind} is modified to take an output argument that holds an index for
a vertex we want to return, while the function returns a boolean.
The returned boolean tells us whether the output index is the index of a
solution or not. If the returned boolean is true, then the output argument is
an index of a solution vertex, i.e. the same vertex that would be returned
by \texttt{find}.
If the boolean is false, then the output argument is an index of a vertex
that would be the parent vertex if we were to insert the modifier we were
looking for. While this can sound confusing, Algorithm \ref{alg:mfind} shows
in detail how \texttt{mfind} works.

\begin{algorithm}[ht!]
\caption{\texttt{mfind($\mathcal{M}_l$, $v_k$, $v_*$)}}
\label{alg:mfind}
\begin{algorithm}[H]
\caption{\texttt{mfind($\mathcal{M}_l$, $v_k$)}}
\label{alg:mfind}
%\KwIn{($\mathcal{M}_l$, $\mathcal{Q}_k$)}
\SetAlgoVlined
\ForEach{child vertex $v_i$ of $v_k$}{
  \If{$\mathcal{M}_i \subseteq \mathcal{M}_l$}{
    $a \leftarrow i$\\
    \eIf{$\mathcal{M}_l \subseteq \mathcal{Z}_i$}{
      \textbf{return} (\texttt{true}, $i$)
%      \KwRet{$i$}
    }{
      \texttt{mfind($\mathcal{M}_l$, $v_i$)}
%      \KwRet{find($\mathcal{M}_l$, $v_i$)}
    }
  }
}
\textbf{return} (\texttt{false}, $a$)
\end{algorithm}

\end{algorithm}
To construct a tree such as the one seen in Figure \ref{fig:find}, we start
with some instance $\mathcal{Q}$, a modifier $\mathcal{M}_0 = \{ {} \}$ and
a root $v_0$.
Then, for each possible subinstance $\mathcal{Q}_i$ of $\mathcal{Q}$ such that
$|\mathcal{M}_i| \leq b$, we run \texttt{mfind} and add it to the tree if its
solution is distinct.

\begin{algorithm}[ht!]
\caption{\texttt{construct}($\mathcal{Q}$, $b$)}
\label{alg:construct}
$V \leftarrow \{v_0\}$ \\
$E \leftarrow \emptyset$ \\
\SetAlgoVlined

\ForEach{modifier $\mathcal{M}_i \in \mathcal{P}(\left\{{1,2,\ldots,n}\right\})
, |\mathcal{M}_i| \leq b$}{
    $P \leftarrow$ \texttt{mfind}($\mathcal{M}_i$, $v_0$, $v_k$) \\
    \If{$\neg P$}{
        $V \leftarrow V \cup \{v_i\}$ \\
        $E \leftarrow E \cup \{(k, i)\}$ \\
    }
}
$G \leftarrow (V, E)$ \\
\KwRet{$G$}

\end{algorithm}
Algorithm \ref{alg:construct} describes an algorithm for constructing a tree
$G$, given some instance $\mathcal{Q}$ and a maximum number of breakdowns $b$.
Readers are encouraged to try to construct the tree in Figure \ref{fig:find} on
their own, following Algorithm \ref{alg:construct} and using the
$\mathcal{Z}$-sets above. Note that the tree might change if the order of
insertion changes. The combinations of modifiers under the construction of the
tree in Figure \ref{fig:find} were inserted in lexicographical order.
