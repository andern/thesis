\section{Subinstances}
For one or more given variables, their lower and upper bounds are set to zero,
i.e. forcing the variables to zero.
Let us denote the set of variables that are to be forced to zero by
$\mathcal{M}_k$ for some $1 \leq k \leq 2^n - 1$ where $n$ is the number
of edges in the network.
Forcing these variables to zero in some instance $\mathcal{Q}$ forms
a subinstance $\mathcal{Q}_k$. We refer to $\mathcal{M}_k$ as a
\emph{modifier} of $\mathcal{Q}$.
To simplify notation from here on, we let
$\mathcal{Q}_0 = \mathcal{Q}$ and loosen the terminology of
\emph{subinstance} so that $\mathcal{Q}_0$ can be called a subinstance of
$\mathcal{Q}$ where its modifier $\mathcal{M}_0 = \emptyset$.

To distinguish between optimal solutions over several subinstances, we denote
an optimal solution of $\mathcal{Q}_k$ as $x_k^*$.
In any solution of any instance there might be variables that are zero.
We denote the set of variables that are zero in an \emph{optimal} solution
$x_k^*$ by $\mathcal{Z}_k$.
Additionally, we denote the set of variables that are non-zero in an 
optimal solution $x_k^*$ by $\mathcal{N}_k$.
If a modifier $\mathcal{M}_k$ only has variables that are zero in $x_0^*$,
i.e. if $\mathcal{M}_k \subseteq \mathcal{Z}_0$, then $x_k^* = x_0^*$. However,
if $\mathcal{M}_k$ has some variables that are non-zero in $x_0^*$, then
we solve $\mathcal{Q}_k$ to achieve its solution $x_k^*$ along with the set
$\mathcal{Z}_k$.

Generally, for each $\mathcal{Q}_k$ and its modifier $\mathcal{M}_k$ we
have that $x_l^* = x_k^*$ for all $l=1,2,\ldots,2^n-1$ if
$\mathcal{M}_l = S \cup \mathcal{M}_k$ for some $S \subseteq \mathcal{Z}_k$.
That is, instances $\mathcal{Q}_k$ and $\mathcal{Q}_l$ have identical optimal
solutions if $\mathcal{M}_l$ is obtained by extending $\mathcal{M}_k$ only with
variables that are zero in the optimal solution to $\mathcal{Q}_k$.
Additionally, for each $\mathcal{Q}_k$ and its modifier $\mathcal{M}_k$ we
have that $x_l^* \neq x_k^*$ for all $l=1,2,\ldots,2^n-1$ if
$\mathcal{M}_l = S \cup \mathcal{M}_k$ for some $S\not\subseteq \mathcal{N}_k$.
That is, instances $\mathcal{Q}_k$ and $\mathcal{Q}_l$ have different optimal
solutions if $\mathcal{M}_l$ is obtained by extending $\mathcal{M}_k$ with any
variables that are non-zero in the optimal solution to $\mathcal{Q}_k$.
