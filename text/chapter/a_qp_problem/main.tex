\chapter{A Quadratic Programming (QP) Problem}
\label{ch:problem}
A general formulation of non-linear optimization problems is
\begin{equation}
    \label{eq:generalqp}
    \min_{x \in \mathbb{R}^n} f(x) \quad \textrm{subject to}
    \begin{cases}
        ~c_i(x) = 0,   & i \in \mathcal{E}, \\
        ~c_i(x) \ge 0, & i \in \mathcal{I},
    \end{cases}
\end{equation}
where $f$ and the functions $c_i$ are all smooth, real-valued functions on a
subset of $\mathbb{R}^n$, and $\mathcal{I}$ and $\mathcal{E}$ are two finite
sets of indices~\cite{nocedal}.

The problem discussed in this thesis has a convex, quadratic, separable
objective function
\begin{equation}
    \label{eq:obj}
    f(x) = x^T H x + b^T x
\end{equation}
and linear constraints. It qualifies as a QP problem and can be formulated as
\begin{equation}
    \label{eq:thesisqp}
    \min_{x \in \mathbb{R}^n} f(x)
    \quad \textrm{subject to}
    ~
    Ax = 0
    ~
    \textrm{and}
    ~
    l \le x \le u
\end{equation}
where $H$ is a positive semidefinite diagonal $n \times n$ matrix, $A$ is a
$m \times n$ oriented incidence matrix, and $b, l, u$ and $x$ are vectors in
$\mathbb{R}^n$.

The function $f$ in (\ref{eq:generalqp}) is called the objective function of
the problem. In (\ref{eq:obj}) and (\ref{eq:thesisqp}), $f$ consists of a
quadratic term ($x^T H x$) and a linear term ($b^T x$).

In practical applications, a large proportion of the diagonal elements of H are
zero, and for the non-zero elements we typically have
$10^{-5} \le h_i \le 10^{-1}$. Here $h_i$ denotes the $i$th diagonal element of
$H$. For the components $b_i$ of vector $b$, we typically have
$10 \le |b_i| \le 70$. All elements of $l$ and $u$ are non-positive 
and non-negative, respectively. However, the methods developed in this thesis
do not require these values.
