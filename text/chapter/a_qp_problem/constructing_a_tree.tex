\section{Constructing a Tree}
\label{sec:construction}
To construct a tree such as the one seen in Figure \ref{fig:find}, we start
with some instance $\mathcal{Q}$, a modifier $\mathcal{M}_0 = \{ {} \}$ and
a root $v_0$. After solving $\mathcal{Q}$, we know that all subinstances
$\mathcal{Q}_k$ such that $\mathcal{M}_k \subseteq \mathcal{Z}_0$, will have
a solution $x_k^* = x_0^*$. Because of this fact, we do not want to generate
modifiers $\mathcal{M}_k$ such that $\mathcal{M}_k \subseteq \mathcal{Z}_0$,
but rather generate modifiers $\mathcal{M}_k$ such that $\mathcal{M}_k
\subseteq \mathcal{N}_0$.
For each modifier $\mathcal{M}_k \subseteq \mathcal{N}_0$, we run
\texttt{find($\mathcal{M}_k$, $v_0$)} to check if we already have the solution
$x_k^*$.
If we do not have the solution, we \emph{immediately} solve $\mathcal{Q}_k$
and make $v_k$ a child of $v_0$. Once we have all child vertices $v_k$ of $v_0$
such that all $\mathcal{Q}_k$ have distinct solutions and $\mathcal{M}_k
\subseteq \mathcal{Z}_0$, we continue the same operation on $v_k$ as we did
on $v_0$. Note that we always start Algorithm \ref{alg:find} from $v_0$
regardless of which vertex we are currently operating on.

$V \leftarrow \{v_0\}$ \\
$E \leftarrow \emptyset$ \\
\SetAlgoVlined

\ForEach{modifier $\mathcal{M}_i \in \mathcal{P}(\left\{{1,2,\ldots,n}\right\})
, |\mathcal{M}_i| \leq b$}{
    $P \leftarrow$ \texttt{mfind}($\mathcal{M}_i$, $v_0$, $v_k$) \\
    \If{$\neg P$}{
        $V \leftarrow V \cup \{v_i\}$ \\
        $E \leftarrow E \cup \{(k, i)\}$ \\
    }
}
$G \leftarrow (V, E)$ \\
\KwRet{$G$}

Algorithm \ref{alg:construct} describes an algorithm for constructing a tree
$G$, given some instance $\mathcal{Q}$. Readers are encouraged to try to
construct the tree in Figure \ref{fig:find} on their own.
