\section{Mapping Subinstances}
In this section we will discuss how to map all subinstances of some arbitrary
instance $\mathcal{Q}$. We will also discuss how to find 
subinstances that we do not already know the solution to.
We will construct a \emph{tree} of these ``unique'' solutions which will allow
us to efficiently check if some modifier will form a subinstance with an
unknown solution.

If we were to try to solve all $2^n$ possible subinstances of an
instance, we would most likely end up solving a huge amount of subinstances
with the same optimal solution. This is evident in some of the real
instances; when we solve the unmodified instance \texttt{vlarge}, we already
have the solution to $2^{1749}$ of its possible subinstances, i.e.
$|\mathcal{Z}_0| = 1749$.
Just trying to unnecessarily solve all these subinstances---let alone store the
solutions---would be a huge waste of resources.
Instead, we try to find subinstances that we do not know the solution to.
A good starting point would be to find all the subsets of $\mathcal{N}_0$ and
use them as modifiers of $Q$. That would be equivalent to forcing non-zero
variables in the optimal solution of the unchanged instance $Q$, to zero.
Recall from the previous section that if $\mathcal{Q}$ is known, the modifier
$\mathcal{M}_k$ is enough to find the subinstance $\mathcal{Q}_k$ for some
$k=1,2,\ldots,2^n-1$.
While finding modifiers $\mathcal{M}_k$ such that
$\mathcal{M}_k \subseteq \mathcal{N}_0$ would assure us that $x_k^*\neq x_0^*$,
it does not assure us that $x_k^* \neq x_l^*$ for all $l=1,2,\ldots,2^n-1$.
To make sure that we do not solve any unnecessary subinstances, we need to make
sure to find modifiers $\mathcal{M}_k$ such that
$\mathcal{M}_k \not \subseteq \mathcal{Z}_l$
for \emph{all} $l=1,2,\ldots,2^n-1$.

Consider a set $\mathbb{U}$ that contains all subinstances $\mathcal{Q}_k$
such that $x_k^* \neq x_l^*$ for all $\mathcal{Q}_l \in \mathbb{U}$.
