\begin{figure}[h!]
\begin{center}
\input{include/newhistH}
\end{center}
\caption{Value distribution of non-zero elements in $H$}
\label{fig:histH}
\end{figure}

\section{The Instances}
The instances have a lot in common, but most importantly they all:
\begin{enumerate}
\item Have very large coefficients in the linear term compared to the
quadratic term.
\item All cross-product coefficients are zero in the quadratic term, i.e. in
$x^T H x$, $H$ is diagonal.
\end{enumerate}

The instances are called---based on size---\textit{small}, \textit{large}
and \textit{vlarge}.

Figure \ref{fig:histH} shows the value distribution of non-zero elements in $H$
in the three instances. Note that among all the diagonal elements of H, more
than 50 percent of them are zero, and among the non-zero elements, most of them
are less than $10^{-2}$.
Observing that no non-zero element in the linear term in any of the instances
has absolute value less than 20 reinforces our earlier motivation and
cause to look into SLP.

\begin{table}
    \centering
    \caption{Problem size of each instance}
    \begin{tabular}{lrrr}
    Problem size & \textit{small} & \textit{large} & \textit{vlarge} \\\hline
    Rows         & 82             & 328            & 1127 \\
    Columns      & 238            & 952            & 3437 \\
    Non-zeroes A & 348            & 1392           & 4840 \\
    Non-zeroes H & 108            & 432            & 894 \\
    \end{tabular}
    \label{table:sizes}
\end{table}

Let us take a look at some hard statistics of each instance.
Table \ref{table:sizes} shows the problem size of each instance.
Table \ref{table:maxmin} shows some statistics about the non-zero elements in
the objective function for $i = 1,2,\ldots,n$ where $n$ is the number of
columns.
\begin{table}
    \centering
    \caption{Non-zero values in the objective function of each instance}
    \begin{tabular}{lrrr}
                        & \textit{small}          & \textit{large}          & \textit{vlarge} \\\hline
    $\max(h_{ii})$      & $2.9614 \times 10^{-2}$ & $2.9614 \times 10^{-2}$ & $4.9011 \times 10^{-2}$ \\
    $\min(h_{ii})$      & $4.9290 \times 10^{-5}$ & $4.9290 \times 10^{-5}$ & $1.1026 \times 10^{-5}$ \\
$\textrm{mean}(h_{ii})$ & $5.2864 \times 10^{-3}$ & $5.2864 \times 10^{-3}$ & $5.8984 \times 10^{-3}$ \\
    $\max(b_{i})$       & 20                      & 20                      & 20 \\
    $\min(b_{i})$       & -70                     & -70                     & -50 \\
    \end{tabular}
    \label{table:maxmin}
\end{table}
The rows and columns in each problem represent vertices and edges respectively.
An avid reader might notice that \textit{small} and \textit{large} are very
much alike.
That is entirely justified as \textit{large} is four \textit{small}-networks
connected forming one large network.

For each of these instances, GoodTech AS wants to solve several other instances
with slight variations.
These instances are from here on called \emph{subinstances}.
If one or more variables in an instance $\mathcal{Q}$ is forced to zero to form
a new instance, the newly formed instance is a subinstance of $\mathcal{Q}$.
GoodTech AS wants to simulate that all combinations of edges in the network
falls down, by forcing their corresponding variable values to zero.
This means that for each of the three instances described in this chapter,
they ideally want to solve a total of $2^n$ subinstances.
Solving all $2^n$ subinstances is  impossible.
It is unlikely that all edges in the network break down.
A more realistic approach is to try to solve all subinstances with no more than
$k$ break-downs, i.e. we solve $\sum_{j=0}^k {n \choose j}$ subinstances
with less than or equal to $k$ break-downs.
