\chapter{Implementation}
Recall from Chapter \ref{ch:qp} that we discussed limiting the amount
of subinstances to solve. We did this by introducing a limit on the number of
break-downs to simulate in the network.
Another approach is to implement a service that waits for a query from
a client containing a modifier of some instance. The server keeps a memory
of unique solutions and its corresponding modifier.
For each query, the server checks if it has previously solved a subinstance
that should have the same solution. If that is the case, simply return that
solution. If not, solve the new subinstance and store its solution. We can
impose a memory limit that causes the server to delete old solutions when the
memory is full.

The difference in these two approaches are prominent, but they share the same
core, namely a QP solver. In this chapter we first discuss the implementation
of the solver declared in Chapter \ref{ch:slp}, before we move on to the
implementation of the two approaches of solving several subinstances.
\label{ch:implementation}

The method described in Chapter \ref{ch:slp} relies on repetitively solving
linear programs. Before implementing such a method, it is important to choose
an appropriate LP solver. The solver must
\begin{inparaenum}[\itshape a\upshape)]
\item be released under a free license; and
\item it must allow library calls in C/C++
\end{inparaenum}

Among a list of about 50 solvers (most of them proprietary), the
following three matches the aforementioned criteria:
\begin{description}
\item[Clp] \hfill \\
Clp is short for Coin-or linear programming. It is a free linear programming
solver released under the Common Public License (CPL). It is primarily meant to
be used as a callable library. Its license allows other software to link to the
Clp library without requiring that that software is released under the same
license (permissive). \cite{clp}
\item[GLPK] \hfill \\
GLPK is short for GNU Linear Programming Kit. It is a free linear (integer)
programming software packaged released under the GNU General Public License
(GNU GPL). GLPK is organized in the form of a callable library. Its license
requires linking software to be released under the GNU GPL (reciprocal).
\cite{glpk}
\item[lp\_solve] \hfill \\
lp\_solve is a free linear (integer) programming solver released under the GNU 
Lesser General Public License (GNU LGPL). Its license is permissive like the
CPL. \cite{lpsolve}
\end{description}

Incidentally, these tree solvers are the only three open source LP solvers
suggested by the NEOS Optimization Guide~\cite{neos}.
In addition to the mentioned criteria, it is important that the solver is
fast.
Moreover, it needs to be fast on problems that fit the description in
Section \ref{sec:problem}.

Testing the three solvers on linearized versions of the instances
\textit{small} and \textit{large}---presented in
Chapter \ref{sec:instances}---reveals the running times shown in Table
\ref{table:lpres}.
The running times are the number of seconds in CPU-time after 1000 runs.

\begin{table}[ht!]
    \centering
    \caption{Running time in CPU-seconds used by each solver to solve 1000
             instances of each problem.}
    \begin{tabular}{lrrr}
        Data Set       & Clp    & GLPK   & lp\_solve \\ \hline
        \textit{small} & 6.929  & 26.096 & 4.734 \\
        \textit{large} & 12.832 & 47.977 & 23.376
    \end{tabular}
    \label{table:lpres}
\end{table}

While lp\_solve is the fastest on a smaller LP, it doesn't scale as well as
with larger problems like Clp does.

GoodTech need to solve problems with more than 200 variables, and
\textit{small} is almost hitting that lower limit, so a good result on
\textit{large} is prioritized over the other.
This means that Clp outperforms the other two solvers in this test.

Although this is not an extensive test, it is a pretty good indicator that Clp
will be the fastest solver on similar problems.


Clp is written primarily by John J. Forrest, now retired from IBM Research. At
the time of writing, Clp is under active development. It is currently
managed by John Forrest, Julian Hall, Lou Hafer and Matthew Saltzman.
\cite{clppage}

Matrices in Clp are stored in a compact format using three vectors.
The first vector contains all the non-zero elements of the matrix.
The second vector contains the indices of the elements in the first vector.
The third vector contains an accumulated number of elements in each row/column.
The order of the elements depends whether the matrix is row-ordered or
column-ordered.
The indices represent the column/row position of the elements, and the
accumulated values represent the accumulated number of elements in the
row/columns depending on whether the matrix is row-ordered or column-ordered,
respectively.
To get a better understanding of how they are stored, consider the matrix
\[
\left[
\begin{array}{rrrrrrrr}
    3 & 1 & 0   & -2  & -1 & 0 & 0    & -1 \\
    0 & 2 & 1.1 & 0   & 0  & 0 & 0    & 0  \\
    0 & 0 & 1   & 0   & 0  & 1 & 0    & 0  \\
    0 & 0 & 0   & 2.8 & 0  & 0 & -1.2 & 0  \\
  5.6 & 0 & 0   & 0   & 1  & 0 & 0    & 1.9  

\end{array}
\right]
\]
being stored in row-ordered format, then the element vector would be
\[
\begin{array}{l}
\left[
\begin{array}{rrrrrrr}
    3 & 1 & -2 & -1 & -1 & 2 & 1.1
\end{array}\right. \\
\left.\begin{array}{rrrrrrr}
    1 & 1 & 2.8 & -1.2 & 5.6 & 1 & 1.9
\end{array}
\right]^T,
\end{array}
\]
the vector containing the column indices would be
\[
\left[
\begin{array}{rrrrrrrrrrrrrr}
    0 & 1 & 3 & 4 & 7 & 1 & 2 & 2 & 5 & 3 & 6 & 0 & 4 & 7
\end{array}
\right]^T,
\]
and the vector containing the vector starts (the accumulated indices) would be
\[
\left[
\begin{array}{rrrrrr}
    0 & 5 & 7 & 9 & 11 & 14
\end{array}
\right]^T.
\]
The \texttt{CoinPackedMatrix} class represents such a compact matrix.
To keep overhead at a minimum, it is important to implement the new solver
(from here on called Slp) using the same compact format.

Clp has a base class that holds data for both linear and quadratic models
called \texttt{ClpModel}.
The class itself knows nothing about the implementation of the algorithms, but
it contains all the data about a problem needed to apply an algorithm on that
data.
This is very handy for passing data throughout Slp without much overhead.
It also results in very neat function prototypes as only one parameter
is needed for passing lots of information. For instance, the function prototype
in Slp for the function that performs the line search described in Chapter
\ref{ch:slp} looks like this:
\begin{verbatim}
double lineSearch(const double* a,
                  const double* b,
                  ClpModel& m);
\end{verbatim}
Figure \ref{fig:clpmodel} shows the class hierarchy for \texttt{ClpModel}.
The subclasses contain the implementation of the algorithms that their class
names describe. For instance, \texttt{ClpSimplexDual} contains the
implementation of the Dual Simplex Algorithm.
\begin{figure}
\centering
\begin{tikzpicture}[node distance=2.5cm]
    \tikzstyle{every state}=[rectangle,
                             minimum width=10em,
                             rounded corners,
                             fill=blue!30,
                             edge from parent,
                             text=black]
    \tikzset{node distance=1.7cm};

    \node[state] (a)                    {\texttt{ClpModel}};
    \node[state] (b) [below of = a]     {\texttt{ClpSimplex}};
    \node[state] (c) [right = 0.5cm of b] {\texttt{ClpInterior}};
    \node[state] (d) [below of = b]     {\texttt{ClpSimplexDual}};
    \node[state] (e) [right = 0.5cm of d] {\texttt{ClpSimplexPrimal}};

    \tikzset{edgestyle/.style={-, double=black}}
%    \tikzset{every node/.style={fill=white}}

    \path (a) edge [edgestyle] node {}  (b)
          (a) edge [edgestyle] node {}  (c)
          (b) edge [edgestyle] node {}  (d)
          (b) edge [edgestyle] node {}  (e);
\end{tikzpicture}
\caption{Class hierarchy for \texttt{ClpModel}}
\label{fig:clpmodel}
\end{figure}

