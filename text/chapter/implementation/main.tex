\chapter{Implementation}
Recall from Chapter \ref{ch:qp} that we discussed limiting the amount
of subinstances to solve. We did this by introducing a limit on the number of
simultaneous break-downs in the network by some $\beta$.
Another approach is to implement a discrete-event simulator (DES). Each event
occurs at a particular instant in time that changes the state of the system.
In this case, each event would either be a breakdown in the network, or that
a breakdown is fixed.

The difference in these two approaches are prominent, but they share the same
core, namely a QP solver. In this chapter we first discuss the implementation
of the solver declared in Chapter \ref{ch:slp}, before we move on to the
implementation of the two approaches of solving several subinstances.
\label{ch:implementation}

\section{Finding an LP Solver}
The method described in Chapter \ref{ch:slp} relies on repetitively solving
linear programs. Before implementing such a method, it is important to choose
an appropriate LP solver. The solver should 
\begin{inparaenum}[\itshape a\upshape)]
\item be released under a free license; and
\item allow library calls in C/C++
\end{inparaenum}

Among a list of about 50 solvers (most of them proprietary), the
following three matches the aforementioned criteria:
\begin{description}
\item[Clp] \hfill \\
Clp is short for Coin-or linear programming. It is a free linear programming
solver released under the Common Public License (CPL). It is primarily meant to
be used as a callable library. Its license allows other software to link to the
Clp library without requiring that that software is released under the same
license (permissive). \cite{clp}
\item[GLPK] \hfill \\
GLPK is short for GNU Linear Programming Kit. It is a free linear (integer)
programming software packaged released under the GNU General Public License
(GNU GPL). GLPK is organized in the form of a callable library. Its license
requires linking software to be released under the GNU GPL (reciprocal).
\cite{glpk}
\item[lp\_solve] \hfill \\
lp\_solve is a free linear (integer) programming solver released under the GNU 
Lesser General Public License (GNU LGPL). Its license is permissive like the
CPL. \cite{lpsolve}
\end{description}
Incidentally, these tree solvers are the only three open source LP solvers
suggested by the NEOS Optimization Guide~\cite{neos}.
In addition to the mentioned criteria, it is important that the solver is
fast.
Moreover, it needs to be fast on problems that fit the description in
Section \ref{sec:problem}.

Testing the three solvers on linearized versions of the instances
\textit{small} and \textit{large}---presented in
Chapter \ref{sec:instances}---reveals the running times shown in Table
\ref{table:lpres}.
The running times are the number of seconds in CPU-time after 1000 runs.

\begin{table}[ht!]
    \centering
    \caption{Running time in CPU-seconds used by each solver to solve each
             instance 1000 times.}
    \begin{tabular}{lrrr}
        Data Set       & Clp    & GLPK   & lp\_solve \\ \hline
        \textit{small} & 6.929  & 26.096 & 4.734 \\
        \textit{large} & 12.832 & 47.977 & 23.376
    \end{tabular}
    \label{table:lpres}
\end{table}
While lp\_solve is the fastest solver on a smaller LP, it doesn't scale as well
as with larger problems as Clp does.

GoodTech need to solve problems with more than 200 variables, and
\textit{small} is almost hitting that lower limit, so a good result on
\textit{large} is prioritized over the other.
This means that Clp outperforms the other two solvers in this test.

Although this is not an extensive test, it is a pretty good indicator that Clp
is the fastest solver on similar problems.


\section{Clp}
Clp is written primarily by John J. Forrest, now retired from IBM Research. At
the time of writing, Clp is under active development. It is currently
managed by John Forrest, Julian Hall, Lou Hafer and Matthew Saltzman.
\cite{clppage}

Matrices in Clp are stored in a compact format using three vectors.
The first vector contains all the non-zero elements of the matrix.
The second vector contains the indices of the elements in the first vector.
The third vector contains an accumulated number of elements in each row/column.
The order of the elements depends whether the matrix is row-ordered or
column-ordered.
The indices represent the column/row position of the elements, and the
accumulated values represent the accumulated number of elements in the
row/columns depending on whether the matrix is row-ordered or column-ordered,
respectively.
To get a better understanding of how they are stored, consider the matrix
\[
\left[
\begin{array}{rrrrrrrr}
    3 & 1 & 0   & -2  & -1 & 0 & 0    & -1 \\
    0 & 2 & 1.1 & 0   & 0  & 0 & 0    & 0  \\
    0 & 0 & 1   & 0   & 0  & 1 & 0    & 0  \\
    0 & 0 & 0   & 2.8 & 0  & 0 & -1.2 & 0  \\
  5.6 & 0 & 0   & 0   & 1  & 0 & 0    & 1.9  

\end{array}
\right]
\]
being stored in row-ordered format, then the element vector would be
\[
\begin{array}{l}
\left[
\begin{array}{rrrrrrr}
    3 & 1 & -2 & -1 & -1 & 2 & 1.1
\end{array}\right. \\
\left.\begin{array}{rrrrrrr}
    1 & 1 & 2.8 & -1.2 & 5.6 & 1 & 1.9
\end{array}
\right]^T,
\end{array}
\]
the vector containing the column indices would be
\[
\left[
\begin{array}{rrrrrrrrrrrrrr}
    0 & 1 & 3 & 4 & 7 & 1 & 2 & 2 & 5 & 3 & 6 & 0 & 4 & 7
\end{array}
\right]^T,
\]
and the vector containing the vector starts (the accumulated indices) would be
\[
\left[
\begin{array}{rrrrrr}
    0 & 5 & 7 & 9 & 11 & 14
\end{array}
\right]^T.
\]
The \texttt{CoinPackedMatrix} class represents such a compact matrix.
To keep overhead at a minimum, it is important to implement the new solver
(from here on called Slp) using the same compact format.

Clp has a base class that holds data for both linear and quadratic models
called \texttt{ClpModel}.
The class itself knows nothing about the implementation of the algorithms, but
it contains all the data about a problem needed to apply an algorithm on that
data.
This is very handy for passing data throughout Slp without much overhead.
It also results in very neat function prototypes as only one parameter
is needed for passing lots of information. For instance, the function prototype
in Slp for the function that performs the line search described in Chapter
\ref{ch:slp} looks like this:
\begin{verbatim}
double lineSearch(const double* a,
                  const double* b,
                  ClpModel& m);
\end{verbatim}
Figure \ref{fig:clpmodel} shows the class hierarchy for \texttt{ClpModel}.
The subclasses contain the implementation of the algorithms that their class
names describe. For instance, \texttt{ClpSimplexDual} contains the
implementation of the Dual Simplex Algorithm.
\begin{figure}
\centering
\begin{tikzpicture}[node distance=2.5cm]
    \tikzstyle{every state}=[rectangle,
                             minimum width=10em,
                             rounded corners,
                             fill=blue!30,
                             edge from parent,
                             text=black]
    \tikzset{node distance=1.7cm};

    \node[state] (a)                    {\texttt{ClpModel}};
    \node[state] (b) [below of = a]     {\texttt{ClpSimplex}};
    \node[state] (c) [right = 0.5cm of b] {\texttt{ClpInterior}};
    \node[state] (d) [below of = b]     {\texttt{ClpSimplexDual}};
    \node[state] (e) [right = 0.5cm of d] {\texttt{ClpSimplexPrimal}};

    \tikzset{edgestyle/.style={-, double=black}}
%    \tikzset{every node/.style={fill=white}}

    \path (a) edge [edgestyle] node {}  (b)
          (a) edge [edgestyle] node {}  (c)
          (b) edge [edgestyle] node {}  (d)
          (b) edge [edgestyle] node {}  (e);
\end{tikzpicture}
\caption{Class hierarchy for \texttt{ClpModel}}
\label{fig:clpmodel}
\end{figure}


\section{Slp}
SLP was first referenced in \cite{slp61}, a paper by Griffith and Stewart in
1961~\cite{boggs1985numerical}.
They described a procedure they called Mathematical Approximation Programming
(MAP) that was in use at Shell Oil Company at the time. The paper is considered
a classic in the field of optimization

TODO: Belongs in an introduction chapter?

SLP is a differential technique which utilizes the linear programming algorithm
repetitively in such a way that the solution of a linear problem can converge
to the solution of the nonlinear problem~\cite{slp61}. The results above show
us that applying an SLP algorithm to a QP with the characteristics mentioned in
Section \ref{sec:problem}, with an initial guess 0, will give us an approximate
solution deviating less than $6.42\%$ from the optimal solution, after only one
iteration.


\section{Solving Subinstances}
\section{Solving Subinstances}
In this section, we discuss the different approaches to solving several
subinstances. As mentioned in the beginning of the chapter, the two
main approaches are to solve a limited amount of subinstances, bounded
by some $s$, or to implement a discrete event simulator that only stores
the current state of the network. First we will discuss the implementation
of the tree structure, before we move on to the two main approaches.

\subsection{Tree Structure}
All trees consists of vertices, and in our specific case, each vertex contains
two sets $\mathcal{M}$ and $\mathcal{Z}$, a solution $x^*$, and a number of
child vertices. To keep vertices as simple as possible, vertices were
implemented as simple \texttt{structs}. Each vertex \texttt{struct} looks like
this:
\begin{verbatim}
struct vertex {
    std::set<uint16_t> m;
    std::set<uint16_t> z;
    double* sol;
    std::vector<struct vertex*> children;
};
\end{verbatim}
A reader familiar with \texttt{C++} might notice that we use
\texttt{std::set} instead of using a bit set as we discussed
in Chapter \ref{ch:tree}, and we will come to that later. Note that
the sets consist of indices of primitive type \texttt{uint16\_t}, which is
short for \texttt{unsigned short int}. That means that the sets are limited to
$2^{16} = 65536$ elements, and therefore also putting a limit to the size
of the QP problems that can be solved. This seems like a reasonable limit
as GoodTech wants to solve problems with $200 - 2000$ variables.
If they need to solve larger problems in the future, it is possible to change
the primitive data type.

As soon as a vertex is defined, we are ready to implement \texttt{mfind}.
Remember that \texttt{mfind} is a modified version of \texttt{find} that tells
us which vertex should be the parent of our potentially new vertex in case it
has a distinct solution.
\texttt{mfind} is defined in two parts. The first part is a ''helper'' function
for starting the algorithm on some vertex. The second part is the actual
recursively defined algorithm. These two parts are called \texttt{mfind} and
\texttt{mfindrec}, respectively. We define \texttt{mfind} as follows:
\newpage
\begin{verbatim}
struct vertex* mfind(const std::set<uint16_t>&m,
const struct vertex* v, bool& found) {
    found = true;
    if (isSubset(v->m, m) && isSubset(m, v->z)) return v;
    found = false;
    return mfindrec(m, v, found, v);
}
\end{verbatim}
The parameters for \texttt{mfind} are a modifier, a vertex where the search
will begin and a boolean value that will contain tell us whether we found a
solution or not.
In order to search the whole tree, one would use the root vertex as parameter
\texttt{v} here. The function returns a pointer to a vertex of our interest.
Note that this is a pointer, and not an index like in Algorithm
\ref{alg:mfind}. The function makes a call to \texttt{mfindrec}:
\begin{verbatim}
struct vertex* mfindrec(const std::set<uint16_t>& m,
const struct vertex* v, bool& found, struct vertex* ret) {
    for (struct vertex* vi : v->children) {
        if (isSubset(vi->m, m) {
            ret = vi;
            if (isSubset(m, vi->z)) {
                found = true;
                return vi;
            } else mfindrec(m, vi, found, ret);
        }
    }
    return ret;
}
\end{verbatim}
TODO: Begin to talk about bitset and std::set performance. Show benchmark
results.








\newpage
\begin{comment}
struct vertex* findrec(const std::set<uint16_t>& m, struct vertex* v) {
    for (int i = 0; i < v->children.size(); i++) {
        struct vertex* child = v->children[i];
        if (isSubset(child->m, m) {
            if (isSubset(m, child->z))
                return child;
            else
                findrec(m, child);
        }
    }
    return 0;
}
The function \texttt{isSubset} is just a one-line function that calls
\texttt{std::includes(...)}. One might notice that we do not check if the input
vertex contains the solution to the input modifier. However, we \emph{do} check
this in \texttt{find}:
struct vertex* find(const std::set<uint16_t>& m, struct vertex* v) {
    if (v == 0) return 0;
    if (isSubset(v->m, m) && isSubset(m, v->z) return v;
    return findrec(m, v);
}
\end{comment}


\section{A Quick Overview}
In this section we present a quick overview of the different
implementations.

Although Clp is written primarily for solving linear
programs, it comes with a QP solver. This QP solver uses an implementation
of a primal-dual predictor-corrector interior point method. In addition
to Clp's QP solver, we also have the QP solver presented in Section
\ref{ch:slp}.

There are four different implementations, named
\texttt{construct\_clp}, \texttt{construct\_slp}, \texttt{naive\_clp} and
\texttt{naive\_slp}. The implementations with names ending with
\texttt{\_clp} and \texttt{\_slp} uses Clp's QP solver and Slp, respectively,
whenever a QP needs to be solved.
The \texttt{construct} implementations uses the tree structured presented
in Section \ref{ch:tree} for storing subinstances with distinct solutions.
The \texttt{naive} implementations, however, solves subinstances regardless of
whether they have distinct solutions or not.

In the next section, we present several experiments to perform on our
implementations.
