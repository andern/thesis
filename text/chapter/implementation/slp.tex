\section{Slp}
While implementing an algorithm, it makes sense to start to implement
sub-functions that are not dependant on other functions. And, why not
start implementing the first function in Algorithm \ref{alg:iter}?

Implementing the Taylor series expansion is pretty straight forward, as
we have a closed-form definition for the specific objective function that
we described in Chapter \ref{ch:qp}. The function prototype for the
Taylor series expansion reads:
\begin{verbatim}
void taylor(double* T, const double* x, const ClpModel& model);
\end{verbatim}
where \texttt{ClpModel} gives us access to both the linear and quadratic
objective term. The result of the Taylor-series expansion, i.e. the new
objective function, is put in \texttt{T}.

We find a closed-form definition of the step length $\alpha_k$ by letting
$m(\alpha) = f((1-\alpha) x_k + \alpha \hat{x}_k)$ and setting
$m^\prime(\alpha) = 0$ and solving for $\alpha$:
\[
\alpha_k = \frac{
                2x_k^T H x_k
                - 2\hat{x}_k^T H x_k
                + b^T x_k - b^T \hat{x}_k
                }{
                  2\hat{x}_k^T H \hat{x}_k
                - 4\hat{x}_k^T H x_k
                + 2x_k^T H x_k
                }
\]

