\section{About Clp}
Matrices in Clp are stored in a compact format using three vectors.
The first vector contains all the non-zero elements of the matrix.
The second vector contains the indices of the elements in the first vector.
The third vector contains an accumulated number of elements in each row/column.
The order of the elements depends whether the matrix is row-ordered or
column-ordered.
The indices represent the column/row position of the elements, and the
accumulated values represent the accumulated number of elements in the
row/columns depending on whether the matrix is row-ordered or column-ordered,
respectively.
To get a better understanding of how they are stored, consider the matrix
\[
\left[
\begin{array}{rrrrrrrr}
    3 & 1 & 0   & -2  & -1 & 0 & 0    & -1 \\
    0 & 2 & 1.1 & 0   & 0  & 0 & 0    & 0  \\
    0 & 0 & 1   & 0   & 0  & 1 & 0    & 0  \\
    0 & 0 & 0   & 2.8 & 0  & 0 & -1.2 & 0  \\
  5.6 & 0 & 0   & 0   & 1  & 0 & 0    & 1.9  

\end{array}
\right]
\]
being stored in row-ordered format, then the element vector would be
\[
\left[
\begin{array}{rrrrrrrrrrrrrr}
    3 & 1 & -2 & -1 & -1 & 2 & 1.1 & 1 & 1 & 2.8 & -1.2 & 5.6 & 1 & 1.9
\end{array}
\right]^T,
\]
the vector containing the column indices would be
\[
\left[
\begin{array}{rrrrrrrrrrrrrr}
    0 & 1 & 3 & 4 & 7 & 1 & 2 & 2 & 5 & 3 & 6 & 0 & 4 & 7
\end{array}
\right]^T,
\]
and the vector containing the vector starts (the accumulated indices) would be
\[
\left[
\begin{array}{rrrrrr}
    0 & 5 & 7 & 9 & 11 & 14
\end{array}
\right]^T.
\]
The \texttt{CoinPackedMatrix} class represents such a compact matrix.
To keep overhead at a minimum, it is important to implement the new solver
(from here on called Slp) using the same compact format.

Clp has a base class that holds data for both linear and quadratic models
called \texttt{ClpModel}.
The class itself knows nothing about the implementation of the algorithms, but
it contains all the data about a problem needed to apply an algorithm on that
data.
This is very handy for passing data throughout Slp without much overhead.
It also results in very neat function prototypes as only one parameter
is needed for passing lots of information. For instance, the function prototype
in Slp for the function that performs the line search described in Chapter
\ref{ch:slp} looks like this:
\begin{verbatim}
double lineSearch(double* p1, double* p2, ClpModel model);
\end{verbatim}
Figure \ref{fig:clpmodel} shows the class hierarchy for \texttt{ClpModel}.
The subclasses contain the implementation of the algorithms that their class
names describe. For instance, \texttt{ClpSimplexDual} contains the
implementation of the Dual Simplex Algorithm.
\begin{figure}
\centering
\begin{tikzpicture}[node distance=2.5cm]
    \tikzstyle{every state}=[rectangle,
                             minimum width=10em,
                             rounded corners,
                             fill=blue!30,
                             edge from parent,
                             text=black]
    \tikzset{node distance=1.7cm};

    \node[state] (a)                    {\texttt{ClpModel}};
    \node[state] (b) [below of = a]     {\texttt{ClpSimplex}};
    \node[state] (c) [right = 0.5cm of b] {\texttt{ClpInterior}};
    \node[state] (d) [below of = b]     {\texttt{ClpSimplexDual}};
    \node[state] (e) [right = 0.5cm of d] {\texttt{ClpSimplexPrimal}};

    \tikzset{edgestyle/.style={-, double=black}}
%    \tikzset{every node/.style={fill=white}}

    \path (a) edge [edgestyle] node {}  (b)
          (a) edge [edgestyle] node {}  (c)
          (b) edge [edgestyle] node {}  (d)
          (b) edge [edgestyle] node {}  (e);
\end{tikzpicture}
\label{fig:clpmodel}
\caption{Class hierarchy for \texttt{ClpModel}}
\end{figure}
