\section{Finding an LP solver}
The method described in Chapter \ref{ch:slp} relies on repetitively solving
linear programs. Before implementing such a method, it is important to choose
an appropriate LP solver. The solver must
\begin{inparaenum}[\itshape a\upshape)]
\item be released under a free license; and
\item it must allow library calls in C/C++
\end{inparaenum}

Among a list of about 50 solvers (most of them proprietary), the
following three matches the aforementioned criteria:
\begin{description}
\item[Clp] \hfill \\
Clp is short for Coin-or linear programming. It is a free linear programming
solver released under the Common Public License (CPL). It is primarily meant to
be used as a callable library. Its license allows other software to link to the
Clp library without requiring that that software is released under the same
license (permissive). \cite{clp}
\item[GLPK] \hfill \\
GLPK is short for GNU Linear Programming Kit. It is a free linear (integer)
programming software packaged released under the GNU General Public License
(GNU GPL). GLPK is organized in the form of a callable library. Its license
requires linking software to be released under the GNU GPL (reciprocal).
\cite{glpk}
\item[lp\_solve] \hfill \\
lp\_solve is a free linear (integer) programming solver released under the GNU 
Lesser General Public License (GNU LGPL). Its license is permissive like the
CPL. \cite{lpsolve}
\end{description}

In addition to those criteria, it is important that the solver is fast.
Moreover, it needs to be fast on problems that fit the description in
Section \ref{sec:problem}.

Testing the three solvers on linearized versions of the instances
\textit{small} and \textit{large}---presented in
Chapter \ref{sec:instances}---reveals the running times shown in Table
\ref{table:lpres}.
The running times are the number of seconds in CPU-time after 1000 runs.

\begin{table}[h!]
    \centering
    \caption{Running time in CPU-seconds used by each solver to solve 1000
             instances of each problem}
    \begin{tabular}{lrrr}
        Data Set       & Clp    & GLPK   & lp\_solve \\ \hline
        \textit{small} & 6.929  & 26.096 & 4.734 \\
        \textit{large} & 12.832 & 47.977 & 23.376
    \end{tabular}
    \label{table:lpres}
\end{table}

While lp\_solve is the fastest on a smaller LP, it doesn't scale as well as
with larger problems like Clp does. GoodTech need to solve pro

GoodTech need to solve problems with more than 200 variables, and
\textit{small} is almost hitting that lower limit, so a good result on
\textit{large} is prioritized over the other.
This means that Clp outperforms the other two solvers in this test.

Although this is not an extensive test, it is a pretty good indicator that Clp
will be the fastest solver on similar problems.
