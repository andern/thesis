The method described in Chapter \ref{ch:slp} relies on repetitively solving
linear programs. Before implementing such a method, it is important to choose
an appropriate LP solver. The solver should 
\begin{inparaenum}[\itshape a\upshape)]
\item be released under a free license; and
\item allow library calls in C/C++
\end{inparaenum}

Among a list of about 50 solvers (most of them proprietary), the
following three matches the aforementioned criteria:
\begin{description}
\item[Clp] \hfill \\
Clp is short for Coin-or linear programming. It is a free linear programming
solver released under the Common Public License (CPL). It is primarily meant to
be used as a callable library. Its license allows other software to link to the
Clp library without requiring that that software is released under the same
license (permissive). \cite{clp}
\item[GLPK] \hfill \\
GLPK is short for GNU Linear Programming Kit. It is a free linear (integer)
programming software packaged released under the GNU General Public License
(GNU GPL). GLPK is organized in the form of a callable library. Its license
requires linking software to be released under the GNU GPL (reciprocal).
\cite{glpk}
\item[lp\_solve] \hfill \\
lp\_solve is a free linear (integer) programming solver released under the GNU 
Lesser General Public License (GNU LGPL). Its license is permissive like the
CPL. \cite{lpsolve}
\end{description}
Incidentally, these three solvers are the only three open source LP solvers
suggested by the NEOS Optimization Guide~\cite{neos}.
In addition to the mentioned criteria, it is important that the solver is
fast.
Moreover, it needs to be fast on problems that fit the description in
Section \ref{sec:problem}.

Testing the three solvers on linearized versions of the instances
\textit{small} and \textit{large}---presented in
Section \ref{sec:instances}---reveals the running times shown in Table
\ref{table:lpres}.
The running times are the number of seconds in CPU-time after 1000 runs.

\begin{table}[ht!]
    \centering
    \caption{Running time in CPU-seconds used by each solver to solve each
             instance 1000 times.}
    \begin{tabular}{lrrr}
        Data Set       & Clp    & GLPK   & lp\_solve \\ \hline
        \textit{small} & 6.929  & 26.096 & 4.734 \\
        \textit{large} & 12.832 & 47.977 & 23.376
    \end{tabular}
    \label{table:lpres}
\end{table}
While lp\_solve is the fastest solver on a smaller LP problem, it doesn't
scale as well as with larger problems as Clp does.

GoodTech need to solve problems with more than 200 variables, and
\textit{small} is almost hitting that lower limit, so a good result on
\textit{large} is prioritized over the other.
This means that Clp outperforms the other two solvers in this test.

Although this is not an extensive test, it is a pretty good indicator that Clp
is the fastest solver on similar problems.
