\textbf{Experiment 1} is an experiment we perform in order to assess
how performance relates to specific tolerances. We test the different
implementations on the instance \textit{small} with varying tolerance.
For each specific tolerance, we solve $s(2, 238)$ subinstances, three times,
and record the lowest of the three running times.

\textbf{Experiment 2} is an experiment where we test our implementation of the
tree structure on the instances introduced in Section \ref{sec:instances}.
The experiment will work as an indicator to how well the implementations work
on real problems that Goodtech have faced, rather than randomly generated one.
We run each implementation on each of the instances three times, and record
the best out of the three runs.

\textbf{Experiment 3} is an experiment to indicate how fast our
our tree structure method is. We compare the
\texttt{cClp} implementation where we only solve subinstances
with distinct solutions, with the \texttt{nClp} implementation where
subinstances are solved regardless whether they are distinct or not.
Testing these implementations on problems of increasing size indicates how well
they scale, in addition to indicating their speed. 
As $n$ reaches high values, anything besides $b \leq 1$ is unrealistic, so we
let $b = 1$.

\textbf{Experiment 4} is an experiment similar to Experiment 1. We continue
to compare the tree structure method with naively solving all
subinstances regardless whether they are distinct or not. In this experiment, we
see how the two implementations scale as $b$ increases. Since the number
of subinstances increases rapidly as $b$ increases, this experiment is only
feasible for small values of $n$. We consider the experiment infeasible for
any $n > 50$, so we let $n = 50$.

All experiments are performed on a machine running 64 bit Gentoo Linux
with version 3.6.11 of the linux kernel. The machine has the following
relevant hardware specifications:
\begin{itemize}
    \item Intel(R) Core(TM) i7 CPU 930 @ 2.80GHz
    \item Corsair XMS3 DDR3 1600MHz 12GB CL9.
    \item Socket LGA 1366
\end{itemize}
The CPU has 4 cores and has the following multi-level cache specifications:
~\cite{intel}
\begin{itemize}
    \item 32KB L1 data cache and 32KB L1 instruction cache per core
    \item 256KB L2 data cache per core
    \item 8MB L3 data cache shared by all cores
\end{itemize}

All codes are compiled with GCC version 4.6.3 with \texttt{-O3}
optimization flag enabled.
