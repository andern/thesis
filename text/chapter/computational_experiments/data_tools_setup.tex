\textbf{Experiment 1} is an experiment to indicate how fast our
our implementation of the tree structure is. We compare implementations where
we only solve subinstances with distinct solutions, with implementations where
subinstances are solved regardless if they are distinct or not. Testing these
implementations on problems of increasing size indicates how well they scale,
in addition to indicating their speed.
In order to generate networks that are roughly as sparse as those
presented in Section \ref{sec:instances}, we let $m = \frac{7}{20}n$, while
increasing $n$. As $n$ reaches high values, anything besides $b = 1$ is
unrealistic, so we let $b = 1$.

For a given problem size $n$, we generate $10$ random instances. Then,
for each generated instance, we solve $s(1, n)$ subinstances.

For running experiments, we are using a machine running 64 bit Gentoo Linux
with version 3.6.11 of the linux kernel. The machine has the following
relevant hardware specifications:
\begin{itemize}
    \item Intel(R) Core(TM) i7 CPU 930 @ 2.80GHz
    \item Corsair XMS3 DDR3 1600MHz 12GB CL9
\end{itemize}
