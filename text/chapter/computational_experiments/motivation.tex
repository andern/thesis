Now that we are finished with implementation, an obvious question about the
implementation is how it performs. Normally, we want to compare the performance
of our implementation to the performance of some other implementations.
An obvious performance test is how \emph{fast} our implementation is, but also
its memory consumption.

We want to test our implementation on the specific problems described in
Section \ref{sec:instances}, to see how it performs on problems that follow
the exact characteristics that the methods were developed for.
Testing the performance of our implementation on specific problems will tell us
how it performs on those specific problems, but not how it performs in general.
While the methods developed in this thesis are specified for problems with
specific properties, the methods do not \emph{require} these properties.
To test the performance of our implementation in general, we need to test it
on random problems.
These random problems should be limited by some parameters, so that we can test
our implementation on random problems with those specific parameters.
Generating random problems of some specific size, we can see how our
implementation performs on specific problem sizes. Running the same
tests on problems with increasing size, we can see how well our implementation
\emph{scales}. In the next section, we describe experiments in detail.
