SLP was first referenced in \cite{slp61}, a paper by Griffith and Stewart in
1961~\cite{boggs1985numerical}.
They described a procedure they called Mathematical Approximation Programming
(MAP) that was in use at Shell Oil Company at the time. The paper is considered
a classic in the field of optimization

TODO: Belongs in an introduction chapter?

SLP is a differential technique which utilizes the linear programming algorithm
repetitively in such a way that the solution of a linear problem can converge
to the solution of the nonlinear problem~\cite{slp61}. The results above show
us that applying an SLP algorithm to a QP with the characteristics mentioned in
Section \ref{sec:problem}, with an initial guess 0, will give us an approximate
solution deviating less than $6.42\%$ from the optimal solution, after only one
iteration.
