One interesting property shared by several practical applications is that the
coefficients in the linear term of the objective function is of a much higher
magnitude than those in the quadratic term.
From this, one might think that the quadratic term is close to negligible.
By removing the quadratic term from the objective function, leaving only the
linear term, the QP transforms into an LP.
Linearizing the objective function using a first-order Taylor series expansion
at 0 would give us the same result.

To see this, consider a first-order (linear) Taylor series expansion of a
general function $f$ at a point $a$:
\[
T_a(x) = f(a) + \nabla f(a)^T(x-a)
\]
where $\nabla f(a)$ is the gradient of $f$ evaluated at $a$~\cite{apostol}.
A linear approximation of function (\ref{eq:obj}) becomes
\[
T_a(x) = a^THa + b^Ta + (2a^TH + b^T)(x - a) \\
\]
There are a number of terms that cancel each other out. Cancelling them,
the approximation becomes
\begin{equation}
\label{eq:texp}
T_a(x) = - a^THa + 2a^THx + b^Tx
\end{equation}

Now, if $a = 0$ in (\ref{eq:texp}) we are left with
\[
T_0(x) = b^Tx
\]

Using $g(x) = T_0(x) = b^Tx$, we formulate a new LP that we call $\mathcal{L}$:
\[
\min_{x \in \mathbb{R}^n} g(x)
\quad \textrm{subject to}
~
Ax = 0
~
\textrm{and}
~
l \le x \le u
\]
where $H$ is a positive semidefinite diagonal $n \times n$ matrix, $A$ is an
$m \times n$ oriented incidence matrix of some network, and $b, l, u$ and $x$ are vectors in
$\mathbb{R}^n$.
We let $\hat{x}$ denote some optimal solution of $\mathcal{L}$, and let $x^*$
denote some optimal solution of $\mathcal{Q}$.

By minimizing $\mathcal{L}$, we can get an idea about the quality of
$\mathcal{L}$'s solution with respect to $\mathcal{Q}$ by substituting
$\hat{x}$ into (\ref{eq:obj}) and checking how much the numerical value
deviates from the optimal objective value of $\mathcal{Q}$.
For any QP and its linearized LP, we denote this deviation by $\Delta$.
More specifically, using (\ref{eq:obj}):
\[
\Delta = \left|\frac{f(\hat{x}) - f(x^*)}{f(x^*)}\right|
\]

Figure \ref{fig:sparsobj} shows how $\Delta$ changes in respect to the sparsity
in the objective function, both in the linear and the quadratic terms, on a
specific network. The x- and y- axes represent the percentage of zeroes in the
diagonal elements of $H$ and the elements of $b$, respectively. The y-axis
stops at 75\% because $\Delta$ started reaching values above 1.

\begin{figure}[h!]
\begin{center}
    Hidden. TODO: Show
%    \begin{tikzpicture}
    \begin{axis}[
        width=\textwidth*0.80,
%        grid=major,
        view={245}{30},
%        xmax=99,
%        ymax=99,
%        zmax=1,
%        restrict z to domain=0:1,
        xlabel=x,
        ylabel=y,
        zlabel=$100\Delta$]

        \addplot3[surf,mesh/rows=20,mesh/cols=20] table[x=b,y=h,z=dev]
                                                   {include/data/delta3d.dat};


%        \addplot3[
%            surf,
%            mesh/rows=101,
%            mesh/cols=100,
%        ] 
%        table[x=b,y=h,z=dev] {include/data/delta3d.dat};
\end{axis}
\end{tikzpicture}

\end{center}
\caption{Density in the objective function}
\label{fig:sparsobj}
\end{figure}

$\mathcal{L}$ was always solved more than 85\% faster than $\mathcal{Q}$.
(TODO: Test against CPLEX)
These results give us an incentive to look into solution methods that are based
upon linear programming, such as Successive Linear Programming (SLP).
