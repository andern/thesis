Consider the quadratic program $\mathcal{Q}$:
\[
\begin{array}{lcrcrcr}
\textrm{Minimize}           & &   (x-1)^2 &+&   (y-1)^2 & - &  2 \\
\textrm{subject to}         & &         x &+&         y &\leq& 3 \\
\textrm{and}                & &         x &-&         y &\leq& 1 \\
\textrm{and}                & &         x &+&       3 y &\leq& 4 \\
\textrm{and}                & &           & &       x,y &\geq& 0
\end{array}
\]
and let $x^*$ denote one of its optimal points and let $f$ denote its objective
function.

We will solve $\mathcal{Q}$ using the method described in the previous
section. While we solve it, we will look at a visual representation of the
method, along with the linear program in standard form.

Let $x_0 = (0,0)$ such that $T_0 = -2x - 2y$. The linear program
$\mathcal{L}_0$ in standard form reads:
\[
\begin{array}{lcrcrl}
    \textrm{Maximize}   & & 2 x &+& 2 y \\
    \textrm{subject to} & &   x &+&   y & \leq 3 \\
    \textrm{and}        & &   x &-&   y & \leq 1 \\
    \textrm{and}        &-&   x &+& 3 y & \leq 4 \\
    \textrm{and}        & &   x &,&   y & \geq 0
\end{array}
\]
Note that the objective function has been negated because we are maximizing
instead of minimizing.
Figure \ref{fig:lp1} shows a visual representation of $\mathcal{L}_0$ and
$\mathcal{Q}$.
Note that the feasible region of $\mathcal{L}_k$ and
$\mathcal{Q}$ are equal for all $k$, as the constraints never change during the
iterations of the algorithm. Also note that $\mathcal{L}_0$ has multiple
optimal solutions, so $\hat{x}_0$ may differ between implementations.

\begin{figure}[ht!]
    \centering
    \begin{tikzpicture}
    % grid and axes
    \draw[scale=1.5,->,name path=xaxis] (-0.2,0) -- (2.2,0) node[right] {$x$};
    \draw[scale=1.5,->,name path=yaxis] (0,-0.2) -- (0,2.2) node[above] {$y$};

    % draw lines
    \draw[scale=1.5,name path=line1, domain=0.6:2.5] plot(\x,{-\x + 3})
                                                    node[above right=0.2cm] {};
    \draw[scale=1.5,name path=line2, domain=-0.2:1.6] plot(\x,{0.33*\x + 1.33})
                                                    node[above right=0.2cm] {};
    \draw[scale=1.5,name path=line3, domain=0.7:2.3] plot(\x,{\x - 1})
                                                    node[above right=0.2cm] {};

    % calculate intersection points
    \node[name intersections={of=line1 and line2}] (a) at (intersection-1) {};
    \node[name intersections={of=line1 and line3}] (b) at (intersection-1) {};
    \node[name intersections={of=xaxis and line3}] (c) at (intersection-1) {};
    \node[name intersections={of=xaxis and yaxis}] (d) at (intersection-1) {};
    \node[name intersections={of=line2 and yaxis}] (e) at (intersection-1) {};

    % write the coordinates of the corners.
    \path let \p0 = (a) in node [above right=0.0cm of a] {($1.25$, $1.75$)};
    \path let \p0 = (b) in node [right=0.1cm of b] {($2$, $1$)};
    \path let \p0 = (c) in node [below right=-0.35cm of c] {($1$, $0$)};
    %\path let \p0 = (d) in node [left=0.0cm of d] {($0$, $0$)};
    \path let \p0 = (e) in node [left=0.0cm of e] {($0$, $1.\overline{3}$)};

    \node at (1.5, 1.5) (qopt) {};
    \node at (3, 1.5) (lopt) {};
    \node at (-0.75, 1.125) (qoptdesc) {$x^*$};
    \node at (-1.5, 0.375) (x0desc) {$x_0$};
    \node[text=red] at (3.75, 2.625) (loptdesc) {$\hat{x}_0$};
    \draw[scale=1.5,->] (qoptdesc) .. controls ([xshift=1cm] qoptdesc) and ([xshift=-1cm]
                                                               qopt) .. (qopt);
    \draw[scale=1.5,->] (loptdesc) .. controls ([xshift=-1cm] loptdesc) and ([yshift=1cm]
                                                               lopt) .. (lopt);
    \draw[scale=1.5,->] (x0desc) .. controls ([xshift=1cm] x0desc) and ([xshift=-1cm] d)
                                                                     .. (d);

    \draw[scale=1.5,fill] (qopt) circle [radius=0.02];

    % draw the big polygin
    \fill[very thick,fill=blue,fill opacity=0.1] (a.center) -- (b.center) --
                                                 (c.center) -- (d.center) --
                                                 (e.center) -- cycle;

    \draw[scale=1.5,thick, red, name path=lobj, domain=-0.4:0.4] plot(\x, {-\x}) {};

    % draw the quadratic objective function
    \draw[scale=1.5,thin, dashed] plot[id=qobj1, raw gnuplot] function {
        f(x,y) = x**2 + y**2 - 2*x - 2*y + 1.75;
        set xrange[-1:2];
        set yrange[-2:2];
        set view 0,0;
        set isosamples 1000,1000;
        set size square;
        set cont base;
        set cntrparam levels incre 0,0.1,0;
        unset surface;
        splot f(x,y);
    };
    \draw[scale=1.5,thin, dashed] plot[id=qobj2, raw gnuplot] function {
        f(x,y) = x**2 + y**2 - 2*x - 2*y + 1.9;
        set xrange[-1:2];
        set yrange[-2:2];
        set view 0,0;
        set isosamples 1000,1000;
        set size square;
        set cont base;
        set cntrparam levels incre 0,0.1,0;
        unset surface;
        splot f(x,y);
    };
    \draw[scale=1.5,thin, dashed] plot[id=qobj3, raw gnuplot] function {
        f(x,y) = x**2 + y**2 - 2*x - 2*y + 1.98;
        set xrange[-1:2];
        set yrange[-2:2];
        set view 0,0;
        set isosamples 1000,1000;
        set size square;
        set cont base;
        set cntrparam levels incre 0,0.1,0;
        unset surface;
        splot f(x,y);
    };
\end{tikzpicture}

    \caption{A visual representation of $\mathcal{L}_0$ and $\mathcal{Q}$.
             The figure depicts an initial state
             of Slp, where you can see the initial guess $x_0$. It also
             depicts the unconstrained minimum $x^u$, which in this case
             happens to be the same as $x^*$, the optimal solution to
             $\mathcal{Q}$. An optimal solution $\hat{x}_0$ to $\mathcal{L}_0$
             is also depicted. The striped circles are the contours of $f$.}
    \label{fig:lp1}
\end{figure}

We perform a line search between the points $x_0$ and $\hat{x}_0$.
We find a function $m$ of $\alpha$ that range all points collinear with $x_0$
and $\hat{x}_0$:
\[
f((1-\alpha) x_0 + \alpha \hat{x}_0) = 5\alpha^2 - 6\alpha = m(\alpha).
\]
To find the minimum of this one-dimensional convex parabola we set
$m^\prime(\alpha) = 0$ and solve for $\alpha$ to achieve $\alpha = 0.6$
(see Figure \ref{fig:steplength}).
\begin{figure}[ht!]
    \centering
    \begin{tikzpicture}
    % grid and axes
    \draw[scale=1.5,->,name path=xaxis] (-0.2,0) -- (2.2,0) node[right] {$x$};
    \draw[scale=1.5,->,name path=yaxis] (0,-0.2) -- (0,2.2) node[above] {$y$};

    % draw lines
    \draw[scale=1.5,name path=line1, domain=0.6:2.5] plot(\x,{-\x + 3});
    \draw[scale=1.5,name path=line2, domain=-0.2:1.6] plot(\x,{0.33*\x + 1.33});
    \draw[scale=1.5,name path=line3, domain=0.7:2.3] plot(\x,{\x - 1});

    % calculate intersection points
    \node[name intersections={of=line1 and line2}] (a) at (intersection-1) {};
    \node[name intersections={of=line1 and line3}] (b) at (intersection-1) {};
    \node[name intersections={of=xaxis and line3}] (c) at (intersection-1) {};
    \node[name intersections={of=xaxis and yaxis}] (d) at (intersection-1) {};
    \node[name intersections={of=line2 and yaxis}] (e) at (intersection-1) {};

    % draw the big polygon
    \fill[very thick,fill=tleblue] (a.center) -- (b.center)
                                              -- (c.center) -- (d.center)
                                              -- (e.center) -- cycle;

    \draw[scale=1.5,name path=linesearch, domain=-0.1:2.1] plot(\x,{0.5*\x});

    \node at (1.2*1.5,0.6*1.5) (x1) {};
    \node at (1.95*1.5, 0.35*1.5) (x1desc) {$x_1$};
    \draw[scale=1.5,->] (x1desc) .. controls ([yshift=0.5cm] x1desc) and ([xshift=0.5cm]
                                                                   x1) .. (x1);


    \node[red] at (0.7*1.5, 0.1*1.5) (alphadesc) {$\alpha$};
    \node at (1.2*1.5, 0.6*1.5) (alpha) {};
    \node at (1*1.5, 1*1.5) (qopt) {};
    \node at (2*1.5, 1*1.5) (lopt) {};
    \node at (-0.5*1.5, 0.75*1.5) (qoptdesc) {$x^*$};
    \node at (2.5*1.5, 1.75*1.5) (loptdesc) {$\hat{x}_0$};
    \node at (-1*1.5, 0.25*1.5) (x0desc) {$x_0$};
    \draw[scale=1.5,->] (qoptdesc) .. controls ([xshift=1cm] qoptdesc) and
                                                ([xshift=-1cm] qopt) .. (qopt);
    \draw[scale=1.5,->] (loptdesc) .. controls ([xshift=-1cm] loptdesc) and
                                                 ([yshift=1cm] lopt) .. (lopt);
    \draw[scale=1.5,->] (x0desc) .. controls ([xshift=1cm] x0desc) and
                                                      ([xshift=-1cm] d) .. (d);

    \draw[scale=1.5,fill] (qopt) circle [radius=0.02];
    \draw[scale=1.5,fill] (alpha) circle [radius=0.02];

    \draw [red, decorate,decoration={brace,mirror}] (d) -- (alpha);

    % draw the quadratic objective function
    \draw[scale=1.5,thin, dashed] plot[id=qobjalpha, raw gnuplot] function {
        f(x,y) = x**2 + y**2 - 2*x - 2*y + 1.8;
        set xrange[-1:2];
        set yrange[-2:2];
        set view 0,0;
        set isosamples 1000,1000;
        set size square;
        set cont base;
        set cntrparam levels incre 0,0.1,0;
        unset surface;
        splot f(x,y);
    };
%    \draw[thin, dashed] plot[id=qobj2, raw gnuplot] function {
%        f(x,y) = x**2 + y**2 - 2*x - 2*y + 1.9;
%        set xrange[-1:2];
%        set yrange[-2:2];
%        set view 0,0;
%        set isosamples 1000,1000;
%        set size square;
%        set cont base;
%        set cntrparam levels incre 0,0.1,0;
%        unset surface;
%        splot f(x,y);
%    };
%    \draw[thin, dashed] plot[id=qobj3, raw gnuplot] function {
%        f(x,y) = x**2 + y**2 - 2*x - 2*y + 1.98;
%        set xrange[-1:2];
%        set yrange[-2:2];
%        set view 0,0;
%        set isosamples 1000,1000;
%        set size square;
%        set cont base;
%        set cntrparam levels incre 0,0.1,0;
%        unset surface;
%        splot f(x,y);
%    };
\end{tikzpicture}

    \caption{Line search between $x_0$ and $\hat{x}_0$. The optimal point
             between $x_0$ and $\hat{x}_0$ becomes $x_1$, our starting point
             for the next iteration.}
    \label{fig:steplength}
\end{figure}

Note that $\alpha \in [0, 1]$ because the unconstrained minimum of $f$ is
inside the feasible region.
Now let
\[
x_1 = 0.4x_0 + 0.6\hat{x}_0 = (1.2, 0.6),
\]
and let us apply Equation (\ref{eq:texp}) to calculate a Taylor series
expansion at $x_1$:
\[
    T_1 = 0.4x - 0.8y - 1.8.
\]
\begin{comment}
\[
\begin{array}{rcl}
T_1 &=& -\left[
         \begin{array}{cc}
            1.2 & 0.6
         \end{array}
         \right]

         \left[
         \begin{array}{cc}
            1 & 0 \\ 0 & 1
         \end{array}
         \right]
         
         \left[
         \begin{array}{c}
             1.2 \\ [5pt] 0.6
         \end{array}
         \right]
         + 2
         \left[
         \begin{array}{cc}
            1.2 & 0.6
         \end{array}
         \right]
         
         \left[
         \begin{array}{cc}
            1 & 0 \\ 0 & 1
         \end{array}
         \right]
         
         \left[
         \begin{array}{c}
            x \\ y
         \end{array}
         \right] \\
         
     & & +
         \left[
         \begin{array}{cc}
            -2 & -2
         \end{array}
         \right]
         
         \left[
         \begin{array}{c}
            x \\ y
         \end{array}
         \right] \\ [15pt]
    &=& \displaystyle 0.4x - 0.8y - 1.8
\end{array}
\]
\end{comment}

Now, we minimize $T_1$ subject to the original constraints and call the LP for
$\mathcal{L}_1$.
The linear program of $\mathcal{L}_1$ in standard form, reads:
\[
\begin{array}{lcrcrl}
    \textrm{Maximize}   &-& 0.4 x &+& 0.8 y \\
    \textrm{subject to} & &     x &+&     y & \leq 3 \\
    \textrm{and}        & &     x &-&     y & \leq 1 \\
    \textrm{and}        &-&     x &+&   3 y & \leq 4 \\
    \textrm{and}        & &     x &,&     y & \geq 0
\end{array}
\]
We solve $\mathcal{L}_1$ and achieve an optimal solution $\hat{x}_1$
(see Figure \ref{fig:lp2}).

\begin{figure}[ht!]
\centering
\begin{figure}[htbp]
\scalebox{1}{
\begin{minipage}{0.4\linewidth}
\centering
\begin{tikzpicture}
    % grid and axes
    \draw[->,name path=xaxis] (-0.2,0) -- (2.2,0) node[right] {$x$};
    \draw[->,name path=yaxis] (0,-0.2) -- (0,2.2) node[above] {$y$};

    % draw lines
    \draw[name path=line1, domain=0.6:2.5] plot(\x,{-\x + 3})
                                                    node[above right=0.2cm] {};
    \draw[name path=line2, domain=-0.2:1.6] plot(\x,{0.33*\x + 1.33})
                                                    node[above right=0.2cm] {};
    \draw[name path=line3, domain=0.7:2.3] plot(\x,{\x - 1})
                                                    node[above right=0.2cm] {};

    % calculate intersection points
    \node[name intersections={of=line1 and line2}] (a) at (intersection-1) {};
    \node[name intersections={of=line1 and line3}] (b) at (intersection-1) {};
    \node[name intersections={of=xaxis and line3}] (c) at (intersection-1) {};
    \node[name intersections={of=xaxis and yaxis}] (d) at (intersection-1) {};
    \node[name intersections={of=line2 and yaxis}] (e) at (intersection-1) {};

    % write the coordinates of the corners.
    \path let \p0 = (a) in node [above right=0.0cm of a] {($1.25$, $1.75$)};
    \path let \p0 = (b) in node [right=0.1cm of b] {\printpoint{\x0}{\y0}};
    \path let \p0 = (c) in node [below right=-0.35cm of c]
                                                       {\printpoint{\x0}{\y0}};
    \path let \p0 = (d) in node [left=0.0cm of d] {\printpoint{\x0}{\y0}};
    \path let \p0 = (e) in node [left=0.0cm of e] {($0$, $1.\overline{3}$)};
    %\path let \p0 = (e) in node [left=0.0cm of e] {\printpoint{\x0}{\y0}};

    \node at (1, 1) (qopt) {};
    \node at (0, 1.33) (lopt) {};
    \node at (1.2,0.6) (x1) {};
    \node[red] at (-0.5, 1.95) (loptdesc) {$\hat{x}_1$};
    \node at (-0.5, 0.75) (qoptdesc) {$x^*$};
    \node at (1.95, 0.35) (x1desc) {$x_1$};
    \draw[->] (qoptdesc) .. controls ([xshift=1cm] qoptdesc)
                                        and ([xshift=-1cm] qopt) .. (qopt);
    \draw[->] (loptdesc) .. controls ([xshift=0.5cm] loptdesc)
                                        and ([xshift=-0.5cm] lopt) .. (lopt);
    \draw[->] (x1desc) .. controls ([yshift=0.5cm] x1desc)
                                        and ([xshift=0.5cm] x1) .. (x1);

    \draw[fill] (qopt) circle [radius=0.02];

    \draw[fill] (x1) circle [radius=0.02];

    % draw the big polygin
    \fill[very thick,fill=blue,fill opacity=0.3] (a.center) -- (b.center)
                                              -- (c.center) -- (d.center)
                                              -- (e.center) -- cycle;

    \draw[thick, red, name path=lobj, domain=0.8:1.6] plot(\x, {0.5*\x}) {};

    % draw the quadratic objective function
    \draw[thin, dashed] plot[id=qobj1, raw gnuplot] function {
        f(x,y) = x**2 + y**2 - 2*x - 2*y + 1.75;
        set xrange[-1:2];
        set yrange[-2:2];
        set view 0,0;
        set isosamples 1000,1000;
        set size square;
        set cont base;
        set cntrparam levels incre 0,0.1,0;
        unset surface;
        splot f(x,y);
    };
    \draw[thin, dashed] plot[id=qobj2, raw gnuplot] function {
        f(x,y) = x**2 + y**2 - 2*x - 2*y + 1.9;
        set xrange[-1:2];
        set yrange[-2:2];
        set view 0,0;
        set isosamples 1000,1000;
        set size square;
        set cont base;
        set cntrparam levels incre 0,0.1,0;
        unset surface;
        splot f(x,y);
    };
    \draw[thin, dashed] plot[id=qobj3, raw gnuplot] function {
        f(x,y) = x**2 + y**2 - 2*x - 2*y + 1.98;
        set xrange[-1:2];
        set yrange[-2:2];
        set view 0,0;
        set isosamples 1000,1000;
        set size square;
        set cont base;
        set cntrparam levels incre 0,0.1,0;
        unset surface;
        splot f(x,y);
    };
\end{tikzpicture}
\end{minipage}
}
\scalebox{0.9}{
\begin{minipage}{0.6\linewidth}
\centering
\[
\begin{array}{lcrcrl}
    \textrm{Maximize}   &-& 0.4 x &+& 0.8 y \\
    \textrm{subject to} & &     x &+&     y & \leq 3 \\
    \textrm{and}        & &     x &-&     y & \leq 1 \\
    \textrm{and}        &-&     x &+&   3 y & \leq 4 \\
    \textrm{and}        & &     x &,&     y & \geq 0
\end{array}
\]
\end{minipage}
}
\caption{$\mathcal{L}_1$}
\label{fig:lp2}
\end{figure}

\caption{A visual representation of $\mathcal{L}_1$ and $\mathcal{Q}$.
         Note that $x_1$ now has a red line going through it. This is
         the objective function of $\mathcal{L}_1$.}
\label{fig:lp2}
\end{figure}

Again we perform a line search, but this time between the points $x_1$ and
%$\hat{x}_1$ and find that $\alpha = 0.73$, $x_2 = (0.88, 0.8)$ and
$\hat{x}_1$ and find that $\alpha = 0.27$, $x_2 = (0.88, 0.8)$ and
$T_2 = -0.25x -0.4y -0.4$.

Minimize $T_2$ subject to the original constraints and call the LP for
$\mathcal{L}_2$.
The linear program of $\mathcal{L}_2$ in standard form, reads:
\[
\begin{array}{lcrcrl}
    \textrm{Maximize}   & & 0.25 x &+& 0.4 y & \\
    \textrm{subject to} & &      x &+&     y & \leq 3 \\
    \textrm{and}        & &      x &-&     y & \leq 1 \\
    \textrm{and}        &-&      x &+&   3 y & \leq 4 \\
    \textrm{and}        & &      x &,&     y & \geq 0
\end{array}
\]
We solve $\mathcal{L}_2$ and achieve an optimal solution $\hat{x}_2
= (1.25, 1.75)$ (see Figure \ref{fig:lp3}).
\begin{figure}[ht!]
\centering
\begin{tikzpicture}
    % grid and axes
    \draw[->,name path=xaxis] (-0.2,0) -- (2.2,0) node[right] {$x$};
    \draw[->,name path=yaxis] (0,-0.2) -- (0,2.2) node[above] {$y$};

    % draw lines
    \draw[name path=line1, domain=0.6:2.5] plot(\x,{-\x + 3})
                                                    node[above right=0.2cm] {};
    \draw[name path=line2, domain=-0.2:1.6] plot(\x,{0.33*\x + 1.33})
                                                    node[above right=0.2cm] {};
    \draw[name path=line3, domain=0.7:2.3] plot(\x,{\x - 1})
                                                    node[above right=0.2cm] {};

    % calculate intersection points
    \node[name intersections={of=line1 and line2}] (a) at (intersection-1) {};
    \node[name intersections={of=line1 and line3}] (b) at (intersection-1) {};
    \node[name intersections={of=xaxis and line3}] (c) at (intersection-1) {};
    \node[name intersections={of=xaxis and yaxis}] (d) at (intersection-1) {};
    \node[name intersections={of=line2 and yaxis}] (e) at (intersection-1) {};

    % write the coordinates of the corners.
    \path let \p0 = (a) in node [above right=0.0cm of a] {($1.25$, $1.75$)};
    \path let \p0 = (b) in node [right=0.1cm of b] {\printpoint{\x0}{\y0}};
    \path let \p0 = (c) in node [below right=-0.35cm of c]
                                                       {\printpoint{\x0}{\y0}};
    \path let \p0 = (d) in node [left=0.0cm of d] {\printpoint{\x0}{\y0}};
    \path let \p0 = (e) in node [left=0.0cm of e] {($0$, $1.\overline{3}$)};
    %\path let \p0 = (e) in node [left=0.0cm of e] {\printpoint{\x0}{\y0}};

    \node at (1, 1) (qopt) {};
    \node at (1.25, 1.75) (lopt) {};
    \node at (0.88,0.8) (x1) {};
    \node[red] at (0.4, 1.95) (loptdesc) {$\hat{x}_2$};
    \node at (-0.5, 0.75) (qoptdesc) {$x^*$};
    \node at (1.95, 0.35) (x1desc) {$x_2$};
    \draw[->] (qoptdesc) .. controls ([xshift=1cm] qoptdesc)
                                        and ([xshift=-1cm] qopt) .. (qopt);
    \draw[->] (loptdesc) .. controls ([xshift=0.5cm] loptdesc)
                                        and ([xshift=-0.5cm] lopt) .. (lopt);
    \draw[->] (x1desc) .. controls ([yshift=0.5cm] x1desc)
                                        and ([xshift=0.5cm] x1) .. (x1);
    
    \draw[fill] (qopt) circle [radius=0.02];

    \draw[fill] (x1) circle [radius=0.02];

    % draw the big polygin
    \fill[very thick,fill=blue,fill opacity=0.3] (a.center) -- (b.center)
                                              -- (c.center) -- (d.center)
                                              -- (e.center) -- cycle;

    \draw[thick, red, name path=lobj, domain=0.48:1.28] plot(\x, {-0.625*\x + 1.35})
                                                                            {};

    % draw the quadratic objective function
    \draw[thin, dashed] plot[id=qobj1, raw gnuplot] function {
        f(x,y) = x**2 + y**2 - 2*x - 2*y + 1.75;
        set xrange[-1:2];
        set yrange[-2:2];
        set view 0,0;
        set isosamples 1000,1000;
        set size square;
        set cont base;
        set cntrparam levels incre 0,0.1,0;
        unset surface;
        splot f(x,y);
    };
    \draw[thin, dashed] plot[id=qobj2, raw gnuplot] function {
        f(x,y) = x**2 + y**2 - 2*x - 2*y + 1.9;
        set xrange[-1:2];
        set yrange[-2:2];
        set view 0,0;
        set isosamples 1000,1000;
        set size square;
        set cont base;
        set cntrparam levels incre 0,0.1,0;
        unset surface;
        splot f(x,y);
    };
    \draw[thin, dashed] plot[id=qobj3, raw gnuplot] function {
        f(x,y) = x**2 + y**2 - 2*x - 2*y + 1.98;
        set xrange[-1:2];
        set yrange[-2:2];
        set view 0,0;
        set isosamples 1000,1000;
        set size square;
        set cont base;
        set cntrparam levels incre 0,0.1,0;
        unset surface;
        splot f(x,y);
    };
\end{tikzpicture}

\caption{A visual representation of $\mathcal{L}_2$ and $\mathcal{Q}$.}
\label{fig:lp3}
\end{figure}

Performing line search between $x_2$ and $\hat{x}_2$ we get that
$\alpha = 0.77$ and that $x_3 = (0.96, 1.02)$.


Assume $\epsilon = 0.5$, which imples that
$\frac{f(x_2) - f(x_3)}{|f(x_2)|} \leq \epsilon$, so we stop and return $x_3$
as our solution.
