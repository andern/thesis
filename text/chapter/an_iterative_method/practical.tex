In practical applications, immediately after a problem is solved, another very
similar problem may need to be solved.
Often this problem has minor changes in the constraints to the problem before
it.
If many problems are solved consecutively, it is important with as little
``downtime'' as possible between each problem.

We have that
\begin{align}
    x_{\mathcal{B}}^* &= B^{-1}b \label{eq:vanderbei1} \\
    z_{\mathcal{N}}^* &= (B^{-1}N)^Tc_{\mathcal{B}}-c_{\mathcal{N}}
                                                      \label{eq:vanderbei2} \\
    \zeta^* &= c_{\mathcal{B}}^TB^{-1}b \label{eq:vanderbei3}
\end{align}
where $\mathcal{B}$ denotes the collection of indices corresponding to the
basic variables and $\mathcal{N}$ denotes the indices corresponding to the
nonbasic variables.
Vector $x^*$ is some \emph{primal} optimal solution and $z^*$ is some
\emph{dual} optimal solution.
The matrix $B$ denotes the matrix corresponding to the basic rows and $N$
denotes the matrix corresponding to the non-basic columns.
We let $b$ denote the vector corresponding to the right hand side values of the
constraints.
The value $\zeta^*$ denotes the optimal objective value and $c$ corresponds to
the coefficients of the objective function~\cite{vanderbei}.

In (\ref{eq:vanderbei2}) we see that a change in $b$ does not require us to
recompute $z_{\mathcal{N}}^*$.
This means that the dictionary is guaranteed to stay dually feasible when $b$
changes.
%To illustrate this, let us take a look at the optimal dictionary of
%$\mathcal{L}_2$ from Section \ref{sec:example}:
%\[
%    \begin{array}{rcrcrcr}
%        \zeta &=& 1.01 &-& 0.10 w_3 &-& 0.10 w_1 \\ \hline
%            y &=& 1.75 &-& 0.25 w_3 &-& 0.25 w_1 \\
%            x &=& 1.25 &+& 0.25 w_3 &-& 0.75 w_1 \\
%          w_2 &=& 1.50 &-& 0.50 w_3 &+& 0.50 w_1
%    \end{array}
%\]
%Imagine that the third constraint changes from $-x + 3y \leq 4$ to
To illustrate this, imagine that the third constraint in $\mathcal{L}_2$ from
Section \ref{sec:example} changes from $-x + 3y \leq 4$ to
$-x + 3y \leq 0$. The new linear program then reads:
\[
\begin{array}{lcrcrl}
    \textrm{Maximize}   & & 0.25 x &+& 0.4 y & \\
    \textrm{subject to} & &      x &+&     y & \leq 3 \\
    \textrm{and}        & &      x &-&     y & \leq 1 \\
    \textrm{and}        &-&      x &+&   3 y & \leq 0 \\
    \textrm{and}        & &      x &,&     y & \geq 0
\end{array}
\]
Figure \ref{fig:prac1} illustrates the new feasible region.
Readers are encouraged to compare Figure \ref{fig:lp3} and Figure
\ref{fig:prac1} to note how the feasible region has changed and made the
unconstrained minimum infeasible.

\begin{figure}[ht!]
\centering
\begin{tikzpicture}
    % grid and axes
    \draw[->,name path=xaxis] (-0.2,0) -- (2.2,0) node[right] {$x$};
    \draw[->,name path=yaxis] (0,-0.2) -- (0,2.2) node[above] {$y$};

    % draw lines
    \draw[name path=line2, domain=-0.1:1.7] plot(\x,{0.33*\x})
                                                    node[above right=0.2cm] {};
    \draw[name path=line3, domain=0.9:1.7] plot(\x,{\x - 1})
                                                    node[above right=0.2cm] {};

    % calculate intersection points
    \node[name intersections={of=xaxis and yaxis}] (a) at (intersection-1) {};
    \node[name intersections={of=line2 and line3}] (b) at (intersection-1) {};
    \node[name intersections={of=xaxis and line3}] (c) at (intersection-1) {};

    % write the coordinates of the corners.
    \path let \p0 = (a) in node [left=-0.0cm of a] {\printpoint{\x0}{\y0}};
    \path let \p0 = (b) in node [right=0.15cm of b] {(1.5,0.5)};
    \path let \p0 = (c) in node [below=-0.2cm of c] {\printpoint{\x0}{\y0}};

    \node at (1, 1) (xu) {};
    \node at (0, 0) (xhat) {};
    \node at (1.2, 0.4) (xstar) {};
    \node at (1.5,0.5) (x0) {};
    \node at (1.95, 1.3) (xudesc) {$x^u$};
    \node at (0.4, 0.4) (xstardesc) {$x^*$};
    \node[red] at (-0.4, -1.0) (xhatdesc) {$\hat{x}_0$};
    \node at (1.8, -0.3) (x0desc) {$x_0$};
    \draw[->] (xudesc) .. controls ([xshift=-1cm] xudesc)
                                        and ([xshift=1cm] xu) .. (xu);
    \draw[->] (x0desc) .. controls ([yshift=0.1cm] x0desc)
                                        and ([xshift=0.5cm] x0) .. (x0);
%    \draw[->] (loptdesc) .. controls ([xshift=0.5cm] loptdesc)
%                                        and ([xshift=-0.5cm] lopt) .. (lopt);
    \draw[->] (xhatdesc) .. controls ([yshift=1cm] xhatdesc)
                                        and ([yshift=-1cm] xhat) .. (xhat);
    \draw[->] (xstardesc) .. controls ([xshift=1cm] xstardesc)
                                        and ([xshift=-1cm] xstar) .. (xstar);
    
    \draw[fill] (xu) circle [radius=0.02];

    \draw[fill] (xstar) circle [radius=0.02];

    % draw the big polygin
%    \fill[very thick,fill=blue,fill opacity=0.3] (a.center) -- (b.center)
%                                              -- (c.center) -- (d.center)
%                                              -- (e.center) -- cycle;

    \fill[very thick,fill=blue,fill opacity=0.3] (a.center) -- (b.center)
                                              -- (c.center) -- cycle;

%    \draw[thick, red, name path=lobj, domain=0.48:1.28] plot(\x, {-0.625*\x + 1.35})
%                                                                            {};

    % draw the quadratic objective function
    \draw[thin, dashed] plot[id=qobj1, raw gnuplot] function {
        f(x,y) = x**2 + y**2 - 2*x - 2*y + 1.75;
        set xrange[-1:2];
        set yrange[-2:2];
        set view 0,0;
        set isosamples 1000,1000;
        set size square;
        set cont base;
        set cntrparam levels incre 0,0.1,0;
        unset surface;
        splot f(x,y);
    };
    \draw[thin, dashed] plot[id=qobj2, raw gnuplot] function {
        f(x,y) = x**2 + y**2 - 2*x - 2*y + 1.9;
        set xrange[-1:2];
        set yrange[-2:2];
        set view 0,0;
        set isosamples 1000,1000;
        set size square;
        set cont base;
        set cntrparam levels incre 0,0.1,0;
        unset surface;
        splot f(x,y);
    };
    \draw[thin, dashed] plot[id=qobj3, raw gnuplot] function {
        f(x,y) = x**2 + y**2 - 2*x - 2*y + 1.98;
        set xrange[-1:2];
        set yrange[-2:2];
        set view 0,0;
        set isosamples 1000,1000;
        set size square;
        set cont base;
        set cntrparam levels incre 0,0.1,0;
        unset surface;
        splot f(x,y);
    };
\end{tikzpicture}

\caption{A visual representation of the linearized version of the new QP.}
\label{fig:prac1}
\end{figure}

Although we do not need to recompute $z_{\mathcal{N}}^*$, we do however,
need to recompute $x_{\mathcal{B}}^*$ and $\zeta^*$.
From the optimal dictionary of $\mathcal{L}_2$, we have that
\[
B = 
\left[
\begin{array}{rrr}
     1 &  1 & 0 \\
    -1 &  1 & 1 \\
     3 & -1 & 0
\end{array}
\right],
%\qquad
%c_{\mathcal{B}} =
%\left[
%\begin{array}{ccc}
%    0.4 & 0.25 & 0
%\end{array}
%\right],
\]
and $b$ has now changed to
\[
b = \left[
\begin{array}{rrr}
    3 & 1 & 0
\end{array}
\right]^T.
\]
Using (\ref{eq:vanderbei1}) and (\ref{eq:vanderbei3}) we can calculate a new
basic solution
\[
x_{\mathcal{B}}^* =
\left[
\begin{array}{ccc}
    0.75 & 2.25 & -0.5
\end{array}
\right],
\]
and update our dictionary to become
\[
    \begin{array}{rcrcrcr}
        \zeta &=& 1.01 &-& 0.10 w_3 &-& 0.10 w_1 \\ \hline
            y &=& 0.75 &-& 0.25 w_3 &-& 0.25 w_1 \\
            x &=& 2.25 &+& 0.25 w_3 &-& 0.75 w_1 \\
          w_2 &=&-0.50 &-& 0.50 w_3 &+& 0.50 w_1
    \end{array}
\]

Note that this dictionary is not primally feasible. 
In the current dictionary, we run the Dual Simplex method by letting $w_1$
enter the basis, while letting $w_2$ leave the basis.
This leaves us with the following optimal dictionary:
\[
    \begin{array}{rcrcrcr}
        \zeta &=& 0.58 &-& 0.20 w_3 &-& 0.20 w_2 \\ \hline
            y &=& 0.50 &-& 0.50 w_3 &-& 0.50 w_2 \\
            x &=& 1.50 &-& 0.50 w_3 &-& 1.50 w_2 \\
          w_1 &=& 1.00 &+& 1.00 w_3 &+& 2.00 w_2
    \end{array}
\]
From here we start a new iteration of the algorithm described in Section
\ref{sec:problem} and \ref{sec:example} while preserving the old classification
of basic and nonbasic variables and using our current primal solution as an
initial guess $x_0$. We get that
\[
    T_0 = -2.5 + x - y,
\]
which gives us the objective function in Figure \ref{fig:prac1}. Solving that
linear program, we end up with a basic solution $\hat{x}_0 = (0,0)$.
Since $x_0$ and $\hat{x}_0$ are two adjacent vertices of the feasible region,
the entire line search takes place on the edge between those points.
Since the unconstrained minimum $x^u$ is infeasible, the solution $x^*$ must
lie on the boundary of the feasible region.
It follows from Figure \ref{fig:prac1} that $x^*$ lies on the edge joining 
$x_0$ and $\hat{x}_0$.
This means that a line search will find $x^*$ and the algorithm will terminate.
We perform the line search and find that $\alpha = 0.2$ and that
\[
    x_1 = 0.8 x_0 + 0.2 \hat{x}_0 = (1.2, 0.4) = x^*.
\]
