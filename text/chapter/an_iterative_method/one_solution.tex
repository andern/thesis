\section{One Solution Fits All}
Before any there are any changes in the column bounds, we have a solution for
the first instance, or, $\mathcal{Q}_1$ if you will. Subsequently, we find
the solution for $\mathcal{Q}_2, \mathcal{Q}_3, \ldots$ and so on.

Table \ref{table:zeroes} shows the number of zeroes in an optimal solution of
$\mathcal{Q}_1$ in each of the instances described in Chapter
\ref{ch:instances}. 
\begin{table}
    \centering
    \caption{The number of zeroes in an optimal solution of $\mathcal{Q}_1$} 
    \begin{tabular}{lrrr}
              & \textit{small} & \textit{large} & \textit{vlarge} \\\hline
    Variables & 238            & 952            & 3437 \\
    Zeroes    & 50             & 192            & 1749
    \end{tabular}
    \label{table:zeroes}
\end{table}

Consider the set of variables that are equal to zero in any solution. Now, this
set might be the empty set, or a set containing any number of variables.
To distinguish between optimal solutions over several instances, let us denote
an optimal solution of $\mathcal{Q}_k$ as $x_k^*$ for $k=1,2,\ldots$.

Consider a set $\mathcal{M}_k$ of variables that are to be forced to
zero, representing nodes in the network whose connection is down. Forcing
these variables to zero in the instance $\mathcal{Q}_1$ forms a new instance
$\mathcal{Q}_k$. Let us denote the set of variables that are equal to zero in
$x_k^*$ as $\mathcal{q}_k$.

If the \emph{modifier} $\mathcal{M}_2$ only has variables
that already are zero in $x^*_1$, then $x^*_2 = x^*_1$. If however,
$\mathcal{M}_2$ has some variables that are non-zero in $x^*_1$, then we
solve the new instance $\mathcal{Q}_2$ to achieve its solution $x^*_2$
along with the set $\mathcal{q}_2$.

Generally, for each $\mathcal{Q}_k$ and its modifier $\mathcal{M}_k$ we
have that all $\mathcal{Q}_l$ for $l=1,2,\ldots$ share the same solution
$x_k^*$ if $\mathcal{M}_l = S \cup \mathcal{M}_k, S \in
\mathcal{P}(\mathcal{q}_k)$ where $\mathcal{P}(\mathcal{q}_k)$ denotes the
power set of $\mathcal{q}_k$.

Generally, for each $\mathcal{Q}_k$ and its modifier $\mathcal{M}_k$ we
have that all $\mathcal{Q}_l$ for $l=1,2,\ldots$ share the same solution
$x_k^*$ if $\mathcal{M}_l = S \cup \mathcal{M}_k$ where $S \subseteq 
\mathcal{q}_k$.

Since we only care about whether a variable is zero \emph{or not}, we can
store these sets very compactly. By using one bit per variable, with a total
of $n$ variables, each set will require exactly $n$ bits of storage. We
store a $1$ for each zero element in a set, and $0$ for each non-zero element.
This way, we can check whether a set is a subset of another set by doing
bitwise operations. More specifically, \emph{implication} allows us to do this
comparison quite easily.

Consider two sets of bits, $A^b$ and $B^b$. By using bitwise implication on
each set's corresponding bits and negating the answer, then $A^b$ is a subset
of $B^b$ if and only if the answer is equal to 0. That is, $A^b$ is a subset of
$B^b$ if and only if $\neg(A^b \rightarrow^b B^b) = 0^b = 0$, where
$\Rightarrow^b$ represents bitwise implication and $0^b$ represents a set of
bits where all the bits are $0$. Similarly, a set $A$ is a subset of $B$ if and
only if $B \backslash A = \emptyset$.

Take for example the three sets, $D^b = 0000101001010$, $E^b = 0110101011011$
and $F^b = 1010100010010$. If we want to check whether $D$ and $F$ are subsets
of $E$, we check if the answer to the following operations are equal to zero:
\[
\begin{array}{lrl}
                      & 0000101001010 & D^b \\
  \quad \rightarrow^b & 0110101011011 & E^b \\ \hline
  \quad =             & 1111111111111 \\
  \neg  =             & 0000000000000
\end{array}
\qquad
\begin{array}{lrl}
                      & 1010100010010 & F^b \\
  \quad \rightarrow^b & 0110101011011 & E^b \\ \hline
  \quad =             & 0111111111111 \\
  \neg  =             & 1000000000000
\end{array}
\]
