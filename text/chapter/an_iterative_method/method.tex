\section{The Method}
\label{sec:method}
Consider the quadratic programming problem presented in Section
\ref{sec:problem}:
\[
\min_{x \in \mathbb{R}^n} f(x) = x^T H x + b^T x
\quad \textrm{subject to}
~
Ax = 0
~
\textrm{and}
~
l \le x \le u
\]
and call the objective function $f$.
The constraints are all linear, so by only linearizing the objective function
we end up with a linear program.
Let $T_k$ denote the Taylor series expansion at point $x_k$ of $f$.
Denote the linear program defined by minimization of $T_k$ subject to these
constraints by $\mathcal{L}_k$.
Let $\hat{x}_k$ denote an optimal solution of $\mathcal{L}_k$.

Let $k = 0$ and our initial guess $x_0$ be given.
Applying (\ref{eq:texp}) we have that
\[
T_k = - x_k^THx_k + 2x_k^THx + b^Tx
\]
Assume the linear program $\mathcal{L}_k$ has an optimal solution $\hat{x}_k$.
We note that $\mathcal{L}_k$ is minimizing in the direction of
$x^{\mathcal{U}}$, i.e. the unconstrained minimum of $f$.
Additionally we note that $\hat{x}_k$ lies on the boundary of the feasible
region.
From this we see that if the unconstrained minimum lies inside the feasible
region ($x^* = x^\mathcal{U}$) there exists a point on the line segment between
$x_k$ and $\hat{x}_k$ that is closer to $x^*$ than the two end points.

We perform a \emph{line search} by finding the minimum value of $f$ on the
straight line between the points $x_k$ and $\hat{x}_k$:
\[
\alpha_k \in \argmin_{\alpha \leq 1} f((1-\alpha) x_k + \alpha \hat{x}_k)
\]
Call $\alpha_k$ the \emph{step length}.

We find a closed-form definition of the step length $\alpha_k$ by letting
$m(\alpha) = f((1-\alpha) x_k + \alpha \hat{x}_k)$ and setting
$m^\prime(\alpha) = 0$ and solving for $\alpha$:
\[
\alpha_k = \frac{
                2x_k^T H x_k
                - 2\hat{x}_k^T H x_k
                + b^T x_k - b^T \hat{x}_k
                }{
                  2\hat{x}_k^T H \hat{x}_k
                - 4\hat{x}_k^T H x_k
                + 2x_k^T H x_k
                }
\]

We calculate $x_{k+1}$ using the step length $\alpha_k$:
\[
x_{k+1} = \alpha_k x_k + (1-\alpha_k)\hat{x}_k
\]
If, however, the unconstrained minimum lies outside the feasible region
($x^* \neq x^\mathcal{U}$), $\alpha_k$ might be less than $0$ and the minimum
value of $f$ of all the \emph{feasible} points collinearly with $x^*$ and
$\hat{x}_k$ will be $\hat{x}_k$.
TODO: Illustrate with a figure?

Increase $k$ by one and repeat the process from the calculation of $T_k$ until
we reach our termination condition.
We terminate when we are as close to $x^*$ as we want.
Given some $\epsilon$, we terminate when
\[
\frac{f(x_{k-1}) - f(x_k)}{|f(x_{k-1})|} \leq \epsilon
\]

\begin{algorithm}[ht!]
\caption{\texttt{slp($x_0$, $\epsilon \ge 0$)}}
\label{alg:iter}
\SetAlgoVlined
\DontPrintSemicolon
%Given $x_0$ and some $\epsilon \ge 0$\;
Set $k \leftarrow 0$\;

\Repeat{$\displaystyle \frac{f(x_{k-1}) - f(x_k)}{|f(x_{k-1})|} \leq \epsilon$} {
    $T_k \leftarrow - x_k^THx_k + 2x_k^THx + b^Tx$ \\
    $\hat{x}_k \leftarrow \textrm{optimal solution of } \mathcal{L}_k$ (Solve) \\
    $\alpha_k \in \argmin_{\alpha \leq 1} f((1-\alpha) x_k + \alpha \hat{x}_k)$ \\
%   $\alpha_k \in \argmin_{\alpha \leq 1} f(\alpha x_k + (1-\alpha)\hat{x}_k)$\;
    $x_{k+1} \leftarrow (1-\alpha_k) x_k + \alpha_k\hat{x}_k$ \\
    $k \leftarrow k +1$
}
\end{algorithm}

