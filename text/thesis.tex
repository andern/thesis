\documentclass[a4paper, twocolumn]{article}

%\usepackage{caption}
\usepackage{subcaption}

\newcommand*{\tabbox}[2][t]{%
        \vspace{0pt}\parbox[#1][3.7\baselineskip]{1cm}{\strut#2\strut}}

\usepackage{amsmath}
\usepackage{amsfonts}

\usepackage{tikz}
\usetikzlibrary{intersections,positioning,calc,arrows,shapes,
                decorations.pathreplacing,spy,automata}

\usepackage{pgfplots}
%\pgfplotsset{compat=1.7}

\usepackage{cite}

\usepackage[ruled]{algorithm2e}

\usepackage{paralist}

\usepackage{url}

\usepackage{verbatim}
\usepackage{chngcntr}
\counterwithin{figure}{section}
\counterwithin{equation}{section}
\counterwithin{table}{section}
\counterwithin{algocf}{section}

\DeclareMathOperator*{\argmin}{arg\,min}

\makeatletter
\tikzset{circle split part fill/.style  args={#1,#2}{%
alias=tmp@name, % Jake's idea !!
postaction={%
insert path={
\pgfextra{
\pgfpointdiff{\pgfpointanchor{\pgf@node@name}{center}}
{\pgfpointanchor{\pgf@node@name}{east}}
\pgfmathsetmacro\insiderad{\pgf@x}
\fill[#1] (\pgf@node@name.base) ([xshift=-\pgflinewidth]\pgf@node@name.east) arc
          (0:180:\insiderad-\pgflinewidth)--cycle;
\fill[#2] (\pgf@node@name.base) ([xshift=\pgflinewidth]\pgf@node@name.west) arc
          (180:360:\insiderad-\pgflinewidth)--cycle;
}}}}}
\makeatother

\begin{document}

% print a point given by two coordinates in pt (output is in cm)
\newcommand*\printpoint[2]{(%
\pgfmathparse{0.03514598035*#1}\pgfmathprintnumber{\pgfmathresult},%
\pgfmathparse{0.03514598035*#2}\pgfmathprintnumber{\pgfmathresult})%
}

\title{Fast Solver of a Quadratic Programming (QP) Problem}
\author{Andreas Halle\thanks{}}


%\pagenumbering{roman}
%\tableofcontents
%\listoffigures
%\listoftables

\twocolumn[
\begin{@twocolumnfalse}
\maketitle
\begin{abstract}
Goodtech and MathConsult have developed tools to compute the reliability of
delivery in a masked network where every link has an error rate.
This method is based on reliability theory to find actual error cases for
further analysis, and optimization theory to compute an optimal delivery
through the network for every error case.
The optimization problems are formulated as Quadratic Programming (QP)
problems.
Because the number of possible error cases increases rapidly as the size of the
network increases, it is crucial that every error case is analyzed very
quickly.
The aim is to develop a solver for effectively solving problems
\begin{inparaenum}
  \item with a problem size of 100 to 2000 variables;
  \item with sparse matrices;
  \item with positive-semidefinite quadratic terms in the objective function;
        and
  \item where the QP problem shall be solved many times with small variations
        in the constraints.
\end{inparaenum}
\end{abstract}
\end{@twocolumnfalse}
]
{
  \renewcommand{\thefootnote}%
    {\fnsymbol{footnote}}
  \footnotetext[1]{University of Bergen, Department of Informatics, P.O.Box 7803, N-5020 Bergen, Norway}
}

\section{Introduction}
As electrical transmission systems increase in complexity, it becomes
more difficult to understand how different components in the system interact.
We often aim to increase the utilization of such systems, and with that comes
the need for good planning and operational tools.
The Transmission System Operator (TSO) is faced
with increasing requirements regarding the reliability of load
delivery, and the cost of not delivering agreed energy can be
substantial\cite{digernes}.
The need for tools to assist the TSO in analyses that can help prevent
unreliable networks is important.
In \cite{digernes}, the authors state:
\begin{quote}
It is of utmost importance for the TSO to be able to perform detailed and
accurate reliability analyses; for the daily operation as well as for
future planning (comparison of reinforcement alternatives etc.). By
including power flow considerations in the calculations, the operator is
able to plan where to locate the spinning reserves to maximize the load
delivery reliability. Reliability analyses are also important for system
planning, for analyzing different alternatives for network reinforcement's
[sic] etc.\cite{digernes}
\end{quote}

Goodtech and MathConsult have developed a tool called PROMAPS
(Probability Methods Applied to Power Systems). It calculates the power
delivery as a function of demand, and the probability for undelivered
energy for each load branch in the system, and in the system as a whole
\cite{trond}. A typical calculation sequence in PROMAPS includes the
following main functions\cite{trond}:
\begin{enumerate}
\item Creation of branch reliability models based on unit Markov models and
      composition and aggregation of states.
\item Selection of a subset of states containing all grid reliability states
      with a significant reliability.
\item Calculation of maximum power delivery capacity for each of the
      significant grid reliability states based on an object function with
      constraints.
\item Calculation of expected power shortage and creation of delivery
      reliability models based on the operations composition and aggregation.
\item Calculation of delivery probability, mean visiting duration and visiting
      frequency for functioning and failed delivery states.
\item Post calculation of various auxiliary variables including economic data.
\end{enumerate}
Among these six functions, the third main function was identified as the most
time-consuming, or more specifically, a call to a QP solver.

The maximum power
supply can be calculated by an objective function that represents the power
delivery profit. A typical objective function is
\[
J = -C_G - P^T \Phi C_D P + C_L P,
\]
where $J$ is the objective function value (E/s), $P$ is the branch power vector
($W$), $C_G$ is a line vector containing specific power generation cost
($E/J$), $\Phi$ is a diagonal matrix representing power loss ($1/W$), $C_D$ is
a diagonal matrix containing specific power transmission cost ($E/J$), and
$C_L$ is a line vector containing specific power delivery price ($E/J$).
The symbol $E$ is the economical unit\cite{digernes}.
%\begingroup
%\let\clearpage\relax
\chapter{A QP Formulation}
\label{ch:qp}
\label{sec:formulation}
A general formulation of non-linear optimization problems is
\begin{equation}
    \label{eq:generalqp}
    \min_{x \in \mathbb{R}^n} f(x) \quad \textrm{subject to}
    \begin{cases}
        ~c_i(x) = 0,   & i \in \mathcal{E}, \\
        ~c_i(x) \ge 0, & i \in \mathcal{I},
    \end{cases}
\end{equation}
where $f$ and the functions $c_i$ are all smooth, real-valued functions on a
subset of $\mathbb{R}^n$, and $\mathcal{I}$ and $\mathcal{E}$ are two finite
sets of indices~\cite{nocedal}.

The problem discussed in this thesis has (from (\ref{eq:obj})) a convex,
quadratic, separable objective function
\[
    f(x) = x^T H x + b^T x
\]
and linear constraints. It qualifies as a convex QP problem and is formulated in
(\ref{eq:thesisqp}) as
\[
    \min_{x \in \mathbb{R}^n} f(x)\quad\textrm{subject to}~Ax = 0, ~l \leq x \leq u
\]
%\begin{equation}
%    \label{eq:thesisqp}
%    \min_{x \in \mathbb{R}^n} f(x)
%    \min_{x_{\textrm{min}} \le x \le x_{\textrm{max}}} f(x)
%    \min_{l \le x \le u} f(x)
%    \quad \textrm{subject to}
%    ~
%    Ax = 0
%    ~
%    \textrm{and}
%    ~
%    l \le x \le u
%\end{equation}
where $H$ is a positive semidefinite diagonal $n \times n$ matrix, $A$ is a
$m \times n$ oriented incidence matrix, and $b, l, u$ and $x$ are vectors in
$\mathbb{R}^n$.

The function $f$ is called the objective function of
the problem. In (\ref{eq:obj}) and (\ref{eq:thesisqp}), $f$ consists of a
quadratic term ($x^T H x$) and a linear term ($b^T x$).

In practical applications, a large proportion of the diagonal elements of $H$
and $b$ are zero.
For the non-zero elements of $H$, we typically have
$10^{-5} \le h_i \le 10^{-1}$.
Here $h_i$ denotes the $i$th diagonal element of $H$.
For the non-zero elements $b_i$ of vector $b$, we typically have
$10 \le |b_i| \le 70$. All elements of $l$ and $u$ are non-positive 
and non-negative, respectively. However, the methods developed in this thesis
do not require these values.
These values come part from conversations with Goodtech, and
part from a couple of real instances that they have supplied.
The next section describes the supplied instances in detail.

\section{Real Instances}
\label{sec:instances}
\begin{figure}[h!]
\begin{center}
\input{include/newhistH}
\end{center}
\caption{Value distribution of non-zero elements in $H$}
\label{fig:histH}
\end{figure}

\section{The Instances}
The instances have a lot in common, but most importantly they all:
\begin{enumerate}
\item Have very large coefficients in the linear term compared to the
quadratic term.
\item All cross-product coefficients are zero in the quadratic term, i.e. in
$x^T H x$, $H$ is diagonal.
\end{enumerate}

The instances are called---based on size---\textit{small}, \textit{large}
and \textit{vlarge}.

Figure \ref{fig:histH} shows the value distribution of non-zero elements in $H$
in the three instances. Note that among all the diagonal elements of H, more
than 50 percent of them are zero, and among the non-zero elements, most of them
are less than $10^{-2}$.
Observing that no non-zero element in the linear term in any of the instances
has absolute value less than 20 reinforces our earlier motivation and
cause to look into SLP.

\begin{table}
    \centering
    \caption{Problem size of each instance}
    \begin{tabular}{lrrr}
    Problem size & \textit{small} & \textit{large} & \textit{vlarge} \\\hline
    Rows         & 82             & 328            & 1127 \\
    Columns      & 238            & 952            & 3437 \\
    Non-zeroes A & 348            & 1392           & 4840 \\
    Non-zeroes H & 108            & 432            & 894 \\
    \end{tabular}
    \label{table:sizes}
\end{table}

Let us take a look at some hard statistics of each instance.
Table \ref{table:sizes} shows the problem size of each instance.
Table \ref{table:maxmin} shows some statistics about the non-zero elements in
the objective function for $i = 1,2,\ldots,n$ where $n$ is the number of
columns.
\begin{table}
    \centering
    \caption{Non-zero values in the objective function of each instance}
    \begin{tabular}{lrrr}
                        & \textit{small}          & \textit{large}          & \textit{vlarge} \\\hline
    $\max(h_{ii})$      & $2.9614 \times 10^{-2}$ & $2.9614 \times 10^{-2}$ & $4.9011 \times 10^{-2}$ \\
    $\min(h_{ii})$      & $4.9290 \times 10^{-5}$ & $4.9290 \times 10^{-5}$ & $1.1026 \times 10^{-5}$ \\
$\textrm{mean}(h_{ii})$ & $5.2864 \times 10^{-3}$ & $5.2864 \times 10^{-3}$ & $5.8984 \times 10^{-3}$ \\
    $\max(b_{i})$       & 20                      & 20                      & 20 \\
    $\min(b_{i})$       & -70                     & -70                     & -50 \\
    \end{tabular}
    \label{table:maxmin}
\end{table}
The rows and columns in each problem represent vertices and edges respectively.
An avid reader might notice that \textit{small} and \textit{large} are very
much alike.
That is entirely justified as \textit{large} is four \textit{small}-networks
connected forming one large network.

For each of these instances, GoodTech AS wants to solve several other instances
with slight variations.
These instances are from here on called \emph{subinstances}.
If one or more variables in an instance $\mathcal{Q}$ is forced to zero to form
a new instance, the newly formed instance is a subinstance of $\mathcal{Q}$.
GoodTech AS wants to simulate that all combinations of edges in the network
falls down, by forcing their corresponding variable values to zero.
This means that for each of the three instances described in this chapter,
they ideally want to solve a total of $2^n$ subinstances.
Solving all $2^n$ subinstances is  impossible.
It is unlikely that all edges in the network break down.
A more realistic approach is to try to solve all subinstances with no more than
$k$ break-downs, i.e. we solve $\sum_{j=0}^k {n \choose j}$ subinstances
with less than or equal to $k$ break-downs.

\section{Subinstances}
\section{The Subinstances}
For one or more given variables, their lower and upper bounds are set to zero,
i.e. forcing the variables to zero.
Let us denote the set of variables that are to be forced to zero by
$\mathcal{M}_k$ for some $1 \leq k \leq 2^n - 1$ where $n$ is the number
of edges in the network.
Forcing these variables to zero in some instance $\mathcal{Q}$ forms
a subinstance $\mathcal{Q}_k$. We refer to $\mathcal{M}_k$ as a
\emph{modifier} of $\mathcal{Q}$.
To simplify notation from here on, we let
$\mathcal{Q}_0 = \mathcal{Q}$ and loosen the terminology of
\emph{subinstance} so that $\mathcal{Q}_0$ can be called a subinstance of
$\mathcal{Q}$ where its modifier $\mathcal{M}_0 = \emptyset$.

To distinguish between optimal solutions over several subinstances, we denote
an optimal solution of $\mathcal{Q}_k$ as $x_k^*$.
In any solution of any instance there might be variables that are zero.
We denote the set of variables that are zero in an \emph{optimal} solution
$x_k^*$ by $\mathcal{Z}_k$.
If a modifier $\mathcal{M}_k$ only has variables that are zero in $x_0^*$,
i.e. if $\mathcal{M}_k \subseteq \mathcal{Z}_0$, then $x_k^* = x_0^*$. However,
if $\mathcal{M}_k$ has some variables that are non-zero in $x_0^*$, then
we solve $\mathcal{Q}_k$ to achieve its solution $x_k^*$ along with the set
$\mathcal{Z}_k$.

Generally, for each $\mathcal{Q}_k$ and its modifier $\mathcal{M}_k$ we
have that $x_l^* = x_k^*$ for all $l=1,2,\ldots,2^n-1$ if
$\mathcal{M}_l = S \cup \mathcal{M}_k$ for some $S \subseteq \mathcal{Z}_k$.
That is, instances $\mathcal{Q}_k$ and $\mathcal{Q}_l$ have identical optimal
solutions if $\mathcal{M}_l$ is obtained by extending $\mathcal{M}_k$ only with
variables that are zero in the optimal solution to $\mathcal{Q}_k$.


\chapter{A QP Formulation}
\label{ch:qp}
\label{sec:formulation}
A general formulation of non-linear optimization problems is
\begin{equation}
    \label{eq:generalqp}
    \min_{x \in \mathbb{R}^n} f(x) \quad \textrm{subject to}
    \begin{cases}
        ~c_i(x) = 0,   & i \in \mathcal{E}, \\
        ~c_i(x) \ge 0, & i \in \mathcal{I},
    \end{cases}
\end{equation}
where $f$ and the functions $c_i$ are all smooth, real-valued functions on a
subset of $\mathbb{R}^n$, and $\mathcal{I}$ and $\mathcal{E}$ are two finite
sets of indices~\cite{nocedal}.

The problem discussed in this thesis has (from (\ref{eq:obj})) a convex,
quadratic, separable objective function
\[
    f(x) = x^T H x + b^T x
\]
and linear constraints. It qualifies as a convex QP problem and is formulated in
(\ref{eq:thesisqp}) as
\[
    \min_{x \in \mathbb{R}^n} f(x)\quad\textrm{subject to}~Ax = 0, ~l \leq x \leq u
\]
%\begin{equation}
%    \label{eq:thesisqp}
%    \min_{x \in \mathbb{R}^n} f(x)
%    \min_{x_{\textrm{min}} \le x \le x_{\textrm{max}}} f(x)
%    \min_{l \le x \le u} f(x)
%    \quad \textrm{subject to}
%    ~
%    Ax = 0
%    ~
%    \textrm{and}
%    ~
%    l \le x \le u
%\end{equation}
where $H$ is a positive semidefinite diagonal $n \times n$ matrix, $A$ is a
$m \times n$ oriented incidence matrix, and $b, l, u$ and $x$ are vectors in
$\mathbb{R}^n$.

The function $f$ is called the objective function of
the problem. In (\ref{eq:obj}) and (\ref{eq:thesisqp}), $f$ consists of a
quadratic term ($x^T H x$) and a linear term ($b^T x$).

In practical applications, a large proportion of the diagonal elements of $H$
and $b$ are zero.
For the non-zero elements of $H$, we typically have
$10^{-5} \le h_i \le 10^{-1}$.
Here $h_i$ denotes the $i$th diagonal element of $H$.
For the non-zero elements $b_i$ of vector $b$, we typically have
$10 \le |b_i| \le 70$. All elements of $l$ and $u$ are non-positive 
and non-negative, respectively. However, the methods developed in this thesis
do not require these values.
These values come part from conversations with Goodtech, and
part from a couple of real instances that they have supplied.
The next section describes the supplied instances in detail.

\section{Real Instances}
\label{sec:instances}
\begin{figure}[h!]
\begin{center}
\input{include/newhistH}
\end{center}
\caption{Value distribution of non-zero elements in $H$}
\label{fig:histH}
\end{figure}

\section{The Instances}
The instances have a lot in common, but most importantly they all:
\begin{enumerate}
\item Have very large coefficients in the linear term compared to the
quadratic term.
\item All cross-product coefficients are zero in the quadratic term, i.e. in
$x^T H x$, $H$ is diagonal.
\end{enumerate}

The instances are called---based on size---\textit{small}, \textit{large}
and \textit{vlarge}.

Figure \ref{fig:histH} shows the value distribution of non-zero elements in $H$
in the three instances. Note that among all the diagonal elements of H, more
than 50 percent of them are zero, and among the non-zero elements, most of them
are less than $10^{-2}$.
Observing that no non-zero element in the linear term in any of the instances
has absolute value less than 20 reinforces our earlier motivation and
cause to look into SLP.

\begin{table}
    \centering
    \caption{Problem size of each instance}
    \begin{tabular}{lrrr}
    Problem size & \textit{small} & \textit{large} & \textit{vlarge} \\\hline
    Rows         & 82             & 328            & 1127 \\
    Columns      & 238            & 952            & 3437 \\
    Non-zeroes A & 348            & 1392           & 4840 \\
    Non-zeroes H & 108            & 432            & 894 \\
    \end{tabular}
    \label{table:sizes}
\end{table}

Let us take a look at some hard statistics of each instance.
Table \ref{table:sizes} shows the problem size of each instance.
Table \ref{table:maxmin} shows some statistics about the non-zero elements in
the objective function for $i = 1,2,\ldots,n$ where $n$ is the number of
columns.
\begin{table}
    \centering
    \caption{Non-zero values in the objective function of each instance}
    \begin{tabular}{lrrr}
                        & \textit{small}          & \textit{large}          & \textit{vlarge} \\\hline
    $\max(h_{ii})$      & $2.9614 \times 10^{-2}$ & $2.9614 \times 10^{-2}$ & $4.9011 \times 10^{-2}$ \\
    $\min(h_{ii})$      & $4.9290 \times 10^{-5}$ & $4.9290 \times 10^{-5}$ & $1.1026 \times 10^{-5}$ \\
$\textrm{mean}(h_{ii})$ & $5.2864 \times 10^{-3}$ & $5.2864 \times 10^{-3}$ & $5.8984 \times 10^{-3}$ \\
    $\max(b_{i})$       & 20                      & 20                      & 20 \\
    $\min(b_{i})$       & -70                     & -70                     & -50 \\
    \end{tabular}
    \label{table:maxmin}
\end{table}
The rows and columns in each problem represent vertices and edges respectively.
An avid reader might notice that \textit{small} and \textit{large} are very
much alike.
That is entirely justified as \textit{large} is four \textit{small}-networks
connected forming one large network.

For each of these instances, GoodTech AS wants to solve several other instances
with slight variations.
These instances are from here on called \emph{subinstances}.
If one or more variables in an instance $\mathcal{Q}$ is forced to zero to form
a new instance, the newly formed instance is a subinstance of $\mathcal{Q}$.
GoodTech AS wants to simulate that all combinations of edges in the network
falls down, by forcing their corresponding variable values to zero.
This means that for each of the three instances described in this chapter,
they ideally want to solve a total of $2^n$ subinstances.
Solving all $2^n$ subinstances is  impossible.
It is unlikely that all edges in the network break down.
A more realistic approach is to try to solve all subinstances with no more than
$k$ break-downs, i.e. we solve $\sum_{j=0}^k {n \choose j}$ subinstances
with less than or equal to $k$ break-downs.

\section{Subinstances}
\section{The Subinstances}
For one or more given variables, their lower and upper bounds are set to zero,
i.e. forcing the variables to zero.
Let us denote the set of variables that are to be forced to zero by
$\mathcal{M}_k$ for some $1 \leq k \leq 2^n - 1$ where $n$ is the number
of edges in the network.
Forcing these variables to zero in some instance $\mathcal{Q}$ forms
a subinstance $\mathcal{Q}_k$. We refer to $\mathcal{M}_k$ as a
\emph{modifier} of $\mathcal{Q}$.
To simplify notation from here on, we let
$\mathcal{Q}_0 = \mathcal{Q}$ and loosen the terminology of
\emph{subinstance} so that $\mathcal{Q}_0$ can be called a subinstance of
$\mathcal{Q}$ where its modifier $\mathcal{M}_0 = \emptyset$.

To distinguish between optimal solutions over several subinstances, we denote
an optimal solution of $\mathcal{Q}_k$ as $x_k^*$.
In any solution of any instance there might be variables that are zero.
We denote the set of variables that are zero in an \emph{optimal} solution
$x_k^*$ by $\mathcal{Z}_k$.
If a modifier $\mathcal{M}_k$ only has variables that are zero in $x_0^*$,
i.e. if $\mathcal{M}_k \subseteq \mathcal{Z}_0$, then $x_k^* = x_0^*$. However,
if $\mathcal{M}_k$ has some variables that are non-zero in $x_0^*$, then
we solve $\mathcal{Q}_k$ to achieve its solution $x_k^*$ along with the set
$\mathcal{Z}_k$.

Generally, for each $\mathcal{Q}_k$ and its modifier $\mathcal{M}_k$ we
have that $x_l^* = x_k^*$ for all $l=1,2,\ldots,2^n-1$ if
$\mathcal{M}_l = S \cup \mathcal{M}_k$ for some $S \subseteq \mathcal{Z}_k$.
That is, instances $\mathcal{Q}_k$ and $\mathcal{Q}_l$ have identical optimal
solutions if $\mathcal{M}_l$ is obtained by extending $\mathcal{M}_k$ only with
variables that are zero in the optimal solution to $\mathcal{Q}_k$.


\chapter{A QP Formulation}
\label{ch:qp}
\label{sec:formulation}
A general formulation of non-linear optimization problems is
\begin{equation}
    \label{eq:generalqp}
    \min_{x \in \mathbb{R}^n} f(x) \quad \textrm{subject to}
    \begin{cases}
        ~c_i(x) = 0,   & i \in \mathcal{E}, \\
        ~c_i(x) \ge 0, & i \in \mathcal{I},
    \end{cases}
\end{equation}
where $f$ and the functions $c_i$ are all smooth, real-valued functions on a
subset of $\mathbb{R}^n$, and $\mathcal{I}$ and $\mathcal{E}$ are two finite
sets of indices~\cite{nocedal}.

The problem discussed in this thesis has (from (\ref{eq:obj})) a convex,
quadratic, separable objective function
\[
    f(x) = x^T H x + b^T x
\]
and linear constraints. It qualifies as a convex QP problem and is formulated in
(\ref{eq:thesisqp}) as
\[
    \min_{x \in \mathbb{R}^n} f(x)\quad\textrm{subject to}~Ax = 0, ~l \leq x \leq u
\]
%\begin{equation}
%    \label{eq:thesisqp}
%    \min_{x \in \mathbb{R}^n} f(x)
%    \min_{x_{\textrm{min}} \le x \le x_{\textrm{max}}} f(x)
%    \min_{l \le x \le u} f(x)
%    \quad \textrm{subject to}
%    ~
%    Ax = 0
%    ~
%    \textrm{and}
%    ~
%    l \le x \le u
%\end{equation}
where $H$ is a positive semidefinite diagonal $n \times n$ matrix, $A$ is a
$m \times n$ oriented incidence matrix, and $b, l, u$ and $x$ are vectors in
$\mathbb{R}^n$.

The function $f$ is called the objective function of
the problem. In (\ref{eq:obj}) and (\ref{eq:thesisqp}), $f$ consists of a
quadratic term ($x^T H x$) and a linear term ($b^T x$).

In practical applications, a large proportion of the diagonal elements of $H$
and $b$ are zero.
For the non-zero elements of $H$, we typically have
$10^{-5} \le h_i \le 10^{-1}$.
Here $h_i$ denotes the $i$th diagonal element of $H$.
For the non-zero elements $b_i$ of vector $b$, we typically have
$10 \le |b_i| \le 70$. All elements of $l$ and $u$ are non-positive 
and non-negative, respectively. However, the methods developed in this thesis
do not require these values.
These values come part from conversations with Goodtech, and
part from a couple of real instances that they have supplied.
The next section describes the supplied instances in detail.

\section{Real Instances}
\label{sec:instances}
\begin{figure}[h!]
\begin{center}
\input{include/newhistH}
\end{center}
\caption{Value distribution of non-zero elements in $H$}
\label{fig:histH}
\end{figure}

\section{The Instances}
The instances have a lot in common, but most importantly they all:
\begin{enumerate}
\item Have very large coefficients in the linear term compared to the
quadratic term.
\item All cross-product coefficients are zero in the quadratic term, i.e. in
$x^T H x$, $H$ is diagonal.
\end{enumerate}

The instances are called---based on size---\textit{small}, \textit{large}
and \textit{vlarge}.

Figure \ref{fig:histH} shows the value distribution of non-zero elements in $H$
in the three instances. Note that among all the diagonal elements of H, more
than 50 percent of them are zero, and among the non-zero elements, most of them
are less than $10^{-2}$.
Observing that no non-zero element in the linear term in any of the instances
has absolute value less than 20 reinforces our earlier motivation and
cause to look into SLP.

\begin{table}
    \centering
    \caption{Problem size of each instance}
    \begin{tabular}{lrrr}
    Problem size & \textit{small} & \textit{large} & \textit{vlarge} \\\hline
    Rows         & 82             & 328            & 1127 \\
    Columns      & 238            & 952            & 3437 \\
    Non-zeroes A & 348            & 1392           & 4840 \\
    Non-zeroes H & 108            & 432            & 894 \\
    \end{tabular}
    \label{table:sizes}
\end{table}

Let us take a look at some hard statistics of each instance.
Table \ref{table:sizes} shows the problem size of each instance.
Table \ref{table:maxmin} shows some statistics about the non-zero elements in
the objective function for $i = 1,2,\ldots,n$ where $n$ is the number of
columns.
\begin{table}
    \centering
    \caption{Non-zero values in the objective function of each instance}
    \begin{tabular}{lrrr}
                        & \textit{small}          & \textit{large}          & \textit{vlarge} \\\hline
    $\max(h_{ii})$      & $2.9614 \times 10^{-2}$ & $2.9614 \times 10^{-2}$ & $4.9011 \times 10^{-2}$ \\
    $\min(h_{ii})$      & $4.9290 \times 10^{-5}$ & $4.9290 \times 10^{-5}$ & $1.1026 \times 10^{-5}$ \\
$\textrm{mean}(h_{ii})$ & $5.2864 \times 10^{-3}$ & $5.2864 \times 10^{-3}$ & $5.8984 \times 10^{-3}$ \\
    $\max(b_{i})$       & 20                      & 20                      & 20 \\
    $\min(b_{i})$       & -70                     & -70                     & -50 \\
    \end{tabular}
    \label{table:maxmin}
\end{table}
The rows and columns in each problem represent vertices and edges respectively.
An avid reader might notice that \textit{small} and \textit{large} are very
much alike.
That is entirely justified as \textit{large} is four \textit{small}-networks
connected forming one large network.

For each of these instances, GoodTech AS wants to solve several other instances
with slight variations.
These instances are from here on called \emph{subinstances}.
If one or more variables in an instance $\mathcal{Q}$ is forced to zero to form
a new instance, the newly formed instance is a subinstance of $\mathcal{Q}$.
GoodTech AS wants to simulate that all combinations of edges in the network
falls down, by forcing their corresponding variable values to zero.
This means that for each of the three instances described in this chapter,
they ideally want to solve a total of $2^n$ subinstances.
Solving all $2^n$ subinstances is  impossible.
It is unlikely that all edges in the network break down.
A more realistic approach is to try to solve all subinstances with no more than
$k$ break-downs, i.e. we solve $\sum_{j=0}^k {n \choose j}$ subinstances
with less than or equal to $k$ break-downs.

\section{Subinstances}
\section{The Subinstances}
For one or more given variables, their lower and upper bounds are set to zero,
i.e. forcing the variables to zero.
Let us denote the set of variables that are to be forced to zero by
$\mathcal{M}_k$ for some $1 \leq k \leq 2^n - 1$ where $n$ is the number
of edges in the network.
Forcing these variables to zero in some instance $\mathcal{Q}$ forms
a subinstance $\mathcal{Q}_k$. We refer to $\mathcal{M}_k$ as a
\emph{modifier} of $\mathcal{Q}$.
To simplify notation from here on, we let
$\mathcal{Q}_0 = \mathcal{Q}$ and loosen the terminology of
\emph{subinstance} so that $\mathcal{Q}_0$ can be called a subinstance of
$\mathcal{Q}$ where its modifier $\mathcal{M}_0 = \emptyset$.

To distinguish between optimal solutions over several subinstances, we denote
an optimal solution of $\mathcal{Q}_k$ as $x_k^*$.
In any solution of any instance there might be variables that are zero.
We denote the set of variables that are zero in an \emph{optimal} solution
$x_k^*$ by $\mathcal{Z}_k$.
If a modifier $\mathcal{M}_k$ only has variables that are zero in $x_0^*$,
i.e. if $\mathcal{M}_k \subseteq \mathcal{Z}_0$, then $x_k^* = x_0^*$. However,
if $\mathcal{M}_k$ has some variables that are non-zero in $x_0^*$, then
we solve $\mathcal{Q}_k$ to achieve its solution $x_k^*$ along with the set
$\mathcal{Z}_k$.

Generally, for each $\mathcal{Q}_k$ and its modifier $\mathcal{M}_k$ we
have that $x_l^* = x_k^*$ for all $l=1,2,\ldots,2^n-1$ if
$\mathcal{M}_l = S \cup \mathcal{M}_k$ for some $S \subseteq \mathcal{Z}_k$.
That is, instances $\mathcal{Q}_k$ and $\mathcal{Q}_l$ have identical optimal
solutions if $\mathcal{M}_l$ is obtained by extending $\mathcal{M}_k$ only with
variables that are zero in the optimal solution to $\mathcal{Q}_k$.


\chapter{A QP Formulation}
\label{ch:qp}
\label{sec:formulation}
A general formulation of non-linear optimization problems is
\begin{equation}
    \label{eq:generalqp}
    \min_{x \in \mathbb{R}^n} f(x) \quad \textrm{subject to}
    \begin{cases}
        ~c_i(x) = 0,   & i \in \mathcal{E}, \\
        ~c_i(x) \ge 0, & i \in \mathcal{I},
    \end{cases}
\end{equation}
where $f$ and the functions $c_i$ are all smooth, real-valued functions on a
subset of $\mathbb{R}^n$, and $\mathcal{I}$ and $\mathcal{E}$ are two finite
sets of indices~\cite{nocedal}.

The problem discussed in this thesis has (from (\ref{eq:obj})) a convex,
quadratic, separable objective function
\[
    f(x) = x^T H x + b^T x
\]
and linear constraints. It qualifies as a convex QP problem and is formulated in
(\ref{eq:thesisqp}) as
\[
    \min_{x \in \mathbb{R}^n} f(x)\quad\textrm{subject to}~Ax = 0, ~l \leq x \leq u
\]
%\begin{equation}
%    \label{eq:thesisqp}
%    \min_{x \in \mathbb{R}^n} f(x)
%    \min_{x_{\textrm{min}} \le x \le x_{\textrm{max}}} f(x)
%    \min_{l \le x \le u} f(x)
%    \quad \textrm{subject to}
%    ~
%    Ax = 0
%    ~
%    \textrm{and}
%    ~
%    l \le x \le u
%\end{equation}
where $H$ is a positive semidefinite diagonal $n \times n$ matrix, $A$ is a
$m \times n$ oriented incidence matrix, and $b, l, u$ and $x$ are vectors in
$\mathbb{R}^n$.

The function $f$ is called the objective function of
the problem. In (\ref{eq:obj}) and (\ref{eq:thesisqp}), $f$ consists of a
quadratic term ($x^T H x$) and a linear term ($b^T x$).

In practical applications, a large proportion of the diagonal elements of $H$
and $b$ are zero.
For the non-zero elements of $H$, we typically have
$10^{-5} \le h_i \le 10^{-1}$.
Here $h_i$ denotes the $i$th diagonal element of $H$.
For the non-zero elements $b_i$ of vector $b$, we typically have
$10 \le |b_i| \le 70$. All elements of $l$ and $u$ are non-positive 
and non-negative, respectively. However, the methods developed in this thesis
do not require these values.
These values come part from conversations with Goodtech, and
part from a couple of real instances that they have supplied.
The next section describes the supplied instances in detail.

\section{Real Instances}
\label{sec:instances}
\begin{figure}[h!]
\begin{center}
\input{include/newhistH}
\end{center}
\caption{Value distribution of non-zero elements in $H$}
\label{fig:histH}
\end{figure}

\section{The Instances}
The instances have a lot in common, but most importantly they all:
\begin{enumerate}
\item Have very large coefficients in the linear term compared to the
quadratic term.
\item All cross-product coefficients are zero in the quadratic term, i.e. in
$x^T H x$, $H$ is diagonal.
\end{enumerate}

The instances are called---based on size---\textit{small}, \textit{large}
and \textit{vlarge}.

Figure \ref{fig:histH} shows the value distribution of non-zero elements in $H$
in the three instances. Note that among all the diagonal elements of H, more
than 50 percent of them are zero, and among the non-zero elements, most of them
are less than $10^{-2}$.
Observing that no non-zero element in the linear term in any of the instances
has absolute value less than 20 reinforces our earlier motivation and
cause to look into SLP.

\begin{table}
    \centering
    \caption{Problem size of each instance}
    \begin{tabular}{lrrr}
    Problem size & \textit{small} & \textit{large} & \textit{vlarge} \\\hline
    Rows         & 82             & 328            & 1127 \\
    Columns      & 238            & 952            & 3437 \\
    Non-zeroes A & 348            & 1392           & 4840 \\
    Non-zeroes H & 108            & 432            & 894 \\
    \end{tabular}
    \label{table:sizes}
\end{table}

Let us take a look at some hard statistics of each instance.
Table \ref{table:sizes} shows the problem size of each instance.
Table \ref{table:maxmin} shows some statistics about the non-zero elements in
the objective function for $i = 1,2,\ldots,n$ where $n$ is the number of
columns.
\begin{table}
    \centering
    \caption{Non-zero values in the objective function of each instance}
    \begin{tabular}{lrrr}
                        & \textit{small}          & \textit{large}          & \textit{vlarge} \\\hline
    $\max(h_{ii})$      & $2.9614 \times 10^{-2}$ & $2.9614 \times 10^{-2}$ & $4.9011 \times 10^{-2}$ \\
    $\min(h_{ii})$      & $4.9290 \times 10^{-5}$ & $4.9290 \times 10^{-5}$ & $1.1026 \times 10^{-5}$ \\
$\textrm{mean}(h_{ii})$ & $5.2864 \times 10^{-3}$ & $5.2864 \times 10^{-3}$ & $5.8984 \times 10^{-3}$ \\
    $\max(b_{i})$       & 20                      & 20                      & 20 \\
    $\min(b_{i})$       & -70                     & -70                     & -50 \\
    \end{tabular}
    \label{table:maxmin}
\end{table}
The rows and columns in each problem represent vertices and edges respectively.
An avid reader might notice that \textit{small} and \textit{large} are very
much alike.
That is entirely justified as \textit{large} is four \textit{small}-networks
connected forming one large network.

For each of these instances, GoodTech AS wants to solve several other instances
with slight variations.
These instances are from here on called \emph{subinstances}.
If one or more variables in an instance $\mathcal{Q}$ is forced to zero to form
a new instance, the newly formed instance is a subinstance of $\mathcal{Q}$.
GoodTech AS wants to simulate that all combinations of edges in the network
falls down, by forcing their corresponding variable values to zero.
This means that for each of the three instances described in this chapter,
they ideally want to solve a total of $2^n$ subinstances.
Solving all $2^n$ subinstances is  impossible.
It is unlikely that all edges in the network break down.
A more realistic approach is to try to solve all subinstances with no more than
$k$ break-downs, i.e. we solve $\sum_{j=0}^k {n \choose j}$ subinstances
with less than or equal to $k$ break-downs.

\section{Subinstances}
\section{The Subinstances}
For one or more given variables, their lower and upper bounds are set to zero,
i.e. forcing the variables to zero.
Let us denote the set of variables that are to be forced to zero by
$\mathcal{M}_k$ for some $1 \leq k \leq 2^n - 1$ where $n$ is the number
of edges in the network.
Forcing these variables to zero in some instance $\mathcal{Q}$ forms
a subinstance $\mathcal{Q}_k$. We refer to $\mathcal{M}_k$ as a
\emph{modifier} of $\mathcal{Q}$.
To simplify notation from here on, we let
$\mathcal{Q}_0 = \mathcal{Q}$ and loosen the terminology of
\emph{subinstance} so that $\mathcal{Q}_0$ can be called a subinstance of
$\mathcal{Q}$ where its modifier $\mathcal{M}_0 = \emptyset$.

To distinguish between optimal solutions over several subinstances, we denote
an optimal solution of $\mathcal{Q}_k$ as $x_k^*$.
In any solution of any instance there might be variables that are zero.
We denote the set of variables that are zero in an \emph{optimal} solution
$x_k^*$ by $\mathcal{Z}_k$.
If a modifier $\mathcal{M}_k$ only has variables that are zero in $x_0^*$,
i.e. if $\mathcal{M}_k \subseteq \mathcal{Z}_0$, then $x_k^* = x_0^*$. However,
if $\mathcal{M}_k$ has some variables that are non-zero in $x_0^*$, then
we solve $\mathcal{Q}_k$ to achieve its solution $x_k^*$ along with the set
$\mathcal{Z}_k$.

Generally, for each $\mathcal{Q}_k$ and its modifier $\mathcal{M}_k$ we
have that $x_l^* = x_k^*$ for all $l=1,2,\ldots,2^n-1$ if
$\mathcal{M}_l = S \cup \mathcal{M}_k$ for some $S \subseteq \mathcal{Z}_k$.
That is, instances $\mathcal{Q}_k$ and $\mathcal{Q}_l$ have identical optimal
solutions if $\mathcal{M}_l$ is obtained by extending $\mathcal{M}_k$ only with
variables that are zero in the optimal solution to $\mathcal{Q}_k$.


\chapter{A QP Formulation}
\label{ch:qp}
\label{sec:formulation}
A general formulation of non-linear optimization problems is
\begin{equation}
    \label{eq:generalqp}
    \min_{x \in \mathbb{R}^n} f(x) \quad \textrm{subject to}
    \begin{cases}
        ~c_i(x) = 0,   & i \in \mathcal{E}, \\
        ~c_i(x) \ge 0, & i \in \mathcal{I},
    \end{cases}
\end{equation}
where $f$ and the functions $c_i$ are all smooth, real-valued functions on a
subset of $\mathbb{R}^n$, and $\mathcal{I}$ and $\mathcal{E}$ are two finite
sets of indices~\cite{nocedal}.

The problem discussed in this thesis has (from (\ref{eq:obj})) a convex,
quadratic, separable objective function
\[
    f(x) = x^T H x + b^T x
\]
and linear constraints. It qualifies as a convex QP problem and is formulated in
(\ref{eq:thesisqp}) as
\[
    \min_{x \in \mathbb{R}^n} f(x)\quad\textrm{subject to}~Ax = 0, ~l \leq x \leq u
\]
%\begin{equation}
%    \label{eq:thesisqp}
%    \min_{x \in \mathbb{R}^n} f(x)
%    \min_{x_{\textrm{min}} \le x \le x_{\textrm{max}}} f(x)
%    \min_{l \le x \le u} f(x)
%    \quad \textrm{subject to}
%    ~
%    Ax = 0
%    ~
%    \textrm{and}
%    ~
%    l \le x \le u
%\end{equation}
where $H$ is a positive semidefinite diagonal $n \times n$ matrix, $A$ is a
$m \times n$ oriented incidence matrix, and $b, l, u$ and $x$ are vectors in
$\mathbb{R}^n$.

The function $f$ is called the objective function of
the problem. In (\ref{eq:obj}) and (\ref{eq:thesisqp}), $f$ consists of a
quadratic term ($x^T H x$) and a linear term ($b^T x$).

In practical applications, a large proportion of the diagonal elements of $H$
and $b$ are zero.
For the non-zero elements of $H$, we typically have
$10^{-5} \le h_i \le 10^{-1}$.
Here $h_i$ denotes the $i$th diagonal element of $H$.
For the non-zero elements $b_i$ of vector $b$, we typically have
$10 \le |b_i| \le 70$. All elements of $l$ and $u$ are non-positive 
and non-negative, respectively. However, the methods developed in this thesis
do not require these values.
These values come part from conversations with Goodtech, and
part from a couple of real instances that they have supplied.
The next section describes the supplied instances in detail.

\section{Real Instances}
\label{sec:instances}
\begin{figure}[h!]
\begin{center}
\input{include/newhistH}
\end{center}
\caption{Value distribution of non-zero elements in $H$}
\label{fig:histH}
\end{figure}

\section{The Instances}
The instances have a lot in common, but most importantly they all:
\begin{enumerate}
\item Have very large coefficients in the linear term compared to the
quadratic term.
\item All cross-product coefficients are zero in the quadratic term, i.e. in
$x^T H x$, $H$ is diagonal.
\end{enumerate}

The instances are called---based on size---\textit{small}, \textit{large}
and \textit{vlarge}.

Figure \ref{fig:histH} shows the value distribution of non-zero elements in $H$
in the three instances. Note that among all the diagonal elements of H, more
than 50 percent of them are zero, and among the non-zero elements, most of them
are less than $10^{-2}$.
Observing that no non-zero element in the linear term in any of the instances
has absolute value less than 20 reinforces our earlier motivation and
cause to look into SLP.

\begin{table}
    \centering
    \caption{Problem size of each instance}
    \begin{tabular}{lrrr}
    Problem size & \textit{small} & \textit{large} & \textit{vlarge} \\\hline
    Rows         & 82             & 328            & 1127 \\
    Columns      & 238            & 952            & 3437 \\
    Non-zeroes A & 348            & 1392           & 4840 \\
    Non-zeroes H & 108            & 432            & 894 \\
    \end{tabular}
    \label{table:sizes}
\end{table}

Let us take a look at some hard statistics of each instance.
Table \ref{table:sizes} shows the problem size of each instance.
Table \ref{table:maxmin} shows some statistics about the non-zero elements in
the objective function for $i = 1,2,\ldots,n$ where $n$ is the number of
columns.
\begin{table}
    \centering
    \caption{Non-zero values in the objective function of each instance}
    \begin{tabular}{lrrr}
                        & \textit{small}          & \textit{large}          & \textit{vlarge} \\\hline
    $\max(h_{ii})$      & $2.9614 \times 10^{-2}$ & $2.9614 \times 10^{-2}$ & $4.9011 \times 10^{-2}$ \\
    $\min(h_{ii})$      & $4.9290 \times 10^{-5}$ & $4.9290 \times 10^{-5}$ & $1.1026 \times 10^{-5}$ \\
$\textrm{mean}(h_{ii})$ & $5.2864 \times 10^{-3}$ & $5.2864 \times 10^{-3}$ & $5.8984 \times 10^{-3}$ \\
    $\max(b_{i})$       & 20                      & 20                      & 20 \\
    $\min(b_{i})$       & -70                     & -70                     & -50 \\
    \end{tabular}
    \label{table:maxmin}
\end{table}
The rows and columns in each problem represent vertices and edges respectively.
An avid reader might notice that \textit{small} and \textit{large} are very
much alike.
That is entirely justified as \textit{large} is four \textit{small}-networks
connected forming one large network.

For each of these instances, GoodTech AS wants to solve several other instances
with slight variations.
These instances are from here on called \emph{subinstances}.
If one or more variables in an instance $\mathcal{Q}$ is forced to zero to form
a new instance, the newly formed instance is a subinstance of $\mathcal{Q}$.
GoodTech AS wants to simulate that all combinations of edges in the network
falls down, by forcing their corresponding variable values to zero.
This means that for each of the three instances described in this chapter,
they ideally want to solve a total of $2^n$ subinstances.
Solving all $2^n$ subinstances is  impossible.
It is unlikely that all edges in the network break down.
A more realistic approach is to try to solve all subinstances with no more than
$k$ break-downs, i.e. we solve $\sum_{j=0}^k {n \choose j}$ subinstances
with less than or equal to $k$ break-downs.

\section{Subinstances}
\section{The Subinstances}
For one or more given variables, their lower and upper bounds are set to zero,
i.e. forcing the variables to zero.
Let us denote the set of variables that are to be forced to zero by
$\mathcal{M}_k$ for some $1 \leq k \leq 2^n - 1$ where $n$ is the number
of edges in the network.
Forcing these variables to zero in some instance $\mathcal{Q}$ forms
a subinstance $\mathcal{Q}_k$. We refer to $\mathcal{M}_k$ as a
\emph{modifier} of $\mathcal{Q}$.
To simplify notation from here on, we let
$\mathcal{Q}_0 = \mathcal{Q}$ and loosen the terminology of
\emph{subinstance} so that $\mathcal{Q}_0$ can be called a subinstance of
$\mathcal{Q}$ where its modifier $\mathcal{M}_0 = \emptyset$.

To distinguish between optimal solutions over several subinstances, we denote
an optimal solution of $\mathcal{Q}_k$ as $x_k^*$.
In any solution of any instance there might be variables that are zero.
We denote the set of variables that are zero in an \emph{optimal} solution
$x_k^*$ by $\mathcal{Z}_k$.
If a modifier $\mathcal{M}_k$ only has variables that are zero in $x_0^*$,
i.e. if $\mathcal{M}_k \subseteq \mathcal{Z}_0$, then $x_k^* = x_0^*$. However,
if $\mathcal{M}_k$ has some variables that are non-zero in $x_0^*$, then
we solve $\mathcal{Q}_k$ to achieve its solution $x_k^*$ along with the set
$\mathcal{Z}_k$.

Generally, for each $\mathcal{Q}_k$ and its modifier $\mathcal{M}_k$ we
have that $x_l^* = x_k^*$ for all $l=1,2,\ldots,2^n-1$ if
$\mathcal{M}_l = S \cup \mathcal{M}_k$ for some $S \subseteq \mathcal{Z}_k$.
That is, instances $\mathcal{Q}_k$ and $\mathcal{Q}_l$ have identical optimal
solutions if $\mathcal{M}_l$ is obtained by extending $\mathcal{M}_k$ only with
variables that are zero in the optimal solution to $\mathcal{Q}_k$.


%\endgroup

%\onecolumn
%\newpage
%\twocolumn

\bibliography{thesis}{}
\bibliographystyle{ieeetr}

\onecolumn
\appendix
\section{Codes}
\subsection{\texttt{bitset} vs \texttt{set}}
\label{app:setbench}
\begin{verbatim}

#define N 65
#define SIZE 32*N

static set<uint16_t>* copy(const bitset<SIZE>* a) {
  set<uint16_t>* res = new set<uint16_t>;
  for (uint16_t i = 0; i < SIZE; i++) {
    if (a->test(i)) res->insert(i);
  }
  return res;
}

int main() {
  srand((unsigned int)time(NULL));

  vector<bitset<SIZE>*> bitsets;

  for (int i = 0; i < 3000; i++) {
    stringstream ss;
    for (int j = 0; j < N; j++) {
      bitset<32> bs(rand());
      ss << bs.to_string();
    }
    bitsets.push_back(new bitset<SIZE>(ss.str()));
  }

  vector<set<uint16_t>*> sets;
  for (int i = 0; i < 3000; i++) {
    sets.push_back(copy(bitsets[i]));
  }

  bool fool;

  double t1 = omp_get_wtime();
  for (uint16_t i = 0; i < sets.size(); i++) {
    for (uint16_t j = 0; j < sets.size(); j++) {
      fool = isSubset(*sets[i], *sets[j]);
    }
  }
  double t2 = omp_get_wtime();

  for (uint16_t i = 0; i < bitsets.size(); i++) {
    bitset<SIZE> notb = ~(*bitsets[i]);
    for (uint16_t j = 0; j < bitsets.size(); j++) {
      fool = isSubset_bit(notb, *bitsets[j]);
    }
  }
  double t3 = omp_get_wtime();

  cout << "(" << SIZE << ", " << (t2-t1) << ")\n";
  cout << "(" << SIZE << ", " << (t3-t2) << ")\n";
}

\end{verbatim}

\subsection{next\_combination}
\label{app:nextcombination}
\begin{verbatim}
template <class BidIt>
inline bool next_combination(
BidIt n_begin, BidIt n_end,
BidIt r_begin, BidIt r_end)
{
  bool boolmarked=false;
  BidIt r_marked;

  BidIt n_it1=n_end;
  --n_it1;


  BidIt tmp_r_end=r_end;
  --tmp_r_end;

  for(BidIt r_it1=tmp_r_end;
  r_it1!=r_begin || r_it1==r_begin;
  --r_it1,--n_it1)
  {
    if(*r_it1==*n_it1 )
    { 
      if(r_it1!=r_begin)
      {
        boolmarked=true;
        r_marked=(--r_it1);
        ++r_it1;
        continue;
      }
      else
        return false;    
    }
    else
    {
      if(boolmarked==true)
      {
        BidIt n_marked;
        for (BidIt n_it2=n_begin;
        n_it2!=n_end;++n_it2)
        {
          if(*r_marked==*n_it2) {
            n_marked=n_it2;break;
          }
        }


        BidIt n_it3=++n_marked;  
        for  (BidIt r_it2=r_marked;
        r_it2!=r_end;++r_it2,++n_it3)
        {
          *r_it2=*n_it3;
        }
        return true;
      }
      for(BidIt n_it4=n_begin;
      n_it4!=n_end; ++n_it4)
        if(*r_it1==*n_it4)
        {
          *r_it1=*(++n_it4);
          return true;       
        }
    }
  }  

  return true;//will never reach
}
\end{verbatim}

\subsection{Generate Random Instance}
\label{app:random}
\begin{verbatim}
static CoinPackedMatrix packMatrix(double** m, int rows, int cols) {
    std::vector<int> row_ind_vec;
    std::vector<int> col_ind_vec;
    std::vector<double> ele_vec;
    CoinBigIndex numels = 0;

    for (int i = 0; i < rows; i++) {
        for (int j = 0; j < cols; j++) {
            double num = m[i][j];
            if (num != 0.0) {
                row_ind_vec.push_back(i);
                col_ind_vec.push_back(j);
                ele_vec.push_back(num);
                numels++;
            }   
        }   
    }   

    const int *rowIndices = &row_ind_vec[0];
    const int *colIndices = &col_ind_vec[0];
    const double *elements = &ele_vec[0];

    CoinPackedMatrix cpm(false, rowIndices, colIndices, elements, numels);
    cpm.setDimensions(rows, cols);

    return cpm;
}

static double g_randd(double min, double max) {
    double num = (double) rand() / RAND_MAX;
    return (min + (num * (max - min)));
}

static int g_randi(int min, int max) {
    return (rand() % (max-min) + min);
}

ClpModel randomInstance(int vertices, int edges, double Hzero, double bzero) {
    srand((uint16_t)time(NULL));

    /* Matrix A */
    double** m = (double**) malloc(vertices*sizeof(double*));
    for (int i = 0; i < vertices; i++)
        m[i] = (double*) calloc(edges,sizeof(double));

    int row = 0;
    for (int i = 0; i < edges; i++) {
        m[row][i] = 1;
        int r = row;
        while (r == row) row = rand() % vertices;
        m[row][i] = -1;
    }
    CoinPackedMatrix A = packMatrix(m, vertices, edges);

    for (int i = 0; i < vertices; i++)
        free(m[i]);
    free(m);

    /* Matrix H */
    std::vector<int> row_ind_vec;
    std::vector<int> col_ind_vec;
    std::vector<double> ele_vec;
    int numels = 0;

    for (int i = 0; i < edges; i++) {
        /* If not 0 */
        if (((double)rand() / RAND_MAX) >= Hzero) {
            double num = g_randd(0.00001, 0.01);
            row_ind_vec.push_back(i);
            col_ind_vec.push_back(i);
            ele_vec.push_back(num);
            numels++;
        }
    }
    const int *rowIndices = &row_ind_vec[0];
    const int *colIndices = &col_ind_vec[0];
    const double *elements = &ele_vec[0];

    CoinPackedMatrix H(false, rowIndices, colIndices, elements, numels);
    H.setDimensions(edges, edges);

    /* Vector b */
    double* b = (double*) calloc(edges,sizeof(double));

    for (int i = 0; i < edges; i++) {
        /* If not 0 */
        if (((double)rand() / RAND_MAX) >= bzero) {
            double num = g_randi(10, 70);
            if (num >= 30) num = -num;
            b[i] = num;
        }
    }

    /* Vector lb */
    double* lb = (double*) malloc(edges*sizeof(double));
    for (int i = 0; i < edges; i++) {
        lb[i] = -g_randd(0.0, 1000.0);
    }

    /* Vector ub */
    double* ub = (double*) malloc(edges*sizeof(double));
    for (int i = 0; i < edges; i++) {
        ub[i] = g_randd(0.0, 1000.0);
    }

    ClpSimplex model;

    model.loadProblem(A, lb, ub, b, 0, 0);
    model.loadQuadraticObjective(H);

    free(b);
    free(lb);
    free(ub);

    return model;
}
\end{verbatim}

\section{Tables}
\subsection{Results From Experiment 1}
\label{app:exp1}
\begin{tabular}{lrr}
     $n$ & cClp               & nClp \\ \hline
     100 &   0.1              &   0.2 \\
     200 &   0.5              &   0.7 \\
     300 &   1.3              &   1.7 \\
     400 &   2.8              &   3.5 \\
     500 &   4.9              &   5.9 \\
     600 &   7.7              &  10.9 \\
     700 &  13.2              &  15.6 \\
     800 &  20.3              &  26.2 \\
     900 &  29.8              &  36.7 \\
    1000 &  42.1              &  53.0 \\
    1100 &  57.2              &  75.4 \\
    1200 &  76.4              & 107.4 \\
    1300 & 107.4              & 137.0 \\
    1400 & 141.7              & 180.3 \\
    1500 & 181.5              & 234.5 \\
    1600 & 216.7              & 300.6 \\
    1700 & 295.8              & 381.4 \\
    1800 & 366.5              & 483.6 \\
    1900 & 448.0              & 611.5 \\
    2000 & 547.1              & 710.2
\end{tabular}


\end{document}
