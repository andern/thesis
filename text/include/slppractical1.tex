\begin{figure}[htbp]
\scalebox{1}{
\begin{minipage}{0.5\linewidth}
\centering
\begin{tikzpicture}
    % grid and axes
    \draw[->,name path=xaxis] (-0.2,0) -- (2.2,0) node[right] {$x$};
    \draw[->,name path=yaxis] (0,-0.2) -- (0,2.2) node[above] {$y$};

    % draw lines
    \draw[name path=line2, domain=-0.1:1.7] plot(\x,{0.33*\x})
                                                    node[above right=0.2cm] {};
    \draw[name path=line3, domain=0.9:1.7] plot(\x,{\x - 1})
                                                    node[above right=0.2cm] {};

    % calculate intersection points
    \node[name intersections={of=xaxis and yaxis}] (a) at (intersection-1) {};
    \node[name intersections={of=line2 and line3}] (b) at (intersection-1) {};
    \node[name intersections={of=xaxis and line3}] (c) at (intersection-1) {};

    % write the coordinates of the corners.
    \path let \p0 = (a) in node [left=-0.0cm of a] {\printpoint{\x0}{\y0}};
    \path let \p0 = (b) in node [right=0.15cm of b] {(1.5,0.5)};
    \path let \p0 = (c) in node [below=-0.2cm of c] {\printpoint{\x0}{\y0}};

    \node at (1, 1) (xu) {};
    \node at (0, 0) (xhat) {};
    \node at (1.2, 0.4) (xstar) {};
    \node at (1.5,0.5) (x0) {};
    \node at (1.95, 1.3) (xudesc) {$x^u$};
    \node at (0.4, 0.4) (xstardesc) {$x^*$};
    \node[red] at (-0.4, -1.0) (xhatdesc) {$\hat{x}_0$};
    \node at (1.8, -0.3) (x0desc) {$x_0$};
    \draw[->] (xudesc) .. controls ([xshift=-1cm] xudesc)
                                        and ([xshift=1cm] xu) .. (xu);
    \draw[->] (x0desc) .. controls ([yshift=0.1cm] x0desc)
                                        and ([xshift=0.5cm] x0) .. (x0);
%    \draw[->] (loptdesc) .. controls ([xshift=0.5cm] loptdesc)
%                                        and ([xshift=-0.5cm] lopt) .. (lopt);
    \draw[->] (xhatdesc) .. controls ([yshift=1cm] xhatdesc)
                                        and ([yshift=-1cm] xhat) .. (xhat);
    \draw[->] (xstardesc) .. controls ([xshift=1cm] xstardesc)
                                        and ([xshift=-1cm] xstar) .. (xstar);
    
    \draw[fill] (xu) circle [radius=0.02];

    \draw[fill] (xstar) circle [radius=0.02];

    % draw the big polygin
%    \fill[very thick,fill=blue,fill opacity=0.3] (a.center) -- (b.center)
%                                              -- (c.center) -- (d.center)
%                                              -- (e.center) -- cycle;

    \fill[very thick,fill=blue,fill opacity=0.3] (a.center) -- (b.center)
                                              -- (c.center) -- cycle;

%    \draw[thick, red, name path=lobj, domain=0.48:1.28] plot(\x, {-0.625*\x + 1.35})
%                                                                            {};

    % draw the quadratic objective function
    \draw[thin, dashed] plot[id=qobj1, raw gnuplot] function {
        f(x,y) = x**2 + y**2 - 2*x - 2*y + 1.75;
        set xrange[-1:2];
        set yrange[-2:2];
        set view 0,0;
        set isosamples 1000,1000;
        set size square;
        set cont base;
        set cntrparam levels incre 0,0.1,0;
        unset surface;
        splot f(x,y);
    };
    \draw[thin, dashed] plot[id=qobj2, raw gnuplot] function {
        f(x,y) = x**2 + y**2 - 2*x - 2*y + 1.9;
        set xrange[-1:2];
        set yrange[-2:2];
        set view 0,0;
        set isosamples 1000,1000;
        set size square;
        set cont base;
        set cntrparam levels incre 0,0.1,0;
        unset surface;
        splot f(x,y);
    };
    \draw[thin, dashed] plot[id=qobj3, raw gnuplot] function {
        f(x,y) = x**2 + y**2 - 2*x - 2*y + 1.98;
        set xrange[-1:2];
        set yrange[-2:2];
        set view 0,0;
        set isosamples 1000,1000;
        set size square;
        set cont base;
        set cntrparam levels incre 0,0.1,0;
        unset surface;
        splot f(x,y);
    };
\end{tikzpicture}
\end{minipage}
}
\scalebox{0.9}{
\begin{minipage}{0.5\linewidth}
\centering
\[
\begin{array}{lcrcrl}
    \textrm{Maximize}   &-&   x &+&   y \\
    \textrm{subject to} & &   x &+&   y & \leq 3 \\
    \textrm{and}        & &   x &-&   y & \leq 1 \\
    \textrm{and}        &-&   x &+& 3 y & \leq 0 \\
    \textrm{and}        & &   x &,&   y & \geq 0
\end{array}
%    \begin{array}{rcrcrcr}
 %       \zeta &=& 1.01 &-& 0.10 w_3 &-& 0.10 w_1 \\ \hline
 %           y &=& 0.75 &-& 0.25 w_3 &-& 0.25 w_1 \\
 %           x &=& 2.25 &+& 0.25 w_3 &-& 0.75 w_1 \\
 %         w_2 &=&-0.50 &-& 0.50 w_3 &+& 0.50 w_1
 %   \end{array}
\]
\end{minipage}
}
\caption{A linearization of the new QP}
\label{fig:prac1}
\end{figure}
