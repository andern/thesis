\begin{tikzpicture}
    % grid and axes
    \draw[scale=1.5,->,name path=xaxis] (-0.2,0) -- (2.2,0) node[right] {$x$};
    \draw[scale=1.5,->,name path=yaxis] (0,-0.2) -- (0,2.2) node[above] {$y$};

    % draw lines
    \draw[scale=1.5,name path=line1, domain=0.6:2.5] plot(\x,{-\x + 3})
                                                    node[above right=0.2cm] {};
    \draw[scale=1.5,name path=line2, domain=-0.2:1.6] plot(\x,{0.33*\x + 1.33})
                                                    node[above right=0.2cm] {};
    \draw[scale=1.5,name path=line3, domain=0.7:2.3] plot(\x,{\x - 1})
                                                    node[above right=0.2cm] {};
    % calculate intersection points
    \node[name intersections={of=line1 and line2}] (a) at (intersection-1) {};
    \node[name intersections={of=line1 and line3}] (b) at (intersection-1) {};
    \node[name intersections={of=xaxis and line3}] (c) at (intersection-1) {};
    \node[name intersections={of=xaxis and yaxis}] (d) at (intersection-1) {};
    \node[name intersections={of=line2 and yaxis}] (e) at (intersection-1) {};

    % draw the big polygin
    \fill[very thick,fill=tleblue] (a.center) -- (b.center) --
                                                 (c.center) -- (d.center) --
                                                 (e.center) -- cycle;

    % write the coordinates of the corners.
    \path let \p0 = (a) in node [above right=0.0cm of a] {($1.25$, $1.75$)};
    \path let \p0 = (b) in node [right=0.1cm of b] {($2$, $1$)};
    \path let \p0 = (c) in node [below right=-0.35cm of c] {($1$, $0$)};
    %\path let \p0 = (d) in node [left=0.0cm of d] {($0$, $0$)};
    \path let \p0 = (e) in node [left=0.0cm of e] {($0$, $1.\overline{3}$)};

    \node at (1.5, 1.5) (qopt) {};
    \node at (3, 1.5) (lopt) {};
    \node at (-0.75, 1.125) (qoptdesc) {$x^*$};
    \node at (-1.5, 0.375) (x0desc) {$x_0$};
    \node[text=red] at (3.75, 2.625) (loptdesc) {$\hat{x}_0$};
    \draw[scale=1.5,->] (qoptdesc) .. controls ([xshift=1cm] qoptdesc) and ([xshift=-1cm]
                                                               qopt) .. (qopt);
    \draw[scale=1.5,->] (loptdesc) .. controls ([xshift=-1cm] loptdesc) and ([yshift=1cm]
                                                               lopt) .. (lopt);
    \draw[scale=1.5,->] (x0desc) .. controls ([xshift=1cm] x0desc) and ([xshift=-1cm] d)
                                                                     .. (d);

    \draw[scale=1.5,fill] (qopt) circle [radius=0.02];

    \draw[scale=1.5,thick, red, name path=lobj, domain=-0.4:0.4] plot(\x, {-\x}) {};

    % draw the quadratic objective function
    \draw[scale=1.5,thin, dashed] plot[id=qobj1, raw gnuplot] function {
        f(x,y) = x**2 + y**2 - 2*x - 2*y + 1.75;
        set xrange[-1:2];
        set yrange[-2:2];
        set view 0,0;
        set isosamples 1000,1000;
        set size square;
        set cont base;
        set cntrparam levels incre 0,0.1,0;
        unset surface;
        splot f(x,y);
    };
    \draw[scale=1.5,thin, dashed] plot[id=qobj2, raw gnuplot] function {
        f(x,y) = x**2 + y**2 - 2*x - 2*y + 1.9;
        set xrange[-1:2];
        set yrange[-2:2];
        set view 0,0;
        set isosamples 1000,1000;
        set size square;
        set cont base;
        set cntrparam levels incre 0,0.1,0;
        unset surface;
        splot f(x,y);
    };
    \draw[scale=1.5,thin, dashed] plot[id=qobj3, raw gnuplot] function {
        f(x,y) = x**2 + y**2 - 2*x - 2*y + 1.98;
        set xrange[-1:2];
        set yrange[-2:2];
        set view 0,0;
        set isosamples 1000,1000;
        set size square;
        set cont base;
        set cntrparam levels incre 0,0.1,0;
        unset surface;
        splot f(x,y);
    };
\end{tikzpicture}
