\documentclass[a4paper,twocolumn]{report}
\usepackage[utf8]{inputenc}
\begin{document}
\begin{enumerate}
\item Velkommen til presentasjon av masteroppgaven min
\item Fortelle litt om hvordan oppgaven har blitt til, eller oppstått
\item Goodtech og Mathconsult har utviklet et program PROMAPS, som kalkulerer
      leveransepåliteligheten i et nettverk ved å kalkulere risiko ved utfall
      av grener i nettverket
\item Kommer tilbake til dette
\item Dette er formulert som et QP-problem, eller en rekke veldig like
      QP-problemer
\item Det viser seg at QP-løseren er flaskehals. Og derfor oppdaterer
      skjermbildet seg hvert 5. minutt
\item Vi skal se på et skjermbilde av PROMAPS her i en artikkel fra Teknisk
      Ukeblad
\item Tar en titt på QP-problemet
\item Objektfunksjonen representerer leveransekostnader, det her er da
      pengeenhet per sekund, eller penger per sekund
\item $x$ er en ukjent, og representerer antall Watt vi sender over hver gren
\item $\Phi$ representerer strømtap over hver gren i Watt
\item $D$ er kostnader for å sende strøm over hver gren, penger per Joule
\item $g$ er kostnader for å generere strøm i penger per Joule
\item $c$ er leveransepris i penger per Joule, dette er da altså inntekt
\item Vi forkorter dette til en mer gjenkjennelig objektfunksjon
\item Her representerer både $H$ og $b$ kostnader. $H$ er en diagonalmatrise og
      er positivt semidefinitt, som betyr at vi jobber med et konveks
      QP-problem
\item Vi definerer et QP-problem min... underlagt $Ax = 0$, og $x$ mellom $l$
      $u$. $l$ og $u$ er nedre og øvre grenkapasitet i Watt
\item Goodtech vil løse QP-problemet vi definerte, bare med utfall i
      forskjellige grener. Vi modellerer dette ved å sette $l_i=u_i=0$
\item Vi har et QP problem $\mathcal{Q}$ som definert, uten utfall, som vi
      kaller en instans
\item Så har vi også subinstanser $\mathcal{Q}_k$ som er en instans med et
      eller flere utfall
\item Vi vil helst løse så mange subinstanser som mulig, for å få mest mulig
      nøyaktig analyse
\item For et problem med $n$ grener, er det $2^n$ subinstanser. Men usannsynlig
      det er mange utfall
\item Prøver å begrense antall subinstanser ved å sette en realistisk grense
      for antall utfall
\item Kalkulerer antall utfall ved funksjonen sigma
\item Hver variabel representerer en gren, så vi har en mengde med variabler
      som representerer utfall, som vi noterer $\mathcal{M}_k$, for
      modifikator
\item Vi har en en-til-en korrenspondanse mellom kombinasjoner av utfall og
      deres indekser
\item Vi har en subinstans $\mathcal{Q}_k$ som defineres av $\mathcal{Q}$ og
      $\mathcal{M}_k$, og vi noterer dens optimale løsning $x_k^*$
\item Jeg fikk tilsendt tre instanser av Goodtech, small, large og vlarge. Her
      ser vi verdiene på diagonalen til $H$, etter kolonne
\item Vi ser at alle verdiene er lavere enn $10^{-1}$, og at de fleste verdiene
      er lavere enn $10^{-2}$
\item Videre så ser vi her størrelsen på de tre instansene, og merk at over
      $50\%$ av diagonalelementene i $H$ er $0$ i alle tre instansene
\item Et viktig poeng å dra fram her er at verdiene i det lineære leddet er mye
      høyere enn verdiene i det kvadratiske leddet, så det gir oss motivasjon
      til å se på metoder basert på lineær programmering
\item Hva skjer hvis vi bare ignorerer det kvadratiske leddet? Har det stor
      innflytelse på optimal verdi?
\item Vi noterer en lineær Taylor-utvikling av $f$ i punktet $a$ for $T_a$,
      og da er $T_0 = b^T x$
\item Vi definerer et LP program hvor vi minimerer det lineære leddet,
      underlagt de samme sidekravene som i QP-problemet
\item Kort fortalt her så noterer vi avviket mellom optimal løsning til
      $\mathcal{L}$ og $\mathcal{Q}$ for $\Delta$
\item Vi ser her et 3D-plot som viser avviket mellom $\mathcal{L}$ og
      $\mathcal{Q}$ som en funksjon av tettheten i objektfunksjonen
\item Vi ser at tettheten i $H$ har veldig lite innflytelse på avviket
\item Vi ser også at $b$ har mye større innflytelse på avviket, men legg
      merke til at avviket er aldri større enn $5\%$, det er faktisk såvidt
      over $4\%$
\item På grunn av dette kommer altså successive linear programming inn i
      bildet. Vi vil oppnå $95\%$ av optimal verdi etter første iterasjon
      hvis vi begynner i $0$
\item Her har vi selge algoritmen, vi gjør først en taylorutvikling i det
      punktet vi står i, så løser vi $\mathcal{L}$
\item Så gjør vi et LINJESØK mellom punktet vi står i og optimale løsning til
      $\mathcal{L}$. Vi finner altså optimal målfunksjonsverdi av alle punktene
      på linja mellom de to endepunktene
\item Så flytter vi oss til det punktet vi fant i linjesøket, og fortsetter
      algoritmen helt til vi når termineringskravet vårt
\item Et eksempel. Vi minimerer dette problemet. Det optimalet punktet er her
      $x = 1$ og $y = 1$ hvor optimale objektverdi er $-2$
\item Vi ser her den lineære objektfunksjonen
\item Her har vi et bilde over det tilatte området. Det i rødt representerer
      ting som har med LP å gjøre. Vi ser her den lineære objektfunksjonen.
\item Vi gjør et linjesøk her mellom $x_0$ og $\hat{x}_0$ som vist her
\item Gjør et nytt linjesøk her, og finner da $x_2$
\item Nytt linjesøk, og havner på $x_3 = (0.96,1.02)$
\item Her ser vi stien som algoritmen tar fra startpunktet til vi terminerer
\item Etter at vi har løst en instans, så er det veldig sannsynlig at vi skal
      løse et veldig likt problem like etterpå, med en liten endring i
      høyresiden av sidekravene.
\item Hvis vi gjør det, så er det mulig at vi gjør den primale løsningen
      ikke-tillatt. Vi vet at en endring i 
\end{enumerate}
\end{document}
